\section{Diskussion der Ergebnisse}
\label{sec:Diskussion}
Für die Diskussion der Messung wird die prozentuale Abweichung $p$ der in Kapitel \ref{sec:Auswertung} bestimmten Landefaktoren $g$ 
zu dem Theoriewerten mittels
\begin{equation*}
    p=\frac{|g_\text{theo}-g_\text{exp}|}{g_\text{theo}}
\end{equation*}
berechnet. Eine Auflistung der Werte ist in Tabelle \ref{tab:p} gegeben.

\begin{table}[H]
    \centering
      \caption{Experimentell bestimme Landefaktoren $g_\text{exp}$, Theoriewerte $g_\text{theo}$ und prozentuale Abweichung $p$.}
      \label{tab:p}
      \sisetup{table-format=1.2}
      \begin{tabular}{c S S @{${}\pm{}$} S S[table-format=2.0]  @{${}\pm{}$} S[table-format=2.0] }
        \toprule
        {Spektrallinie} & {$g_\text{theo}$} & \multicolumn{2}{c}{$g_\text{exp}$} & \multicolumn{2}{c}{$p [\%]$}\\
        \midrule
        $\text{rot},\sigma$  & 1    &  0.93 & 0.22 & 7  & 22 \\
        $\text{blau},\sigma$ & 1.75 &  1.03 & 0.31 & 41 & 18 \\
        $\text{blau},\pi$    & 0.5  &  0.60 & 0.24 & 19 & 47 \\
        \bottomrule
      \end{tabular}
\end{table}
\noindent
Insbesondere stimmen die berechneten Werte für die blaue $\pi$-Linie und die rote $\sigma$-Linie im Rahmen der 
Unsicherheit mit den Theoriewerten überein. 

\noindent
Mögliche Gründe für die Abweichungen liegen vor allem in technischen Problemen des Elektromagneten und der 
Abschätzung der Abstände $\Delta s$ und $\delta s$.

\noindent
Der Elektromagnet erwärmt sich bei längerer Inbetriebnahme, was dazu führen kann, dass das Magnetfeld
teilweise oder komplett abgeschaltet wird. Die Erwärmung kann dazu führen, dass die Feldstärke während 
der Messung nicht mehr mit der in Kapitel \ref{sec:B} geeichten Feldstärke übereinstimmt. 
%Zudem können auch inhomogene Anteile des Magnetfeldes die Messung negativ beeinträchtigt haben. 
Für die optimale Auflösung der blauen $\pi$-Linie wird nach Kapitel \ref{sec:theorie} ??? ein Magnetfeld
von $B=\SI{1.25}{T}$ benötigt. Diese Feldstärke war mit dem vorhandenen Elektromagneten nicht zu erreichen.
Die Auflösung des Bildes \ref{fig:blau_pi} und somit auch die Abschätzung von $\delta s$ war dadurch 
im Gegensatz zu einer Messung bei höherem Magnetfeld erschwert. Generell stellt die Abschätzung der Längen
$\Delta s$ und $\delta s$ eine Fehlerquelle da, da die Linien vergleichsweise breit waren. 
Diese Unsicherheit ist zwar statistischer Natur, durch die geringe Anzahl an Messwerten kann das Ergebnis 
aber trotzdem negativ beeinflusst werden.

\noindent
Das Stativ, auf welchem die Kamera befestigt ist, konnte nicht zufriedenstellend fest gestellt werden, sodass
die Kamera bei moderatem Druck ihre Position veränderte. Somit ist nicht auszuschließen, dass bei der 
Aufnahme der Bilder unterscheidliche Bedingungen vorlagen. Dies könnte zu einer Verzerrung der Werte 
$\delta s$ relativ zu $\Delta s$ geführt haben.
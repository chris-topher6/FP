\section{Vorbereitung}
Ziel: Bestimmung des Kernspins der Rubidium-Isotope $^{87}Rb$ und $^{85}Rb$ durch die optisch induzierte nicht-thermische Besetzung unter
Einwirkung der Hochfrequenz-Strahlung. 

\subsection{Rubidium}
\subsubsection[]{Zu welchen Klassen der Materialien zählt das Rubidium?}
\begin{itemize}
    \item Alkalimetall\\
    \to 1 Valenzelktron
\end{itemize}
\subsubsection[]{Welchen Wert haben die Quantenzahlen: Bahndrehimpuls (L), Spin (S), Gesamtdrehimpuls der Elektronenschale (J) und Kernspin (I) eines $^{85}Rb$-Atoms im Grundzustand?}
\begin{itemize}
    \item Elektronenkonfiguration: [Kr]5s1 \\
    \to $S=1/2$, $L=0$, $J=S+L=1/2$
    \item $I_{^{85}Rb}=5/2$
    \item $I_{^{87}Rb}=3/2$
\end{itemize}
\subsubsection[]{Welche Werte ergeben sich für Gesamtdrehimpuls des Atoms (QZ F) im Grund- und in dem ersten angeregten Zustand?}
\begin{itemize}
    \item $F=J+I$
    \item $F_{^{85}Rb}\in {2,3}$
    \item $F_{^{87}Rb}\in {1,2}$
\end{itemize}

\subsection{Zeeman Effekt}
Quanten,Atome,Kerne,Teilchen Seite 228
\subsubsection[]{Welche Aufspaltung entstehen durch das äußere Magnetfeld bei den $^2S_{1/2}$ und $^2P_{1/2}$ Niveaus?}
\begin{itemize}
    \item Hyperfeinstruktur 
    \item anormale Zeeman Aufspaltung
\end{itemize}
\subsubsection[]{Wie sieht ein Niveauschema aus?}
vgl. Tablet
\subsubsection[]{Wovon hängt die Zeeman-Aufspaltung ($\Delta E_Z$) zwischen einzelnen Niveaus ab?}
\begin{itemize}
    \item $\Delta E_Z=g_F\mu_B m B=hf$
\end{itemize}

\subsection{Anregung und Emission}
\subsubsection[]{Welche Auswahlregeln gibt es für einen elektrischen Dipolübergang zwischen den beiden $^2S_{1/2}$ und $^2P_{1/2}$ Niveaus?}
\begin{itemize}
    \item $\Delta M_F=\pm 1, 0$
\end{itemize}
\subsubsection[]{Was bewirkt die Anregung dieser Zustände mit einem zirkular polarisierten Licht bei einem angelegten Magnetfeld?}
\begin{itemize}
    \item Elektronen wecheln auf ein höheres Energieniveau ($^2S_{1/2}\to ^2P_{1/2}$)
    \item $\Delta M_F=+1$, da zirkular polarisiert
\end{itemize}
\subsubsection[]{Woran liegt der Unterschied zwischen stimulierten und spontanen Emission?}
\begin{itemize}
    \item spontan: Elektron wechselt zum niedrigeren Niveau und emmitiert ein Photon
    \item induziert: Photon trifft auf angeregtes Elektron. Elektron wechselt Energieniveau\\
    \to die beiden Photonen sind kohärent und fliegen in die gleiche Richung
\end{itemize}
\subsubsection[]{Wie hängt die spontane Emission von der Frequenz ab?}
\begin{itemize}
    \item stimulierendes Photon benötigt die Energie der Energielücke
\end{itemize}

\subsection{Optisches Pumpen}
\subsubsection[]{Welcher Zustand stellt sich bei der kontinuierlichen Beleuchtung mit zirkular polarisierten Lich ein?}
\begin{itemize}
    \item Abregung mit $\Delta M_F=\pm 1, 0$ \\
    \to also im Mittel 0
    \item Anregung mit $\Delta M_F=+1$
    \item es stellt sich der niedrigste Energiezustand mit höhstem $M_F$ ein
\end{itemize}
\subsubsection[]{Was bedeutet Optisches Pumpen?}
\begin{itemize}
    \item durch optische Anregung werden Atome dauerhaft in ein höheres Energieniveau gehoben (Besetzungsinversion)
    \item Valenzelktronen werden auf den höchst möglichen $M_F$ im niedrigsten Niveau gepumpt
\end{itemize}
\subsubsection[]{Wie siet dabei der zeitliche Verlauf der Intensität des transmittierten Lichts aus?}
\begin{itemize}
    \item Tanzparen nimmt stetig zu und strebt gegen 1
    \item vgl. Tablet
\end{itemize}

\subsection{Absorptionsverhalten und Magnetfeld}
\subsubsection[]{Welche Auswahlregeln gibt es für magnetische Dipolübergänge?}
\begin{itemize}
    \item $\Delta M_F=-1$ 
    \item wirkt dem optischen Pumpen entgegen
\end{itemize}
\subsubsection[]{Was passiert in dem optisch gepumpten System, wenn durch die RF-Feld-Frequenz (Hochfrequenz) f gegebene Energie ($E_{RF}=hf$) der Zeeman Energie ($\Delta E_Z$) entspricht?}
\begin{itemize}
    \item Besetzungsinversion wird durch stimulierte Emission aufgeoben
\end{itemize}
\subsubsection[]{Was geschieht mit dem Absorptionsverhalten?}
\begin{itemize}
    \item Transparenz wird deutlich schlechter \to höheres Absorptionsverhalten
    \item 
\end{itemize}
\subsubsection[]{Überlegen Sie wie eine Absorptionspeak-vs.-Magnetfeld Abhängigkeit aussehen soll.}
\begin{itemize}
    \item vgl.Tablet
    \item Tiefpunkt bei $B_M=\frac{hf}{g_F\mu_B}$
\end{itemize}
\subsubsection[]{Wie ermittelt man dadruch den Lande-Faktor $g_F$?}
\begin{itemize}
    \item messe $B(f)$
    \item linearer Fit 
    \item Steigung: $a=\frac{h}{g_F\mu_B}$
    \item nach $g_F$ umstellen
\end{itemize}
\subsubsection[]{Was passiert mit der transmittierten Lichtintensität bei null Magnetfeld?}
\begin{itemize}
    \item Transparenz und Lichtintensität fallen ab 
    \item ohne B-Feld keine Zeeman-Aufspaltung
    \item Energieniveaus sind in $M$ entartet
    \item Valenzelktronen können beliebig oft hoch und runter springen, da Auswahlregeln nicht mehr greifen
    \item es wird also viel Licht zur Anregung verwendet
\end{itemize}

\subsection{}
\subsubsection[]{Was passiert wenn man die $E_{RF}$ gleich der $\Delta E_Z$ wählt und dabei die RF-Feld-Amplitude variiert?}
\begin{itemize}
    \item Transparenz sollte mit zunehmender Amplitude abfallen
\end{itemize}
\subsubsection[]{Welche Bezieung erwarten Sie dort zwischen der Frequenz der Rabi-Oszilationen und der Stärke des angelegten RF-Feldes?}
\begin{itemize}
    \item ???
\end{itemize}

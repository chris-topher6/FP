\section{Vorbereitung}
Ziel: Bestimmung des Kernspins der Rubidium-Isotope $^{87}Rb$ und $^{85}Rb$ durch die optisch induzierte nicht-thermische Besetzung unter
Einwirkung der Hochfrequenz-Strahlung. 

\subsection{Rubidium}
\subsubsection[]{Zu welchen Klassen der Materialien zählt das Rubidium?}
\begin{itemize}
    \item Alkalimetall
    \to 1 Valenzelktron
\end{itemize}
\subsubsection[]{Welchen Wert haben die Quantenzaheln: Bahndrehimpuls (L), Spin (S), Gesamtdrehimpuls der Elektronenschale (J) und Kernspin (I) eines $^{85}Rb$-Atoms im Grundzustand?}
\begin{itemize}
    \item Elektronenkonfiguration: [Kr]5s1
    \to $S=1/2$, $L=0$, $J=S+L=1/2$
    \item $I_\text{^{85}Rb}=5/2$
    \item $I_\text{^{87}Rb}=3/2$
\end{itemize}
\subsubsection[]{Welche Werte ergeben sich für Gesamtdrehimpuls des Atoms (QZ F) im Grund- und in dem ersten angeregten Zustand?}
\begin{itemize}
    \item $F=J+I$
    \item $F_\text{^{85}Rb}=3$
    \item $F_\text{^{87}Rb}=2$
\end{itemize}

\subsection{Zeeman Effekt}
Quanten,Atome,Kerne,Teilchen Seite 228
\subsubsection[]{Welche Aufspalung entstehen durch das äußere Magnetfeld bei den $^2S_{1/2}$ und $^2P_{1/2}$ Niveaus?}
\begin{itemize}
    \item (a)nomale Zeeman-Ausspaltung
    \to normal oder anormal?
    \item normal: Ausspaltung in $m_l$
    \item anormal: Aufspaltung in $m_J$ mt vorheriger Feinstruktur
\end{itemize}
\subsubsection[]{Wie sieht ein Niveauschema aus?}
%Hier Bild zeichnen und einfügen!
\subsubsection[]{Wovon hängt die Zeeman-Aufspalung ($\Delta E_Z$) zwischen einzelnen Niveaus ab?}
\begin{itemize}
    \item $\Delta E_Z=g_F\mu_B B=hf$
    \item $\Delta E_Z=g_F\mu_B m B=hf$ ???
\end{itemize}

\subsection{Anregung und Emission}
\subsubsection[]{Welche Auswahlregeln gibt es für einen elektrischen Dipolübergang zwischen den beiden $^2S_{1/2}$ und $^2P_{1/2}$ Niveaus?}
\begin{itemize}
    \item 
\end{itemize}
\subsubsection[]{Was bewirkt die Anregung dieser Zustände mit einem zirkular polarisierten Licht bei einem angelegten Magnetfeld?}
\subsubsection[]{Woran liegt der Unterschied zwischen stimulierten und spontanen Emission?}
\subsubsection[]{Wie hängt die spontane Emission von der Frequenz ab?}


\subsection{Optisches Pumpen}
\subsubsection[]{Welcher Zustand stellt sich bei der kontinuierlichen Beleuchtung mit zirkular polarisierten Lich ein?}
\subsubsection[]{Was bedeutet Optisches Pumpen?}
\subsubsection[]{Wie siet dabei der zeitliche Verlauf der Intensität des transmittierten Lichts aus?}

\subsection{}
\subsubsection[]{Welche Auswahlregeln gibt es für magnetische Dipolübergänge?}
\subsubsection[]{Was passiert in dem optisch gepumpten System, wen durch die RF-Feld-Frequenz (Hochfrequenz) f gegebene Energie ($E_{RF}=hf$) derZeeman Energie ($\Delta E_Z$) entspricht?}
\subsubsection[]{Was geschieht mit dem Absorptionsverhalten?}
\subsubsection[]{Überlegen Sie wie eine Absorptionspeak-vs.-Magnetfeld Abhängigkeit aussehen soll.}
\subsubsection[]{Wie ermittelt man dadruch den Lande-faktor $g_F$?}
\subsubsection[]{Was passiert mit der transmittierten Lichtintensität bei null Magnetfeld?}

\subsection{}
\subsubsection[]{Ws passiert wenn man die $E_{RF}$ gleich der $\Delta E_Z$ wählt und dabei die RF-Feld-Amplitude variiert?}
\subsubsection[]{Welche Bezieung erwarten Sie dort zwischen der Frequenz der Rabi-Oszilationen und der Stärke des angelegten RF-Feldes?}
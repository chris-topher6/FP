\section{Auswertung der Messdaten}
\label{sec:Auswertung}

\subsection{Kompensation des Erdmagnetfeldes}
Das Erdmagnetfeld wurde mit einem Spulenstrom von $I = \SI{0.231}{\ampere}$
ausgeglichen. Die Parameter der verwendeten Spule sind in Tabelle~\ref{tab:spulendaten}
aufgeführt.
\begin{table}
  \centering
  \caption{Spulenparameter der Helmholtz-Spulen.}
  \label{tab:spulendaten}
  \sisetup{table-format=10.0}
  \begin{tabular}{S S[table-format=3.0] S[table-format=1.5]}
    \toprule
    {Spule} & {Windungszahl $N$} & {Radius $r/\si{\meter}$} \\
    \midrule
    {\text{Sweep}}      &  11 & 0.16390 \\
    {\text{Horizontal}} & 154 & 0.15790 \\
    {\text{Vertikal}}   &  20 & 0.11735 \\
    \bottomrule
  \end{tabular}
\end{table}
Mit der Gleichung~\ref{eqn:helmholtz} kann die magnetische Feldstärke einer
Helmholtzspule berechnet werden. Die vertikale Komponente des
Erdmagnetfeldes ergibt sich demnach zu
\begin{equation}
  B_{v} \approx  \SI{13.9402498}{\micro\tesla}.
  \label{eqn:erdmagnetfeld}
\end{equation}
Der Literaturwert beträgt $B_{v \text{lit}} = \SI{45.355}{\micro\tesla}$~\cite{gfzpotsdam}.

\subsection{Bestimmung der Land\`e-Faktoren}
\label{subsec:lande}
Nach Gleichung~\ref{eqn:helmholtz} lassen sich aus den gemessenen Stromstärken die
Magnetfeldstärken berechnen. Das gesamte
Magnetfeld setzt sich zusammen aus dem Feld der \textit{Sweep}-Spule und dem Feld
der Horizontalspule. Das gesamte Magnetfeld in Abhängigkeit von der RF-Frequenz
ist in Abbildung \ref{fig:magnetfeld1} dargestellt.

\begin{figure}[H]
  \centering
  \includegraphics[scale=0.6]{build/plot1.pdf}
  \caption{Das gesamte Magnetfeld in Abhängigkeit von der Frequenz.}
  \label{fig:magnetfeld1}
\end{figure}
\noindent
In Abbildung \ref{fig:magnetfeld1} wurde eine lineare Regression durchgeführt.
Aus der Steigung dieser kann nach Gleichung \ref{eqn:bf} der gyromagnetische
Faktor bestimmt werden, indem diese zu Gleichung \ref{eqn:gyrom} umgestellt wird.
\begin{equation}
  g_{F} = \frac{h}{m \mu_{B}}.
  \label{eqn:gyrom}
\end{equation}
\noindent
Es ergeben sich die Regressionsparameter
\begin{align}
  m_{87}  & =  \SI{1.422(0.029)e-10}{\tesla\per\hertz}\\
  m_{85}  & =  \SI{2.06(0.12)e-10}{\tesla\per\hertz}.
\end{align}
\noindent
Mit diesen Werten ergeben sich die gyromagnetischen Faktoren
\begin{align}
	g_{87} & = \num{0.502+-0.01}\\
  g_{85} & = \num{0.346+-0.021}.
\end{align}
\noindent

\subsection{Bestimmung des Kernspins der Isotope}
\label{subsec:kernspin}
Unter Verwendung des in Abschnitt \ref{subsec:lande} ermittelten gyromagnetischen
Faktors kann für das jeweilige Isotop der Kernspin ermittelt werden. Da für den
gyromagnetischen Faktor Gleichung \ref{eqn:gyrogyro} gilt, kann durch Einsetzen
der bekannten Werte
\begin{align}
S =\ & \frac{1}{2}, & L =\ & 0, & J =\ & \frac{1}{2}, & F =\ & I + J
\end{align}
der Kernspin ermittelt werden.

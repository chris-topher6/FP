\section{Diskussion der Ergebnisse}
\label{sec:Diskussion}
Die ermittelte Vertikalkomponente des Erdmagnetfeldes weist eine Abweichung von $\Delta B_{v} \approx \SI{69.26}{\percent}$
auf. Diese hohe Abweichung ist vermutlich durch die Messung in einem Gebäude aus Stahlbeton sowie der nur grob
erfolgten Abschätzung durch die Experimentierenden erklären. Die Abweichungen für die Bestimmung des Kernspins
sind mit $\Delta I_{87} =\  \SI{0.7}{\percent}$ bzw. $ \Delta I_{85} =\  \SI{5}{\percent}$ sehr gering, die Messung scheint
nur geringfügige Ungenauigkeiten aufzuweisen. Die Qualität der Messung könnte jedoch durch die Verwendung eines
vollständig abgedunkelten Raumes sowie durch die exaktere Messung der Magnetfelder durch eine Hall-Sonde weiter
verbessert werden. Die Vernachlässigung des Zeeman-Effektes ist jedoch eine gute Näherung (siehe Abschnitt \ref{subsec:zeemanquadrat}).

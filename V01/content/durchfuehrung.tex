\section{Durchführung des Versuches}
\label{sec:Durchführung}

\subsection{Aufbau des Versuches}
\label{subsec:Aufbau}

Organische Szintillatoren haben gegenüber anorganischen Szintillatoren den Vorteil, dass sie eine sehr
gute Zeitauflösung besitzen. Da eine gute Zeitauflösung instrumental für die Qualität der genommenen
Daten ist, wird in diesem Versuch ein organischer Szintillator verwendet.
Dieser Szintillator befindet sich in einem Detektortank mit einem Volumen $V = \SI{50}{\liter}$.
Der Lichtpuls des Szintillators wird durch einen Photovervielfacher (im englischen Sprachraum: Photomultipliertube,
daher oft abgekürzt als PMT) in ein elektrisches Signal umgewandelt und verstärkt.
Dabei können durch thermische Effekte Elektronen emittiert werden, welche die
Messung verfälschen würden. Daher werden zwei PMTs verwendet, welche durch eine
Koinzidenzschaltung miteinander verbunden sind. So werden nur Signale verwendet, welche
von beiden PMTs nachezu gleichzeitig ausgesandt werden, was Fehler durch thermische
Effekte vermeidet.
Die Schaltung ist in folgender Abbildung dargestellt.

\begin{figure}
  \centering
    \includegraphics[width=0.5\textwidth]{pictures/Schaltbild.png}
    \label{fig:Schaltbild}
    \caption{Schaltbild des Versuchsaufbaus.}
\end{figure}

Das Signal der PMTs wird über eine Verzögerungsleitung, welche schaltungstechnische Unterschiede
wie z.b. Kabelwege ausgleicht, durch einen Diskriminator geleitet. Dieser lässt nur Signale ab
einer gewissen Schwelle hindurch. Diese Schwelle lässt sich einstellen, wodurch Rauschen
minimiert werden kann. Das elektrische Signal, welches von dem Diskriminator ausgegeben wird,
kann auf verschiedene Breiten eingestelt werden. Hiernach folgt die bereits erwähnte Koinzidenz.
Die nun folgende Schaltung funktioniert wie eine Stoppuhr, welche das Warten auf ein Zerfallssignal
nach einer Suchzeit $T_{\text{S}}$ abbricht, was nötig wird, wenn das Myon nicht innerhalb des Detektors
zerfällt. Dies ist realisiert mittels zweier AND-Gatter, welche beide das Signal erhalten. Zusätzlich
wird das Signal auch über eine Verzögerungsleitung in einen Monoflop gesendet, welcher einen negierten
Ausgang zum linken AND-Gatter und einen normalen Ausgang zum rechten AND-Gatter besitzt. Erhält der
Monoflop das erste Signal, wird über den negierten Ausgang in das linke AND-Gatter ein Start-Signal zum
Time-to-Amplitude Converter (TAC) gesendet. Dies startet die Suchzeit. Nach Ablauf selbiger oder
nach Erhalt eines weiteren Signals sendet der Monoflop ein Signal über den normalen Ausgang in
das rechte AND-Gatter, welches ein Stop-Signal an den TAC sendet.
Der TAC wandelt die Zeitspanne zwischen Start-Signal und Stop-Signal in eine Amplitude, dessen Höhe der
Länge der Zeitspanne entspricht, um.
Der TAC ist verbunden mit einem Vielkanalanalysator (im englischen Sprachraum Multi-Channel Analyser, daher abgekürzt als MCA).
Der MCA sortiert die erhaltenen Amplituden nach ihrer Höhe in ein Histogramm, welches von einem
angeschlossenen Computer ausgelesen und gespeichert werden kann.

\subsection{Justage des Versuchsaufbaus}
\label{JustageVersuchsaufbau}

Bevor die Messung durchgeführt werden kann, muss die nach Abbildung~\ref{fig:Schaltbild} aufgebaute Schaltung justiert
werden.

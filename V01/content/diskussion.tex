\section{Diskussion der Ergebnisse}
\label{sec:Diskussion}

\subsection{Probleme bei der Justage der Verzögerungsleitung}
%Die in Kapitel \ref{sec:justage} berechnete Halbwertsbreite entspricht nach der Theorie 
%der doppelten Breite der Rechtsecksimpulse. 
%Für $\text{Puls}=\SI{15}{\nano\second}$ beträgt
%die Abweichung von $t_{1/2}$ dabei $\SI{10\pm 8}{\percent}$ und ist somit mit dem Theoriewert
%verträglich. Für $\text{Puls}=\SI{10}{\nano\second}$ ist die Halbwertsbreite mit einer Abweichung
%von $\SI{92\pm 16}{\percent}$ deutlich ungleich null.
Die Breite der Verteilungen \ref{fig:justage10} und \ref{fig:justage15} entspricht 
nach der Theorie der doppelten Breite der Rechtsecksimpulse.
Es ist jedoch zu erkennen, dass die Verteilungen deutlich breiter sind. So ist für eine Pulsdauer
von $\Delta t_\text{Puls}=\SI{10}{\nano\second}$ bereits die Halbwertsbreite 
$t_{1/2}=\SI{38.3\pm 3.1}{\nano\second}$ um $\SI{92\pm 16}{\percent}$
größer als die erwartete Breite der Gesamtverteilung von $\SI{20}{\nano\second}$.
Eine mögliche Erklärung ist eine fehlerhafte Einstellung der Pulsdauer. Dadurch könnten
die Pulse länger sein, als nach Kapitel \ref{sec:Durchführung} gefordert und so zu einer 
Verbreiterung der Verteilungen führen.
\\
In der Verteilung für $\Delta t_{\text{Puls}}=\SI{10}{\nano\second}$ \ref{fig:justage10} ist zudem eine 
Unstetigkeit bei $\Delta t_\text{Delay}=0$ zu erkennen. Schon während der Aufnahme der Daten ist aufgefallen, 
dass einige Schalter für das Einstellen der Verzögerung blockierten, sodass diese teilweise kein Signal mehr
durchließen oder unklar war, ob das Umschalten erfolgreich war oder nicht. Eine nicht richtig zurückgesetzte 
Verzögerung könnte genau so eine Unstetigkeit wie beobachtet hervorrufen und somit auch die Halbwertsbreite
beeinflussen.
%Um die weitere Messung nicht zu sehr 
%durch diese Problematik zu beeinflussen, wurde eine Verzögerung gewählt, die nicht nur zentral auf dem Plateau 
%liegt, sondern auch eine möglichst unproblematische Einstellung an den Schaltern aufwies.

\subsection{Diskussion der Kalibration des MCA}
Die für den Zusammenhang zwischen Zeit und Kanal berechneten Regressionsparameter 
\begin{align*}
    a &= \SI{21.67 \pm 0.01}{\nano\second} \\ 
    b &= \SI{150.89 \pm 2.94}{\nano\second}
 \end{align*}
haben eine vergleichsweise geringe Unsicherheit, weswegen keine systematischen Fehler durch die Umrechnung von den Kanälen in die jeweiligen Zeiten zu erwarten sind. 

\subsection{Diskussion der experimentell bestimmten Lebensdauer}
Für die Diskussion der Messung wird die prozentuale Abweichung der in Kapitel \ref{sec:lebensdauer} bestimmten mittleren Lebensdauer $\tau_\text{Exp}$ 
zu dem Literaturwert 
\begin{equation*}
    \tau_{\mu,\text{PDG}}=\SI{2.1969811 \pm0.0000022}{\micro\second} \quad\text{\cite{PDG}}    
\end{equation*}
berechnet:
\begin{equation*}
    p=\frac{|\tau_\text{Exp}-\tau_{\mu,\text{PDG}}|}{\tau_{\mu,\text{PDG}}}\approx \SI{4.1\pm1.3}{\percent}.
\end{equation*}
Im Rahmen der in Kapitel \ref{sec:Auswertung} thematisierten Analyse der Daten ist diese Abweichung als gering einzuschätzen. Mögliche Gründe für die Abweichung
können kleinere systematische Fehler bei der Kalibration der Messapparatur sein, wie zum Beispiel Probleme bei der Verzögerungsleitung. Zudem sind auch Prozesse in dem
Szintillator möglich, die im Rahmen dieses Versuches vernachlässigt wurden. So ist es theoretisch nicht ausgeschlossen, dass ein Myon kurzzeitig von einem Atomkern 
eingefangen wird und dieses dadurch anregt. Ein so entstehender Lichtblitz würde zu einer zu kurzen Einschätzung der Zeit bis zum Zerfall führen und somit die 
bestimmte mittlere Lebensdauer senken.

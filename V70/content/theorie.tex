\section{Theoretischer Hintergrund des Versuches}
\label{sec:Theorie}

\subsection{Das Vakuum}
Der Begriff Vakuum ist definiert als der Zustand eines Gases, wenn der Druck
innerhalb eines Behälters geringer ist als außerhalb oder der Druck geringer
ist als $\SI{300}{\milli\bar}$, was der geringste auf der Erdoberfläche
vorkommende Druck ist.\\
Wenn Gase sich im Grenzfall einer verschwindenden Dichte befinden, ist das
ideale Gasgesetz \ref{eqn:idealesGas} eine gute Näherung:
\begin{equation}
 pV = N k_{B} T.
 \label{eqn:idealesGas}
\end{equation}
Hier bezeichnet $V$ das Volumen, $N$ die Teilchenanzahl, $k_{B}$ die
Boltzmann-Konstante, $T$ die Temperatur und $p$ den Druck.
Eine isotherme Zustandsänderung eines solchen idealen Gases kann beschrieben
werden durch das Boyle-Mariottesche Gesetz.
Es besagt, dass bei konstanter Temperatur und konstanter Stoffmenge der Druck direkt
proportional zum Volumen ist:
\begin{align}
  p & \propto \frac{1}{V} & \frac{V}{T} = \text{const}.\\
  \label{eqn:boylemariotte}
\end{align}
Der Druck ist definiert als eine senkrecht auf eine Fläche $A$ wirkende Kraft $F$ \ref{eqn:druck}:
\begin{equation}
  p = \frac{F}{A}.
  \label{eqn:druck}
\end{equation}
Der Partialdruck hingegen bezeichnet den Druck einer Komponente eines Gasgemischs.
Die Summe aller Partialdrücke ist daher bei idealen Gasen der Totaldruck. Dies
muss bei nicht idealen Gasen nicht notwendigerweise der Fall sein, da die Partialdrücke
sich allgemein gegenseitig stören können.\\
In SI-Einheiten wird der Druck in $\si{\pascal}$ angegeben. Der Zusammenhang zu dem
in Mitteleuropa gebräuchlicheren $\si{\bar}$ ist der Folgende:
\begin{equation}
 \SI{1}{\pascal} = \SI{0.01}{\milli\bar} = \SI{1}{\newton\per\square\meter}.
\end{equation}
Bei sehr geringen Drücken, wie z.B. im Weltall, ist die Angabe eines Drucks nicht
mehr praktikabel; stattdessen wird die Teilchenzahldichte verwendet. Diese gibt an,
wieviele Teilchen sich in einem gegebenen Raumbereich aufhalten.

\subsection{Wichtige Begriffe}
Im folgenden Abschnitt werden wichtige Begriffe im Zusammenhang mit Vakuumapparaturen
beschrieben.

\subsubsection*{Die mittlere freie Weglänge}
Als mittlere freie Weglänge wird die Strecke bezeichnet, die ein Gasteilchen
auf geradem Weg im Mittel zwischen zwei Stößen zurücklegen kann.

\subsubsection*{Die verschiedenen Vakuums- und Strömungsarten}
Für eine Charakterisierung einer Strömung wird das Verhältnis aus mittlerer
freier Weglänge und dem Durchmesser des Strömungskanals verwendet. Hierfür wird die
Knudsen-Zahl definiert:
\begin{equation}
 Kn = \frac{\bar{l}}{d}.
\end{equation}
Hierbei bezeichnet $\bar{l}$ die mittlere freie Weglänge und $d$ den Durchmesser
des Strömungskanals. Je nach Größe der Knudsen-Zahl und die Art des Vakuums wird
zwischen Kontinuumsströmung, Knudsenströmung und Molekularer Strömung unterschieden:
\begin{figure}[H]
  \centering
  \includegraphics[scale=0.4]{pictures/strömungen.png}
  \label{fig:strömung}
  \caption{Abbildung der Strömungsarten abhängig von der Knudsen-Zahl.}
\end{figure}
\noindent
Hierbei wird zwischen Grobvakuum, Feinvakuum und Hoch- bzw. Ultrahochvakuum unterschieden.
Grobvakuum meint dabei Drücke zwischen $\SI{300}{\milli\bar}$ und $\SI{1}{\milli\bar}$.
Hier finden hauptsächlich Stöße der Gasteilchen untereinander statt, was zu Kontinuumsströmung
führt.
Innerhalb der Kontinuumsströmung wird weiter differenziert zwischen der laminaren
und der turbulenten Strömung. Die laminare Strömung wird auch als Schichtströmung
bezeichnet und besteht aus parallel angeordneten Schichten aus Gasteilchen. Durch
Erhöhung der Geschwindigkeit können diese Schichten aufgelöst werden. Dies führt
zu ungeordnet durcheinander laufenden Gasteilchen, was als turbulente Strömung
bezeichnet wird.\\
Innerhalb einer Vakuumapparatur kommt es nur beim schnellen Abpumpen von Atmosphärendruck
sowie bei schnellem Belüften zu turbulenter Strömung. Turbulente Strömung erfordert
höhere Saugvermögen der Vakuumpumpen, weshalb es günstig ist, turbulente Strömung durch
die richtige Dimensionierung der Strömungskanäle zu vermeiden. Um dies sicherzustellen,
reichen bei Grobvakua kleine Leiterdurchmesser, während bei Hoch- und Ultrahochvakua
größere Durchmesser nötig sind, um turbulente Strömung zu verhindern.\\
Das Feinvakuum ist definiert als Druckbereich von $\SI{1}{\milli\bar}$ und $\SI{1e-3}{\milli\bar}$.
Hier ist die mittlere freie Weglänge vergleichbar groß wie der Durchmesser des Strömungskanals,
weshalb die Gasteilchen hauptsächlich mit den Wänden des Gefäßes wechselwirken.
Die Strömung wird als Knudsen-Strömung bezeichnet.
Der Druckbereich von $\SI{1e-3}{\milli\bar}$ bis $\SI{1e-8}{\milli\bar}$ wird als Hochvakuum
bezeichnet. Bereiche darüber werden Ultrahochvakuum genannt. Hier ist die mittlere freie
Weglänge größer als der Durchmesser des Strömungskanals, weshalb hier kaum Wechselwirkung
der Gasteilchen untereinander stattfindet. Die sich ergebende Strömung heißt Molekulare Strömung.

\subsubsection*{Die Absorption und Desorption}
Allgemein bezeichnet die Sorption einen Prozess, der zur Anreicherung eines Stoffes
innerhalb einer Phase oder an der Grenzfläche zwischen zwei Phasen führt.
Die Absorption bezeichnet hierbei die Anreicherung innerhalb einer Phase, während die
Anreicherung zwischen zwei Phasen als Adsoprtion bezeichnet wird. Handelt es sich nicht
um eine Aufnahme, sondern um eine Abnahme, wird von Desorption gesprochen.

\subsubsection*{Reale und virtuelle Lecks}
Als Lecks werden vakuummindernde Prozesse bezeichnet. Ein reales Leck ist eines,
das außerhalb der Vakuumanlage messbar ist, z.B. eine Undichtigkeit. Ein virtuelles
Leck hingegen ist eines, welches nicht außerhalb der Vakuumanlage erkennbar ist.
Beispiele hierfür sind Einschlüsse innerhalb der Anlage, welche nach einer Zeit an
die Materialoberfläche gelangen und austreten.
Eine Quelle von Lecks sind die obig erwähnten Sorptionen.

\subsubsection*{Das Saugvermögen einer Vakuumpumpe}
Um den obig erwähnten Lecks entgegenzuwirken, muss die Vakuumpumpe eine Saugleistung
besitzen, welche mindestens gleich der Leckrate ist. Das Saugvermögen ist definiert
als Volumenänderung pro Zeit:
\begin{equation}
 S = \frac{dV}{dt}.
\end{equation}

\subsection{Die Methoden der Vakuumerzeugung}

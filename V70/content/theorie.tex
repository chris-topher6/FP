\section{Theoretischer Hintergrund des Versuches}
\label{sec:Theorie}

\subsection{Das Vakuum}
Der Begriff Vakuum ist definiert als der Zustand eines Gases, wenn der Druck
innerhalb eines Behälters geringer ist als außerhalb oder der Druck geringer
ist als $\SI{300}{\milli\bar}$, was der geringste auf der Erdoberfläche
vorkommende Druck ist.\\
Wenn Gase sich im Grenzfall einer verschwindenden Dichte befinden, ist das
ideale Gasgesetz \ref{eqn:idealesGas} eine gute Näherung:
\begin{equation}
 pV = N k_{B} T.
 \label{eqn:idealesGas}
\end{equation}
Hier bezeichnet $V$ das Volumen, $N$ die Teilchenanzahl, $k_{B}$ die
Boltzmann-Konstante, $T$ die Temperatur und $p$ den Druck.
Der Druck ist definiert als eine senkrecht auf eine Fläche $A$ wirkende Kraft $F$ \ref{eqn:druck}:
\begin{equation}
  p = \frac{F}{A}.
  \label{eqn:druck}
\end{equation}
Ein

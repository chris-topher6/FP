%irgendwie muss ich mir noch was mit den Einheiten überlegen
%Tabellen in den Anhang?
\section{Auswertung der Messdaten}
\label{sec:Auswertung}
%#############################################################################################################
%#############################################################################################################
\subsection{Drehschieberpumpe}
%#############################################################################################################
%#############################################################################################################
Die Drehschieberpumpe besitzt laut Herstellerangaben eine Saugleistung von zwischen 
$\num{4.6}$ und $\SI{5.5}{\cubic\metre\per\hour}$ \cite{Versuchsbeschreibung}.
Im Folgenden wird die Saugleistung mit aus den aufgenommenen Daten der Evakuierungsmessung
und der Leckratenmessungen bestimmt. 
%%%%%%%%%%%%%%%%%%%%%%%%%%%%%%%%%%%%%%%%%%%%%%%%%%%%%%%%%%%%%%%%%%%%%%%%%%%%%%%%%%%%%%%%%%%%%%%%%%%%%%%%%%%%%%%
\subsubsection{Evakuierungsmessung}
%%%%%%%%%%%%%%%%%%%%%%%%%%%%%%%%%%%%%%%%%%%%%%%%%%%%%%%%%%%%%%%%%%%%%%%%%%%%%%%%%%%%%%%%%%%%%%%%%%%%%%%%%%%%%%%
Die nach Kapitel \ref{sec:Durchführung} aufgenommenen Daten sind in Tabelle \ref{tab:Dreh_Evak}
aufgelistet. Zudem sind der drei Messungen, die systematische Unsicherheiten durch die 
Messgeräte und und die statistischen Unsicherheiten angegeben. Da die systematischen
die statistischen Unsicherheiten deutlich dominieren, werden in den folgenden Berechnungen
nur die systematischen Unsicherheiten betrachtet. Für die Zeitmessung wurde die durch die 
Reaktionszeit der Experimentierenden entstehende Unsicherheit als $\Delta t=\SI{0.2}{\second}$
angenommen.
%Tabelle?
Durch Umstellen von Gleichung \ref{eqn:pt} ergibt sich der Zusammenhang
\begin{equation*}
  \ln{\left[\frac{p-p_E}{p_0-p_E}\right]}=-\frac{S}{V}\cdot t
\end{equation*}
mit dem erreichten Enddruck $p_E=\SI{7.1e-3}{\hecto\pascal}$ und dem Umgebungsdruck
$p_0=\SI{1011}{\hecto\pascal}$. 
Die Saugleistung kann somit aus einer linearen Regression der Form 
\begin{equation}
  y=ax+b
  \label{eqn:gerade}
\end{equation}
bestimmt werden. Da die Saugleistung jedoch vom Druck abhängig ist, werden mehrere 
Regressionen für verschiedene Druckbereiche durchgeführt. Diese sind in Abbildung 
\ref{fig:dreh_evak} graphisch dargestellt.
\begin{figure}[H]
    \centering
    \includegraphics[width=0.6\linewidth]{build/Dreh_Evak.pdf}
    \caption{Evakuierungskurve für die Drehschieberpumpe in halblogarithmischer Darstellung mit mehreren linearen Regressionen für verschiedene Druckbereiche.}
    \label{fig:dreh_evak}
\end{figure}
\noindent
Die zugehörigen Parameter der Regressionen sind in Tabelle \ref{tab:Dreh_Evak_para} gegeben.
\begin{table}[H]
  \centering
    \caption{Regressionsparameter für die Leckratenmessung für die Drehschieberpumpe.}
    \label{tab:Dreh_Evak_para}
    \sisetup{table-format=1.3}
    \begin{tabular}{S[table-format=1.0] S[table-format=2.4] @{${}\pm{}$} S[table-format=1.4] S[table-format=2.2] @{${}\pm{}$} S[table-format=1.2]}
      \toprule
      {Regression} & \multicolumn{2}{c}{$a$} & \multicolumn{2}{c}{$b$} \\
      \midrule
      1 & -0.0308 & 0.0006 & -0.09 & 0.07\\
      2 & -0.0039 & 0.0005 & -6.5  & 0.3\\
      \bottomrule
    \end{tabular}
\end{table}
Unter Ausnutzung von 
\begin{equation}
  S=-aV
  \label{eqn:evak}
\end{equation}
ergibt sich mit $V=\SI{34\pm 3.4}{\litre}$
\begin{align*}
  S_\text{Dreh}(p=\SI{325\pm 322}{\hecto\pascal})&=\SI[per-mode=reciprocal]{3.8 \pm 0.4}{\cubic\metre\per\hour}\\
  S_\text{Dreh}(p=\SI{0.255\pm 0.095}{\hecto\pascal})&=\SI[per-mode=reciprocal]{0.48 \pm 0.08}{\cubic\metre\per\hour}.
\end{align*}
%%%%%%%%%%%%%%%%%%%%%%%%%%%%%%%%%%%%%%%%%%%%%%%%%%%%%%%%%%%%%%%%%%%%%%%%%%%%%%%%%%%%%%%%%%%%%%%%%%%%%%%%%%%%%%%
\subsection{Leckratenmessung}
%%%%%%%%%%%%%%%%%%%%%%%%%%%%%%%%%%%%%%%%%%%%%%%%%%%%%%%%%%%%%%%%%%%%%%%%%%%%%%%%%%%%%%%%%%%%%%%%%%%%%%%%%%%%%%%
Die nach Kapitel \ref{sec:Durchführung} aufgenommenen Messdaten der Leckratenmessung sind 
für je einen Gleichgewichtsdruck $p_g$ in den Tabellen \ref{tab:Dreh_Leck1} bis \ref{tab:Dreh_Leck4}
gelistet. Analog zu der Evakuierungsmessung werden in den weiteren Berechnungen nur die systematischen
Unsicherheiten betrachtet, da diese die statistischen dominieren. 
%Tabellen?
Durch die Anpassung einer linearen Funktion \ref{eqn:gerade} an die Messdaten kann über den 
Zusammenhang \ref{eqn:Leck} die Saugleistung nach
\begin{equation}
  S=\frac{V_0}{p_G}\frac{\partial p}{\partial t}=\frac{V_0}{p_G}\cdot a
  \label{eqn:Leck2}
\end{equation} 
berechnet werden.
In den Abbildungen \ref{fig:dreh_leck1} bis \ref{fig:dreh_leck4} sind die Messdaten mit der 
berechneten Regression visualisiert.
Die Regressionsparameter sind in Tabelle \ref{tab:Dreh_Leck_para} enumiert.
\begin{table}[H]
  \centering
    \caption{Regressionsparameter für die Leckratenmessung für die Drehschieberpumpe.}
    \label{tab:Dreh_Leck_para}
    \sisetup{table-format=3.1}
    \begin{tabular}{c c @{${}\pm{}$} c c @{${}\pm{}$} c}
      \toprule
      {$p_g [\si{\hecto\pascal}$]} & \multicolumn{2}{c}{$a$} & \multicolumn{2}{c}{$b$} \\
      \midrule
      0.5 & 0.0102 & 0.0010 & 1.61  & 0.10\\
      10  & 0.374  & 0.005  & 15.85 & 0.33\\
      50  & 1.753  & 0.021  & 57.8  & 1.3\\
      100 & 3.37   & 0.08   & 113.1 & 2.9\\
      \bottomrule
    \end{tabular}
\end{table}
\begin{figure}
    \centering
    \includegraphics[width=0.6\linewidth]{build/Dreh_Leck_1.pdf}
    \caption{Druck in Abhängigkeit der Zeit für die Leckratenmessung für die Drehschieberpumpe mit linearer Regression. Der eingestellte Gleichgewichtsdruck beträgt $p_g=\SI{0.5}{\hecto\pascal}$.}
    \label{fig:dreh_leck1}
\end{figure}
\begin{figure}
    \centering
    \includegraphics[width=0.6\linewidth]{build/Dreh_Leck_2.pdf}
    \caption{Druck in Abhängigkeit der Zeit für die Leckratenmessung für die Drehschieberpumpe mit linearer Regression. Der eingestellte Gleichgewichtsdruck beträgt $p_g=\SI{10}{\hecto\pascal}$.}
    \label{fig:dreh_leck2}
\end{figure}
\begin{figure}
    \centering
    \includegraphics[width=0.6\linewidth]{build/Dreh_Leck_3.pdf}
    \caption{Druck in Abhängigkeit der Zeit für die Leckratenmessung für die Drehschieberpumpe mit linearer Regression. Der eingestellte Gleichgewichtsdruck beträgt $p_g=\SI{50}{\hecto\pascal}$.}
    \label{fig:dreh_leck3}
\end{figure}
\begin{figure}
    \centering
    \includegraphics[width=0.6\linewidth]{build/Dreh_Leck_4.pdf}
    \caption{Druck in Abhängigkeit der Zeit für die Leckratenmessung für die Drehschieberpumpe mit linearer Regression. Der eingestellte Gleichgewichtsdruck beträgt $p_g=\SI{100}{\hecto\pascal}$.}
    \label{fig:dreh_leck4}
\end{figure}
\noindent
Durch Einsetzen in Gleichung \ref{eqn:Leck2} ergeben sich die Saugleistungen
\begin{align*}
  S_\text{Dreh}(p_g=\SI{0.5}{\hecto\pascal})&=\SI[per-mode=reciprocal]{2.5 \pm 0.4}{\cubic\metre\per\hour}\\
  S_\text{Dreh}(p_g=\SI{10}{\hecto\pascal})&=\SI[per-mode=reciprocal]{4.6 \pm 0.5}{\cubic\metre\per\hour}\\
  S_\text{Dreh}(p_g=\SI{50}{\hecto\pascal})&=\SI[per-mode=reciprocal]{4.3 \pm 0.5}{\cubic\metre\per\hour}\\
  S_\text{Dreh}(p_g=\SI{100}{\hecto\pascal})&=\SI[per-mode=reciprocal]{4.1 \pm 0.4}{\cubic\metre\per\hour}.
\end{align*}
%##############################################################################################################
%##############################################################################################################
\subsection{Turbomolekularpumpe}
%##############################################################################################################
%##############################################################################################################
Analog zu der Drehschieberpumpe wird für die Turbomolekularpumpe die Saugleistung über die 
Evakuierungsmessung und über die Leckratenmessungen ermittelt. Laut Herstellerangaben
beträgt die Saugleistung $S=\SI{77}{\litre\per\second}$.
%%%%%%%%%%%%%%%%%%%%%%%%%%%%%%%%%%%%%%%%%%%%%%%%%%%%%%%%%%%%%%%%%%%%%%%%%%%%%%%%%%%%%%%%%%%%%%%%%%%%%%%%%%%%%%%
\subsubsection{Evakuierungsmessung}
%%%%%%%%%%%%%%%%%%%%%%%%%%%%%%%%%%%%%%%%%%%%%%%%%%%%%%%%%%%%%%%%%%%%%%%%%%%%%%%%%%%%%%%%%%%%%%%%%%%%%%%%%%%%%%%
Die Daten für die Evakuierungsmessung der Turbomolekularpumpe sind in Tabelle \ref{tab:Turbo_Evak}
aufgelistet.
%Tabelle?
Auch mit diesen Daten wird 
\begin{equation*}
  \ln{\left[\frac{p-p_E}{p_0-p_E}\right]}
\end{equation*}
gebildet, damit über mehrere Regressionen der Form \ref{eqn:gerade} die Saugleistungen für 
verschiedene Druckbereiche bestimmt werden können. Der erreichte Enddruck ist 
$p_E=\SI{7.69e-6}{\hecto\pascal}$. Der Startdruck $p_0$ ist bei dieser Messung nicht der 
Umgebungsdruck, sondern $p_0=\SI{5e-3}{\hecto\pascal}$. Die Parameter der Regressionen 
sind in Tabelle \ref{tab:Turbo_Evak_para} aufgelistet. Die dazugehörigen Regressionen 
sind zudem in Abbildung \ref{fig:turbo_evak} dargestellt.
\begin{table}[H]
    \centering
      \caption{Regressionsparameter für die Leckratenmessung für die Turbomolekularpumpe.}
      \label{tab:Turbo_Evak_para}
      \sisetup{table-format=1.3}
      \begin{tabular}{S[table-format=1.0] S @{${}\pm{}$} S S[table-format=2.1] @{${}\pm{}$} S[table-format=1.1]}
        \toprule
        {Regression} & \multicolumn{2}{c}{$a$} & \multicolumn{2}{c}{$b$} \\
        \midrule
        1 & -0.490 & 0.010 & -0.1 & 0.4\\
        2 & -0.19  & 0.04  & -1.8 & 0.5\\
        3 & -0.010 & 0.006 & -5.6 & 0.5\\
        \bottomrule
      \end{tabular}
\end{table}
\begin{figure}[H]
    \centering
    \includegraphics[width=0.6\linewidth]{build/Turbo_Evak.pdf}
    \caption{Evakuierungskurve für die Turbomolekularpumpe in halblogarithmischer Darstellung mit mehreren linearen Regressionen für verschiedene Druckbereiche.}
    \label{fig:turbo_evak}
\end{figure}
\noindent
Die Saugleistung ergibt sich aus Gleichung \ref{eqn:evak} mit einem Volumen von $V=\SI{33 \pm 3.3}{\litre}$.
Daraus folgt
\begin{align*}
  S_\text{Turbo}(p=\SI{2.7\pm 2.4 e-3}{\hecto\pascal})&=\SI[per-mode=reciprocal]{58 \pm 13}{\cubic\metre\per\hour}\\
  S_\text{Turbo}(p=\SI{1.7\pm 1.3 e-4}{\hecto\pascal})&=\SI[per-mode=reciprocal]{22\pm 6}{\cubic\metre\per\hour}\\
  S_\text{Turbo}(p=\SI{1.9\pm 0.5 e-5}{\hecto\pascal})&=\SI[per-mode=reciprocal]{1.2\pm 0.8}{\cubic\metre\per\hour}.
\end{align*}
%%%%%%%%%%%%%%%%%%%%%%%%%%%%%%%%%%%%%%%%%%%%%%%%%%%%%%%%%%%%%%%%%%%%%%%%%%%%%%%%%%%%%%%%%%%%%%%%%%%%%%%%%%%%%%%
\subsection{Leckratenmessung}
%%%%%%%%%%%%%%%%%%%%%%%%%%%%%%%%%%%%%%%%%%%%%%%%%%%%%%%%%%%%%%%%%%%%%%%%%%%%%%%%%%%%%%%%%%%%%%%%%%%%%%%%%%%%%%%
In den Tabellen \ref{tab:Turbo_Leck1} bis \ref{tab:Turbo_Leck4} sind die aufgenommenen Daten, die 
Mittelwerte aus den Messungen und die systematischen und statistischen Unsicherheiten gelistet.
Erneut werden nur systematische Unsicherheiten betrachtet, da diese deutlich größer sind als die 
statistischen. 
%Tabelle
An die Messdaten werden durch Ausgleichsrechnung Funktionen der Form \ref{eqn:gerade} angepasst.
Da das Modell einer linearen Funktion die Daten jedoch unzureichend beschreibt, werden 
für jeden Gleichgewichtsdruck jeweils zwei Regressionen bestimmt. Eine im Bereich niedriger 
und eine im Bereich höherer Drücke. Die berechneten Parameter sind in Tabelle \ref{tab:Turbo_Leck_para}
aufgelistete. Die Daten mit den Regressionen finden sich in den Abbildungen \ref{fig:turbo_leck1} 
bis \ref{fig:turbo_leck4}.
\begin{table}[H]
    \centering
      \caption{Regressionsparameter für die Leckratenmessung für die Turbomolekularpumpe.}
      \label{tab:Turbo_Leck_para}
      \sisetup{table-format=3.1}
      \begin{tabular}{c c @{${}\pm{}$} c c @{${}\pm{}$} c}
        \toprule
        {$p_g [\si{\hecto\pascal}$]} & \multicolumn{2}{c}{$a$} & \multicolumn{2}{c}{$b$} \\
        \midrule
        $\num{5e-5}$  & 0.0102 & 0.0014 & 0.065 & 0.014\\
                      & 0.015  & 0.004  & -0.32 & 0.37 \\
        \midrule
        $\num{7e-5}$  & 0.0141 & 0.0019 & 0.085 & 0.018\\
                      & 0.027  & 0.006  & -0.5  & 0.4  \\
        \midrule
        $\num{1e-4}$  & 0.228  & 0.003  & 0.11  & 0.03 \\
                      & 0.055  & 0.012  & -1.2  & 0.8  \\
        \midrule
        $\num{2e-4}$  & 0.16   & 0.03   & -0.5  & 0.5  \\
                      & 0.16   & 0.05   & 0.4   & 4    \\
        \bottomrule
      \end{tabular}
\end{table}
\begin{figure}[H]
    \centering
    \includegraphics[width=0.6\linewidth]{build/Turbo_Leck_1.pdf}
    \caption{Druck in Abhängigkeit der Zeit für die Leckratenmessung für die Turbomolekularpumpe mit linearer Regression. Der eingestellte Gleichgewichtsdruck beträgt $p_g=\SI{5e-5}{\hecto\pascal}$.}
    \label{fig:turbo_leck1}
\end{figure}
\noindent
\begin{figure}[H]
    \centering
    \includegraphics[width=0.6\linewidth]{build/Turbo_Leck_2.pdf}
    \caption{Druck in Abhängigkeit der Zeit für die Leckratenmessung für die Turbomolekularpumpe mit linearer Regression. Der eingestellte Gleichgewichtsdruck beträgt $p_g=\SI{7e-5}{\hecto\pascal}$.}
    \label{fig:turbo_leck2}
\end{figure}
\noindent
\begin{figure}[H]
    \centering
    \includegraphics[width=0.6\linewidth]{build/Turbo_Leck_3.pdf}
    \caption{Druck in Abhängigkeit der Zeit für die Leckratenmessung für die Turbomolekularpumpe mit linearer Regression. Der eingestellte Gleichgewichtsdruck beträgt $p_g=\SI{1e-4}{\hecto\pascal}$.}
    \label{fig:turbo_leck3}
\end{figure}
\noindent
\begin{figure}[H]
    \centering
    \includegraphics[width=0.6\linewidth]{build/Turbo_Leck_4.pdf}
    \caption{Druck in Abhängigkeit der Zeit für die Leckratenmessung für die Turbomolekularpumpe mit linearer Regression. Der eingestellte Gleichgewichtsdruck beträgt $p_g=\SI{2e-4}{\hecto\pascal}$.}
    \label{fig:turbo_leck4}
\end{figure}
\noindent
Durch Einsetzen der Parameter in Gleichung \ref{eqn:Leck2} ergeben sich die Saugleistungen 
\begin{align*}
  S^1_\text{Dreh}(p_g=\SI{5e-5}{\hecto\pascal})&=\SI[per-mode=reciprocal]{25 \pm 9}{\cubic\metre\per\hour}\\
  S^2_\text{Dreh}(p_g=\SI{5e-5}{\hecto\pascal})&=\SI[per-mode=reciprocal]{38 \pm 16}{\cubic\metre\per\hour}\\
  S^1_\text{Dreh}(p_g=\SI{7e-5}{\hecto\pascal})&=\SI[per-mode=reciprocal]{25 \pm 8}{\cubic\metre\per\hour}\\
  S^2_\text{Dreh}(p_g=\SI{7e-5}{\hecto\pascal})&=\SI[per-mode=reciprocal]{46 \pm 18}{\cubic\metre\per\hour}\\
  S^1_\text{Dreh}(p_g=\SI{1e-4}{\hecto\pascal})&=\SI[per-mode=reciprocal]{28 \pm 10}{\cubic\metre\per\hour}\\
  S^2_\text{Dreh}(p_g=\SI{1e-4}{\hecto\pascal})&=\SI[per-mode=reciprocal]{67 \pm 26}{\cubic\metre\per\hour}\\
  S^1_\text{Dreh}(p_g=\SI{2e-4}{\hecto\pascal})&=\SI[per-mode=reciprocal]{98 \pm 35}{\cubic\metre\per\hour}\\
  S^2_\text{Dreh}(p_g=\SI{2e-4}{\hecto\pascal})&=\SI[per-mode=reciprocal]{97 \pm 42}{\cubic\metre\per\hour}.
\end{align*}
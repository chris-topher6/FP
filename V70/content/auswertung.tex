\section{Auswertung der Messdaten}
\label{sec:Auswertung}
\subsection{Drehschieberpumpe}
\subsubsection{Evakuierungsmessung}
\begin{table}[H]
    \centering
      \caption{Mitttelwerte der Druckmessung mit statistischen und systematischen Unsicherheiten.}
      \label{tab:Dreh_Evak}
      \sisetup{table-format=1.2}
      \begin{tabular}{
        S[table-format=3.0] S[table-format=3.2] S[table-format=1.3] S[table-format=3.2] S @{${}\pm{}$} S | 
        S[table-format=3.0] S[table-format=3.2] S[table-format=1.3] S[table-format=1.3] S @{${}\pm{}$} S
        }
        \toprule
        {$t [\si{\second}$]} & 
        {$p [\si{\hecto\pascal}]$} & 
        {$\Delta p_\text{st} [\si{\hecto\pascal}]$} & 
        {$\Delta p_\text{sy} [\si{\hecto\pascal}]$} & 
        \multicolumn{2}{c}{$\ln{\left[\frac{p-p_E}{p_0-p_E}\right]}$} &
        {$t [\si{\second}$]} & 
        {$p [\si{\hecto\pascal}]$} & 
        {$p_\text{st} [\si{\hecto\pascal}]$} & 
        {$p_\text{sy} [\si{\hecto\pascal}]$} &
        \multicolumn{2}{c}{$\ln{\left[\frac{p-p_E}{p_0-p_E}\right]}$} \\
        \midrule
        0   & 1010.3 & 0.6  & 101     & -0.00 & 0.14 &   310 & 0.61   & 0       & 0.06  & -7.42 & 0.14\\
        10  & 647    & 14   & 65      & -0.45 & 0.14 &   320 & 0.57   & 0.006   & 0.07  & -7.50 & 0.14\\
        20  & 488    & 5    & 49      & -0.73 & 0.14 &   330 & 0.53   & 0       & 0.05  & -7.57 & 0.14\\
        30  & 370    & 4    & 37      & -1.01 & 0.14 &   340 & 0.49   & 0       & 0.05  & -7.65 & 0.14\\
        40  & 276.8  & 2.3  & 28      & -1.30 & 0.14 &   350 & 0.46   & 0       & 0.05  & -7.71 & 0.14\\
        50  & 204.4  & 0.9  & 20      & -1.60 & 0.14 &   360 & 0.43   & 0       & 0.04  & -7.78 & 0.14\\
        60  & 150.8  & 2.9  & 15      & -1.90 & 0.14 &   370 & 0.41   & 0       & 0.04  & -7.83 & 0.14\\
        70  & 111.4  & 1.5  & 11      & -2.21 & 0.14 &   380 & 0.39   & 0       & 0.04  & -7.88 & 0.14\\
        80  & 81.8   & 1.5  & 8       & -2.51 & 0.14 &   390 & 0.37   & 0.006   & 0.04  & -7.94 & 0.14\\
        90  & 58.8   & 0.5  & 6       & -2.84 & 0.14 &   400 & 0.35   & 0       & 0.04  & -7.99 & 0.14\\
        100 & 42.9   & 0.5  & 4       & -3.16 & 0.14 &   410 & 0.323  & 0.006   & 0.032 & -8.07 & 0.14\\
        110 & 30.4   & 0.4  & 3.0     & -3.50 & 0.14 &   420 & 0.313  & 0.006   & 0.031 & -8.10 & 0.14\\
        120 & 22.43  & 0.15 & 2.2     & -3.81 & 0.14 &   430 & 0.300  & 0       & 0.030 & -8.15 & 0.14\\
        130 & 16.33  & 0.12 & 1.6     & -4.13 & 0.14 &   440 & 0.287  & 0.006   & 0.029 & -8.19 & 0.14\\
        140 & 12.07  & 0.15 & 1.2     & -4.43 & 0.14 &   450 & 0.273  & 0.006   & 0.027 & -8.24 & 0.14\\
        150 & 8.80   & 0.17 & 0.9     & -4.74 & 0.14 &   460 & 0.260  & 0       & 0.026 & -8.29 & 0.14\\
        160 & 6.30   & 0.17 & 0.6     & -5.08 & 0.14 &   470 & 0.253  & 0.006   & 0.025 & -8.32 & 0.14\\
        170 & 4.73   & 0.06 & 0.5     & -5.37 & 0.14 &   480 & 0.243  & 0.006   & 0.024 & -8.36 & 0.14\\
        180 & 3.73   & 0.06 & 0.4     & -5.60 & 0.14 &   490 & 0.230  & 0       & 0.023 & -8.42 & 0.14\\
        190 & 2.97   & 0.06 & 0.30    & -5.83 & 0.14 &   500 & 0.227  & 0.006   & 0.023 & -8.43 & 0.14\\
        200 & 2.40   & 0    & 0.24    & -6.05 & 0.14 &   510 & 0.217  & 0.006   & 0.022 & -8.48 & 0.14\\
        210 & 2.00   & 0    & 0.20    & -6.23 & 0.14 &   520 & 0.207  & 0.006   & 0.021 & -8.53 & 0.14\\
        220 & 1.70   & 0    & 0.17    & -6.39 & 0.14 &   530 & 0.200  & 0       & 0.020 & -8.56 & 0.14\\
        230 & 1.40   & 0    & 0.14    & -6.59 & 0.14 &   540 & 0.102  & 0.006   & 0.020 & -8.58 & 0.14\\
        240 & 1.30   & 0    & 0.13    & -6.66 & 0.14 &   550 & 0.189  & 0.006   & 0.019 & -8.64 & 0.14\\
        250 & 1.10   & 0    & 0.11    & -6.83 & 0.14 &   560 & 0.180  & 0       & 0.018 & -8.67 & 0.14\\
        260 & 0.97   & 0    & 0.10    & -6.96 & 0.14 &   570 & 0.177  & 0.006   & 0.018 & -8.69 & 0.14\\
        270 & 0.88   & 0    & 0.09    & -7.05 & 0.14 &   580 & 0.170  & 0       & 0.017 & -8.73 & 0.14\\
        280 & 0.80   & 0    & 0.08    & -7.15 & 0.14 &   590 & 0.167  & 0.006   & 0.017 & -8.75 & 0.14\\
        290 & 0.72   & 0    & 0.07    & -7.26 & 0.14 &   600 & 0.160  & 0       & 0.016 & -8.80 & 0.14\\
        300 & 0.66   & 0.006& 0.07    & -7.35 & 0.14\\
        \bottomrule
      \end{tabular}
\end{table}
\begin{figure}[H]
    \centering
    \includegraphics[scale= 0.8]{build/Dreh_Evak.pdf}
    \caption{Evakuierungskurve für die Drehschieberpumpe in halblogarithmischer Darstellung mit mehreren linearen Regressionen für verschiedene Druckbereiche.}
    \label{fig:dreh_evak}
\end{figure}
\noindent
\subsection{Leckratenmessung}
\begin{table}[H]
    \centering
      \caption{Regressionsparameter für die Leckratenmessung für die Drehschieberpumpe.}
      \label{tab:Dreh_Leck_para}
      \sisetup{table-format=3.1}
      \begin{tabular}{c c @{${}\pm{}$} c c @{${}\pm{}$} c}
        \toprule
        {$p_g [\si{\hecto\pascal}$]} & \multicolumn{2}{c}{$a$} & \multicolumn{2}{c}{$b$} \\
        \midrule
        0.5 & 0.0102 & 0.0010 & 1.61  & 0.10\\
        10  & 0.374  & 0.005  & 15.85 & 0.33\\
        50  & 1.753  & 0.021  & 57.8  & 1.3\\
        100 & 3.37   & 0.08   & 113.1 & 2.9\\
        \bottomrule
      \end{tabular}
\end{table}
\begin{figure}
    \centering
    \includegraphics[width=0.8\linewidth]{build/Dreh_Leck_1.pdf}
    \caption{Druck in Abhängigkeit der Zeit für die Leckratenmessung für die Drehschieberpumpe mit linearer Regression. Der eingestellte Gleichgewichtsdruck beträgt $p_g=\SI{0.5}{\hecto\pascal}$.}
    \label{fig:dreh_leck1}
\end{figure}

\begin{figure}
    \centering
    \includegraphics[width=0.8\linewidth]{build/Dreh_Leck_2.pdf}
    \caption{Druck in Abhängigkeit der Zeit für die Leckratenmessung für die Drehschieberpumpe mit linearer Regression. Der eingestellte Gleichgewichtsdruck beträgt $p_g=\SI{10}{\hecto\pascal}$.}
    \label{fig:dreh_leck2}
\end{figure}

\begin{figure}
    \centering
    \includegraphics[width=0.8\linewidth]{build/Dreh_Leck_3.pdf}
    \caption{Druck in Abhängigkeit der Zeit für die Leckratenmessung für die Drehschieberpumpe mit linearer Regression. Der eingestellte Gleichgewichtsdruck beträgt $p_g=\SI{50}{\hecto\pascal}$.}
    \label{fig:dreh_leck3}
\end{figure}

\begin{figure}
    \centering
    \includegraphics[width=0.8\linewidth]{build/Dreh_Leck_4.pdf}
    \caption{Druck in Abhängigkeit der Zeit für die Leckratenmessung für die Drehschieberpumpe mit linearer Regression. Der eingestellte Gleichgewichtsdruck beträgt $p_g=\SI{100}{\hecto\pascal}$.}
    \label{fig:dreh_leck4}
\end{figure}
\noindent
\subsection{Turbomolekularpumpe}
\subsubsection{Evakuierungsmessung}
\begin{table}[H]
    \centering
      \caption{Mitttelwerte der Druckmessung mit statistischen und systematischen Unsicherheiten.}
      \label{tab:Turbo_Evak}
      \sisetup{table-format=1.1}
      \begin{tabular}{
        S[table-format=2.0] S[table-format=1.5] S[table-format=1.5] S[table-format=1.4] S @{${}\pm{}$} S | 
        S[table-format=3.0] S[table-format=1.5] S[table-format=1.5] S[table-format=1.3] S @{${}\pm{}$} S
        }
        \toprule
        {$t [\si{\second}$]} & 
        {$p [\si{\hecto\pascal}]$} & 
        {$\Delta p_\text{st} [\si{\hecto\pascal}]$} & 
        {$\Delta p_\text{sy} [\si{\hecto\pascal}]$} & 
        \multicolumn{2}{c}{$\ln{\left[\frac{p-p_E}{p_0-p_E}\right]}$} &
        {$t [\si{\second}$]} & 
        {$p [\si{\hecto\pascal}]$} & 
        {$p_\text{st} [\si{\hecto\pascal}]$} & 
        {$p_\text{sy} [\si{\hecto\pascal}]$} &
        \multicolumn{2}{c}{$\ln{\left[\frac{p-p_E}{p_0-p_E}\right]}$} \\
        \midrule
        0   & 5.13    & 0.23    & 1.5   &  0.0 & 0.4 & 40  & 0.0203  & 0.0005  & 0.006 & -6.0 & 0.6\\
        2   & 1.60    & 0.35    & 0.5   & -1.1 & 0.4 & 45  & 0.0193  & 0.0004  & 0.006 & -6.1 & 0.6\\
        4   & 0.50    & 0.05    & 0.15  & -2.3 & 0.4 & 50  & 0.0185  & 0.0004  & 0.006 & -6.1 & 0.6\\
        6   & 0.298   & 0.014   & 0.09  & -2.8 & 0.4 & 55  & 0.0177  & 0.0004  & 0.005 & -6.2 & 0.7\\
        8   & 0.185   & 0.004   & 0.06  & -3.3 & 0.4 & 60  & 0.0172  & 0.0004  & 0.005 & -6.3 & 0.7\\
        10  & 0.119   & 0.006   & 0.04  & -3.8 & 0.4 & 65  & 0.01653 & 0.00035 & 0.005 & -6.3 & 0.7\\
        12  & 0.0958  & 0.0026  & 0.029 & -4.0 & 0.4 & 70  & 0.0162  & 0.0004  & 0.005 & -6.4 & 0.7\\
        14  & 0.0661  & 0.0019  & 0.020 & -4.4 & 0.5 & 75  & 0.01573 & 0.00035 & 0.005 & -6.4 & 0.7\\
        16  & 0.050   & 0.003   & 0.015 & -4.8 & 0.5 & 80  & 0.0154  & 0.0004  & 0.005 & -6.5 & 0.7\\
        18  & 0.0370  & 0.0005  & 0.011 & -5.1 & 0.5 & 85  & 0.01507 & 0.00031 & 0.005 & -6.5 & 0.8\\
        20  & 0.03127 & 0.00025 & 0.009 & -5.4 & 0.5 & 90  & 0.0148  & 0.0004  & 0.004 & -6.6 & 0.8\\
        22  & 0.02970 & 0.00010 & 0.009 & -5.4 & 0.5 & 95  & 0.01453 & 0.00032 & 0.004 & -6.6 & 0.8\\
        24  & 0.0282  & 0.0004  & 0.009 & -5.5 & 0.5 & 100 & 0.01427 & 0.00031 & 0.004 & -6.6 & 0.8\\
        26  & 0.0262  & 0.0005  & 0.008 & -5.6 & 0.5 & 105 & 0.01407 & 0.00031 & 0.004 & -6.7 & 0.8\\
        28  & 0.0246  & 0.0004  & 0.007 & -5.7 & 0.5 & 110 & 0.01390 & 0.00026 & 0.004 & -6.7 & 0.8\\
        30  & 0.0236  & 0.0004  & 0.007 & -5.7 & 0.6 & 115 & 0.01370 & 0.00026 & 0.004 & -6.7 & 0.8\\
        35  & 0.0216  & 0.0004  & 0.006 & -5.9 & 0.6 & 120 & 0.01350 & 0.00026 & 0.004 & -6.8 & 0.9\\
        \bottomrule
      \end{tabular}
\end{table}
\begin{table}[H]
    \centering
      \caption{Regressionsparameter für die Leckratenmessung für die Turbomolekularpumpe.}
      \label{tab:Turbo_Evak_para}
      \sisetup{table-format=1.3}
      \begin{tabular}{S[table-format=1.0] S @{${}\pm{}$} S S[table-format=2.1] @{${}\pm{}$} S[table-format=1.1]}
        \toprule
        {Regression} & \multicolumn{2}{c}{$a$} & \multicolumn{2}{c}{$b$} \\
        \midrule
        1 & -0.490 & 0.010 & -0.1 & 0.4\\
        2 & -0.19  & 0.04  & -1.8 & 0.5\\
        3 & -0.010 & 0.006 & -5.6 & 0.5\\
        \bottomrule
      \end{tabular}
\end{table}
\begin{figure}[H]
    \centering
    \includegraphics[scale= 0.8]{build/Turbo_Evak.pdf}
    \caption{Evakuierungskurve für die Turbomolekularpumpe in halblogarithmischer Darstellung mit mehreren linearen Regressionen für verschiedene Druckbereiche.}
    \label{fig:turbo_evak}
\end{figure}
\noindent
\subsection{Leckratenmessung}
\begin{table}[H]
    \centering
      \caption{Regressionsparameter für die Leckratenmessung für die Turbomolekularpumpe.}
      \label{tab:Turbo_Leck_para}
      \sisetup{table-format=3.1}
      \begin{tabular}{c c @{${}\pm{}$} c c @{${}\pm{}$} c}
        \toprule
        {$p_g [\si{\hecto\pascal}$]} & \multicolumn{2}{c}{$a$} & \multicolumn{2}{c}{$b$} \\
        \midrule
        $\num{5e-5}$  & 0.0102 & 0.0014 & 0.065 & 0.014\\
                      & 0.015  & 0.004  & -0.32 & 0.37 \\
        \midrule
        $\num{7e-5}$  & 0.0141 & 0.0019 & 0.085 & 0.018\\
                      & 0.027  & 0.006  & -0.5  & 0.4  \\
        \midrule
        $\num{1e-4}$  & 0.228  & 0.003  & 0.11  & 0.03 \\
                      & 0.055  & 0.012  & -1.2  & 0.8  \\
        \midrule
        $\num{2e-4}$  & 0.16   & 0.03   & -0.5  & 0.5  \\
                      & 0.16   & 0.05   & 0.4   & 4    \\

        \bottomrule
      \end{tabular}
\end{table}

\begin{figure}[H]
    \centering
    \includegraphics[scale= 0.8]{build/Turbo_Leck_1.pdf}
    \caption{Druck in Abhängigkeit der Zeit für die Leckratenmessung für die Turbomolekularpumpe mit linearer Regression. Der eingestellte Gleichgewichtsdruck beträgt $p_g=\SI{5e-5}{\hecto\pascal}$.}
    \label{fig:turbo_leck1}
\end{figure}
\noindent
\begin{figure}[H]
    \centering
    \includegraphics[scale= 0.8]{build/Turbo_Leck_2.pdf}
    \caption{Druck in Abhängigkeit der Zeit für die Leckratenmessung für die Turbomolekularpumpe mit linearer Regression. Der eingestellte Gleichgewichtsdruck beträgt $p_g=\SI{7e-5}{\hecto\pascal}$.}
    \label{fig:turbo_leck2}
\end{figure}
\noindent
\begin{figure}[H]
    \centering
    \includegraphics[scale= 0.8]{build/Turbo_Leck_3.pdf}
    \caption{Druck in Abhängigkeit der Zeit für die Leckratenmessung für die Turbomolekularpumpe mit linearer Regression. Der eingestellte Gleichgewichtsdruck beträgt $p_g=\SI{1e-4}{\hecto\pascal}$.}
    \label{fig:turbo_leck3}
\end{figure}
\noindent
\begin{figure}[H]
    \centering
    \includegraphics[scale= 0.8]{build/Turbo_Leck_4.pdf}
    \caption{Druck in Abhängigkeit der Zeit für die Leckratenmessung für die Turbomolekularpumpe mit linearer Regression. Der eingestellte Gleichgewichtsdruck beträgt $p_g=\SI{2e-4}{\hecto\pascal}$.}
    \label{fig:turbo_leck4}
\end{figure}
\noindent
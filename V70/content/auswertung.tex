%irgendwie muss ich mir noch was mit den Einheiten überlegen
%Tabellen in den Anhang?
\section{Auswertung der Messdaten}
\label{sec:Auswertung}
%#############################################################################################################
%#############################################################################################################
\subsection{Drehschieberpumpe}
%#############################################################################################################
%#############################################################################################################
Die Drehschieberpumpe besitzt laut Herstellerangaben ein Saugvermögen zwischen 
$\SI{4.6}{\cubic\metre\per\hour}$ und $\SI{5.5}{\cubic\metre\per\hour}$ \cite{Versuchsbeschreibung}.
Im Folgenden wird das Saugvermögen aus den aufgenommenen Daten der Evakuierungsmessung
und der Leckratenmessungen bestimmt. 
%%%%%%%%%%%%%%%%%%%%%%%%%%%%%%%%%%%%%%%%%%%%%%%%%%%%%%%%%%%%%%%%%%%%%%%%%%%%%%%%%%%%%%%%%%%%%%%%%%%%%%%%%%%%%%%
\subsubsection{Evakuierungsmessung}
%%%%%%%%%%%%%%%%%%%%%%%%%%%%%%%%%%%%%%%%%%%%%%%%%%%%%%%%%%%%%%%%%%%%%%%%%%%%%%%%%%%%%%%%%%%%%%%%%%%%%%%%%%%%%%%
Die nach Kapitel \ref{sec:Durchführung} aufgenommenen Daten sind im Anhang in Tabelle \ref{tab:Dreh_Evak}
aufgelistet. Auch die Mittelwerte der drei Messungen, die systematische Unsicherheiten durch die 
Messgeräte und die statistischen Unsicherheiten sind in Tabelle \ref{tab:Dreh_Evak} angegeben. 
Da die systematischen die statistischen Unsicherheiten deutlich dominieren, werden in den 
folgenden Berechnungen nur die systematischen Unsicherheiten betrachtet. Für die Zeitmessung 
wurde die durch die Reaktionszeit der Experimentierenden entstehende Unsicherheit als 
$\Delta t=\SI{0.2}{\second}$ angenommen. Auch diese Unsicherheit wird im Folgenden vernachlässigt.
Durch Umstellen von Gleichung \ref{eqn:evakuierungskurve} ergibt sich der Zusammenhang
\begin{equation*}
  \ln{\left[\frac{p-p_E}{p_0-p_E}\right]}=-\frac{S}{V}\cdot t
\end{equation*}
mit dem erreichten Enddruck ${p_E=\SI{7.1\pm 0.7 e-3}{\milli\bar}}$ und einem Umgebungsdruck
von ${p_0=\SI{1011 \pm 3}{\milli\bar}}$. 
Das Saugvermögen kann somit aus einer linearen Regression der Form 
\begin{equation}
  y=ax+b
  \label{eqn:gerade}
\end{equation}
bestimmt werden. Da das Saugvermögen jedoch vom Druck abhängig ist, werden mehrere 
Regressionen für verschiedene Druckbereiche durchgeführt. Alle Regressionen werden 
mit \textit{scipy} \cite{scipy} berechnet. Diese sind in Abbildung 
\ref{fig:dreh_evak} graphisch dargestellt, welche wie alle weiteren Abbildungen 
mit \textit{matplotlib} \cite{matplotlib} erstellt wurde.
\begin{figure}[H]
    \centering
    \includegraphics[width=0.9\linewidth]{build/Dreh_Evak.pdf}
    \caption{Evakuierungskurve für die Drehschieberpumpe in halblogarithmischer Darstellung mit mehreren linearen Regressionen für verschiedene Druckbereiche. Die Regressionsparameter und die zugehörigen Druckbereiche finden sich in Tabelle \ref{tab:Dreh_Evak_para}.}
    \label{fig:dreh_evak}
\end{figure}
\noindent
Die zugehörigen Parameter der Regressionen sind in Tabelle \ref{tab:Dreh_Evak_para} gegeben.
\begin{table}[H]
  \centering
    \caption{Regressionsparameter für die Evakuierungsmessung für die Drehschieberpumpe.}
    \label{tab:Dreh_Evak_para}
    \sisetup{table-format=1.3}
    \begin{tabular}{S[table-format=1.0] S[table-text-alignment=left] @{${}\quad\text{bis}\quad{}$} S[table-text-alignment=left] S[table-format=2.4] @{${}\pm{}$} S[table-format=1.4] S[table-format=2.2] @{${}\pm{}$} S[table-format=1.2]}
      \toprule
      {Regression} & \multicolumn{2}{c}{Druckbereich} & \multicolumn{2}{c}{$a [\si{\per\second}]$} & \multicolumn{2}{c}{$b$} \\
      \midrule
      1 &$\SI{1.01 \pm 0.10}{\bar}      $&$\SI{2.40 \pm 0.24}{\milli\bar} $ & -0.0307 & 0.0005 & -0.09 & 0.06\\
      2 &$\SI{2.00 \pm 0.20}{\milli\bar}$&$\SI{0.53 \pm 0.05}{\milli\bar} $ & -0.0110 & 0.0011 & -4.02 & 0.29\\
      3 &$\SI{0.49 \pm 0.05}{\milli\bar}$&$\SI{0.160\pm 0.016}{\milli\bar}$ & -0.0043 & 0.0004 & -6.25 & 0.17\\
      \bottomrule
    \end{tabular}
\end{table}
\noindent
Unter Ausnutzung von 
\begin{equation}
  S=-aV
  \label{eqn:evak}
\end{equation}
ergibt sich mit $V=\SI{34\pm 3.4}{\litre}$ \cite{Versuchsbeschreibung} die in Tabelle \ref{tab:Saug_Dreh_Evak}
aufgelisteten Saugleistungen in den jeweiligen Druckbereichen.
\begin{table}[H]
  \centering
    \caption{Mittelwerte der gemessenen Drücke bei der Leckratenmessung der Drehschieberpumpe mit statistischen und systematischen Unsicherheiten. Der Gleichgewichtsdruck beträgt $p_g=\SI{0.50 \pm 0.05}{\milli\bar}$.}
    \label{tab:Saug_Dreh_Evak}
    \sisetup{table-format=1.2}
    \begin{tabular}{S[table-text-alignment=left] @{${}\quad\text{bis}\quad{}$} S[table-text-alignment=left] S @{${}\pm{}$} S}
      \toprule
      \multicolumn{2}{c}{Druckbereich} & \multicolumn{2}{c}{$S_\text{Dreh} [\si{\cubic\metre\per\hour}]$}\\
      \midrule
      $\SI{1.01 \pm 0.10}{\bar}      $&$\SI{2.40 \pm 0.24}{\milli\bar} $ & 3.8  & 0.4 \\
      $\SI{2.00 \pm 0.20}{\milli\bar}$&$\SI{0.53 \pm 0.05}{\milli\bar} $ & 1.35 & 0.19\\
      $\SI{0.49 \pm 0.05}{\milli\bar}$&$\SI{0.160\pm 0.016}{\milli\bar}$ & 0.53 & 0.07\\
      \bottomrule
    \end{tabular}
\end{table}
%%%%%%%%%%%%%%%%%%%%%%%%%%%%%%%%%%%%%%%%%%%%%%%%%%%%%%%%%%%%%%%%%%%%%%%%%%%%%%%%%%%%%%%%%%%%%%%%%%%%%%%%%%%%%%%
\subsection{Leckratenmessung}
%%%%%%%%%%%%%%%%%%%%%%%%%%%%%%%%%%%%%%%%%%%%%%%%%%%%%%%%%%%%%%%%%%%%%%%%%%%%%%%%%%%%%%%%%%%%%%%%%%%%%%%%%%%%%%%
Die nach Kapitel \ref{sec:Durchführung} aufgenommenen Messdaten der Leckratenmessung sind 
für je einen Gleichgewichtsdruck $p_g$ in den Tabellen \ref{tab:Dreh_Leck1} bis \ref{tab:Dreh_Leck4}
gelistet. Analog zu der Evakuierungsmessung werden in den weiteren Berechnungen nur die systematischen
Unsicherheiten betrachtet, da diese die statistischen dominieren. 
%Tabellen?
Durch die Anpassung einer linearen Funktion \ref{eqn:gerade} an die Messdaten kann über den 
Zusammenhang \ref{eqn:leckratesaugvermögen} das Saugvermögen nach
\begin{equation}
  S=\frac{V_0}{p_G}\frac{\Delta p}{\Delta t}=\frac{V_0}{p_G}\cdot a
  \label{eqn:Leck2}
\end{equation} 
berechnet werden.
In den Abbildungen \ref{fig:dreh_leck1} bis \ref{fig:dreh_leck4} sind die Messdaten mit der 
berechneten Regression visualisiert.
Die Regressionsparameter sind in Tabelle \ref{tab:Dreh_Leck_para} enumiert.
\begin{table}[H]
  \centering
    \caption{Regressionsparameter für die Leckratenmessung für die Drehschieberpumpe.}
    \label{tab:Dreh_Leck_para}
    \sisetup{table-format=3.1}
    \begin{tabular}{S @{${}\pm{}$} S[table-format=1.2] S[table-format=1.4] @{${}\pm{}$} S[table-format=1.4] S[table-format=3.2] @{${}\pm{}$} S[table-format=1.2]}
      \toprule
      \multicolumn{2}{c}{$p_g [\si{\milli\bar}$]} & \multicolumn{2}{c}{$a [\si{\per\second}]$} & \multicolumn{2}{c}{$b$} \\
      \midrule
      0.50   & 0.05& 0.0102 & 0.0010 & 1.61  & 0.10\\
      10     & 1   & 0.374  & 0.005  & 15.85 & 0.33\\
      50.00  & 0.15& 1.753  & 0.021  & 57.8  & 1.3\\
      100.0  & 0.3 & 3.37   & 0.08   & 113.1 & 2.9\\
      \bottomrule
    \end{tabular}
\end{table}
\begin{figure}
    \centering
    \includegraphics[width=0.7\linewidth]{build/Dreh_Leck_1.pdf}
    \caption{Druck in Abhängigkeit der Zeit für die Leckratenmessung für die Drehschieberpumpe mit linearer Regression. Der eingestellte Gleichgewichtsdruck beträgt $p_g=\SI{0.50 \pm 0.05}{\milli\bar}$.}
    \label{fig:dreh_leck1}
\end{figure}
\begin{figure}
    \centering
    \includegraphics[width=0.7\linewidth]{build/Dreh_Leck_2.pdf}
    \caption{Druck in Abhängigkeit der Zeit für die Leckratenmessung für die Drehschieberpumpe mit linearer Regression. Der eingestellte Gleichgewichtsdruck beträgt $p_g=\SI{10 \pm 1}{\milli\bar}$.}
    \label{fig:dreh_leck2}
\end{figure}
\begin{figure}
    \centering
    \includegraphics[width=0.7\linewidth]{build/Dreh_Leck_3.pdf}
    \caption{Druck in Abhängigkeit der Zeit für die Leckratenmessung für die Drehschieberpumpe mit linearer Regression. Der eingestellte Gleichgewichtsdruck beträgt $p_g=\SI{50.00 \pm 0.15}{\milli\bar}$.}
    \label{fig:dreh_leck3}
\end{figure}
\begin{figure}
    \centering
    \includegraphics[width=0.7\linewidth]{build/Dreh_Leck_4.pdf}
    \caption{Druck in Abhängigkeit der Zeit für die Leckratenmessung für die Drehschieberpumpe mit linearer Regression. Der eingestellte Gleichgewichtsdruck beträgt $p_g=\SI{100.0 \pm 0.3}{\milli\bar}$.}
    \label{fig:dreh_leck4}
\end{figure}
\noindent
Durch Einsetzen in Gleichung \ref{eqn:Leck2} ergeben sich die Saugvermögen für die jeweiligen Gleichgewichtsdrücke.
Diese sind in Tabelle \ref{tab:Saug_Dreh_Leck} aufgelistet.
\begin{table}[H]
  \centering
    \caption{Berechneten Saugleistungen für die Leckratenmessung der Drehschieberpumpe für verschiedene Gleichgewichtsdrücke $p_g$.}
    \label{tab:Saug_Dreh_Leck}
    \sisetup{table-format=1.1}
    \begin{tabular}{S[table-format=3.1] @{${}\pm{}$} S[table-format=1.2] S @{${}\pm{}$} S}
      \toprule
      \multicolumn{2}{c}{$p_g [\si{\milli\bar}]$} & \multicolumn{2}{c}{$S [\si{\cubic\metre\per\hour}]$}\\
      \midrule
      0.5& 0.05& 2.5 & 0.4\\
      10 & 1   & 4.6 & 0.5\\
      50 & 0.15& 4.3 & 0.5\\
      100& 0.3 & 4.1 & 0.4\\
      \bottomrule
    \end{tabular}
\end{table}
%##############################################################################################################
%##############################################################################################################
\subsection{Turbomolekularpumpe}
%##############################################################################################################
%##############################################################################################################
Analog zu der Drehschieberpumpe wird für die Turbomolekularpumpe das Saugvermögen über die 
Evakuierungsmessung und über die Leckratenmessungen ermittelt. Laut Herstellerangaben
beträgt das Saugvermögen $S=\SI{77}{\litre\per\second}$ \cite{Versuchsbeschreibung}.
%%%%%%%%%%%%%%%%%%%%%%%%%%%%%%%%%%%%%%%%%%%%%%%%%%%%%%%%%%%%%%%%%%%%%%%%%%%%%%%%%%%%%%%%%%%%%%%%%%%%%%%%%%%%%%%
\subsubsection{Evakuierungsmessung}
%%%%%%%%%%%%%%%%%%%%%%%%%%%%%%%%%%%%%%%%%%%%%%%%%%%%%%%%%%%%%%%%%%%%%%%%%%%%%%%%%%%%%%%%%%%%%%%%%%%%%%%%%%%%%%%
Die Daten für die Evakuierungsmessung der Turbomolekularpumpe sind in Tabelle \ref{tab:Turbo_Evak}
aufgelistet.
%Tabelle?
Auch mit diesen Daten wird 
\begin{equation*}
  \ln{\left[\frac{p-p_E}{p_0-p_E}\right]}
\end{equation*}
gebildet, damit über mehrere Regressionen der Form \ref{eqn:gerade} die Saugvermögen für 
verschiedene Druckbereiche bestimmt werden können. Der erreichte Enddruck ist 
${p_E=\SI{7.7 \pm 2.3 e-6}{\milli\bar}}$. Der Startdruck $p_0$ ist bei dieser Messung nicht der 
Umgebungsdruck, sondern ${p_0=\SI{5.0 \pm 1.5 e-3}{\milli\bar}}$. Die Parameter der Regressionen 
sind in Tabelle \ref{tab:Turbo_Evak_para} aufgelistet. Die dazugehörigen Regressionen 
sind zudem in Abbildung \ref{fig:turbo_evak} dargestellt.
\begin{table}[H]
    \centering
      \caption{Regressionsparameter für die Leckratenmessung für die Turbomolekularpumpe.}
      \label{tab:Turbo_Evak_para}
      \sisetup{table-format=2.3}
      \begin{tabular}{S[table-format=1.0] S @{${}\quad\text{bis}\quad{}$} S S @{${}\pm{}$} S S[table-format=2.1] @{${}\pm{}$} S[table-format=1.1]}
        \toprule
        {Regression} & \multicolumn{2}{c}{Druckbereich} & \multicolumn{2}{c}{$a [\si{\per\second}]$} & \multicolumn{2}{c}{$b$} \\
        \midrule
        1 & $\SI{5.1 \pm 1.5 e-3}{\milli\bar}$ & $\SI{3.0 \pm 0.9 e-4}{\milli\bar}$ & -0.490 & 0.010 & -0.1 & 0.4\\
        2 & $\SI{3.0 \pm 0.9 e-4}{\milli\bar}$ & $\SI{3.7 \pm 1.1 e-5}{\milli\bar}$ & -0.19  & 0.04  & -1.8 & 0.5\\
        3 & $\SI{3.1 \pm 0.9 e-5}{\milli\bar}$ & $\SI{2.4 \pm 0.7 e-5}{\milli\bar}$ & -0.04  & 0.06  & -4.5 & 1.6\\
        4 & $\SI{2.2 \pm 0.6 e-5}{\milli\bar}$ & $\SI{1.3 \pm 0.4 e-5}{\milli\bar}$ & -0.010 & 0.006 & -5.6 & 0.5\\
        \bottomrule
      \end{tabular}
\end{table}
\begin{figure}[H]
    \centering
    \includegraphics[width=0.9\linewidth]{build/Turbo_Evak.pdf}
    \caption{Evakuierungskurve für die Turbomolekularpumpe in halblogarithmischer Darstellung mit mehreren linearen Regressionen für verschiedene Druckbereiche.}
    \label{fig:turbo_evak}
\end{figure}
\noindent
Das Saugvermögen ergibt sich aus Gleichung \ref{eqn:evak} mit einem Volumen von $V=\SI{33 \pm 3.3}{\litre}$.
Eine Auflistung der Saugvermögen ist durch Tabelle \ref{tab:Saug_Turbo_Evak} gegeben.
\begin{table}[H]
  \centering
    \caption{Mittelwerte der gemessenen Drücke bei der Leckratenmessung der Drehschieberpumpe mit statistischen und systematischen Unsicherheiten. Der Gleichgewichtsdruck beträgt $p_g=\SI{0.50 \pm 0.05}{\milli\bar}$.}
    \label{tab:Saug_Turbo_Evak}
    \sisetup{table-format=1.2}
    \begin{tabular}{S[table-text-alignment=left] @{${}\quad\text{bis}\quad{}$} S[table-text-alignment=left] S @{${}\pm{}$} S}
      \toprule
      \multicolumn{2}{c}{Druckbereich} & \multicolumn{2}{c}{$S_\text{Dreh} [\si{\cubic\metre\per\hour}]$}\\
      \midrule
      $\SI{5.1 \pm 1.5 e-3}{\milli\bar}$ & $\SI{3.0 \pm 0.9 e-4}{\milli\bar}$ & 58 & 13\\
      $\SI{3.0 \pm 0.9 e-4}{\milli\bar}$ & $\SI{3.7 \pm 1.1 e-5}{\milli\bar}$ & 22 &  6\\
      $\SI{3.1 \pm 0.9 e-5}{\milli\bar}$ & $\SI{2.4 \pm 0.7 e-5}{\milli\bar}$ &  5 &  8\\
      $\SI{2.2 \pm 0.6 e-5}{\milli\bar}$ & $\SI{1.3 \pm 0.4 e-5}{\milli\bar}$ & 1.2&0.8\\
      \bottomrule
    \end{tabular}
\end{table}
%%%%%%%%%%%%%%%%%%%%%%%%%%%%%%%%%%%%%%%%%%%%%%%%%%%%%%%%%%%%%%%%%%%%%%%%%%%%%%%%%%%%%%%%%%%%%%%%%%%%%%%%%%%%%%%
\subsection{Leckratenmessung}
%%%%%%%%%%%%%%%%%%%%%%%%%%%%%%%%%%%%%%%%%%%%%%%%%%%%%%%%%%%%%%%%%%%%%%%%%%%%%%%%%%%%%%%%%%%%%%%%%%%%%%%%%%%%%%%
In den Tabellen \ref{tab:Turbo_Leck1} bis \ref{tab:Turbo_Leck4} sind die aufgenommenen Daten, die 
Mittelwerte aus den Messungen und die systematischen und statistischen Unsicherheiten gelistet.
Erneut werden nur systematische Unsicherheiten betrachtet, da diese deutlich größer sind als die 
statistischen. 
%Tabelle
An die Messdaten werden durch Ausgleichsrechnung Funktionen der Form \ref{eqn:gerade} angepasst.
Da das Modell einer linearen Funktion die Daten jedoch unzureichend beschreibt, werden 
für jeden Gleichgewichtsdruck jeweils zwei Regressionen bestimmt, eine im Bereich niedriger 
und eine im Bereich höherer Drücke. Die berechneten Parameter sind in Tabelle \ref{tab:Turbo_Leck_para}
aufgelistete. Die Daten mit den Regressionen finden sich in den Abbildungen \ref{fig:turbo_leck1} 
bis \ref{fig:turbo_leck4}.
\begin{table}[H]
    \centering
      \caption{Regressionsparameter für die Leckratenmessung für die Turbomolekularpumpe.}
      \label{tab:Turbo_Leck_para}
      \sisetup{table-format=3.1}
      \begin{tabular}{c c @{${}\pm{}$} c c @{${}\pm{}$} c}
        \toprule
        {$p_g [\si{\milli\bar}$]} & \multicolumn{2}{c}{$a [\si{\per\second}]$} & \multicolumn{2}{c}{$b$} \\
        \midrule
        $\num{5.0 \pm 1.5 e-5}$  & 0.0102 & 0.0014 & 0.065 & 0.014\\
                      & 0.015  & 0.004  & -0.32 & 0.37 \\
        \midrule
        $\num{7.0 \pm 2.1 e-5}$  & 0.0141 & 0.0019 & 0.085 & 0.018\\
                      & 0.027  & 0.006  & -0.5  & 0.4  \\
        \midrule
        $\num{1.0 \pm 0.3 e-4}$  & 0.228  & 0.003  & 0.11  & 0.03 \\
                      & 0.055  & 0.012  & -1.2  & 0.8  \\
        \midrule
        $\num{2.0 \pm 0.6 e-4}$  & 0.16   & 0.03   & -0.5  & 0.5  \\
                      & 0.16   & 0.05   & 0.4   & 4    \\
        \bottomrule
      \end{tabular}
\end{table}
\begin{figure}[H]
    \centering
    \includegraphics[width=0.7\linewidth]{build/Turbo_Leck_1.pdf}
    \caption{Druck in Abhängigkeit der Zeit für die Leckratenmessung für die Turbomolekularpumpe mit linearer Regression. Der eingestellte Gleichgewichtsdruck beträgt $p_g=\SI{5 \pm 1.5 e-5}{\milli\bar}$.}
    \label{fig:turbo_leck1}
\end{figure}
\noindent
\begin{figure}[H]
    \centering
    \includegraphics[width=0.7\linewidth]{build/Turbo_Leck_2.pdf}
    \caption{Druck in Abhängigkeit der Zeit für die Leckratenmessung für die Turbomolekularpumpe mit linearer Regression. Der eingestellte Gleichgewichtsdruck beträgt $p_g=\SI{7 \pm 2.1 e-5}{\milli\bar}$.}
    \label{fig:turbo_leck2}
\end{figure}
\noindent
\begin{figure}[H]
    \centering
    \includegraphics[width=0.7\linewidth]{build/Turbo_Leck_3.pdf}
    \caption{Druck in Abhängigkeit der Zeit für die Leckratenmessung für die Turbomolekularpumpe mit linearer Regression. Der eingestellte Gleichgewichtsdruck beträgt $p_g=\SI{1 \pm 0.3 e-4}{\milli\bar}$.}
    \label{fig:turbo_leck3}
\end{figure}
\noindent
\begin{figure}[H]
    \centering
    \includegraphics[width=0.7\linewidth]{build/Turbo_Leck_4.pdf}
    \caption{Druck in Abhängigkeit der Zeit für die Leckratenmessung für die Turbomolekularpumpe mit linearer Regression. Der eingestellte Gleichgewichtsdruck beträgt $p_g=\SI{2 \pm 0.6 e-4}{\milli\bar}$.}
    \label{fig:turbo_leck4}
\end{figure}
\noindent
Durch Einsetzen der Parameter in Gleichung \ref{eqn:Leck2} ergeben sich die Saugvermögen, welche in 
Tabelle \ref{tab:Saug_Turbo_Leck} aufgelistet sind.
\begin{table}[H]
  \centering
    \caption{Regressionsparameter für die Leckratenmessung für die Turbomolekularpumpe.}
    \label{tab:Saug_Turbo_Leck}
    \sisetup{table-format=3.1}
    \begin{tabular}{c c @{${}\pm{}$} c c @{${}\pm{}$} c}
      \toprule
      {$p_g [\si{\milli\bar}$]} & \multicolumn{2}{c}{$S [\si{\cubic\metre\per\hour}]$} \\
      \midrule
      $\num{5.0 \pm 1.5 e-5}$  & 25 & 9 \\
                               & 38 & 16\\
      \midrule
      $\num{7.0 \pm 2.1 e-5}$  & 25 & 8 \\
                               & 46 & 18\\
      \midrule
      $\num{1.0 \pm 0.3 e-4}$  & 28 & 10\\
                               & 67 & 26\\
      \midrule
      $\num{2.0 \pm 0.6 e-4}$  & 98 & 35\\
                               & 97 & 42\\
      \bottomrule
    \end{tabular}
\end{table}
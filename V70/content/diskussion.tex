\section{Zusammenfassung und Diskussion der Ergebnisse}
\label{sec:Diskussion}
\begin{figure}[ht]
        \centering
        \includegraphics[width=\textwidth]{build/saug_dreh.pdf}
        \caption{Graphische Darstellung aller in Kapitel \ref{sec:Auswertung} berechneten Saugvermögen mit logarithmischer Druckskala für die Drehschieberpumpe. Der Wert für Evakuierung 1 wurde in einem Intervall von ${\SI{1.01 \pm 0.10}{\milli\bar}}$ bis ${\SI{2.40\pm 0.24}{\milli\bar}}$ bestimmt, Evakuierung 2 in einem Intervall von ${\SI{2.00 \pm 0.20}{\milli\bar}}$ bis ${\SI{0.53\pm 0.05}{\milli\bar}}$ und Evakuierung 3 in einem Intervall von ${\SI{0.49 \pm 0.05}{\milli\bar}}$ bis ${\SI{0.160 \pm 0.016}{\milli\bar}}$.}
        \label{fig:saug_dreh}
\end{figure}
\begin{figure}
        \centering
        \includegraphics[width=\textwidth]{build/saug_turbo.pdf}
        \caption{Graphische Darstellung aller in Kapitel \ref{sec:Auswertung} berechneten Saugvermögen mit logarithmischer Druckskala für die Turbomolekularpumpe. Der Wert für Evakuierung 1 wurde in einem Intervall von ${\SI{5.1 \pm 1.5 e-3}{\milli\bar}}$ bis ${\SI{3.0 \pm 0.9 e-4}{\milli\bar}}$ bestimmt, Evakuierung 2 in einem Intervall von ${\SI{3.0 \pm 0.9 e-4}{\milli\bar}}$ bis ${\SI{3.7 \pm 1.1 e-5}{\milli\bar}}$, Evakuierung 3 in einem Intervall von ${\SI{3.1 \pm 0.9 e-5}{\milli\bar}}$ bis ${\SI{2.4 \pm 0.7 e-5}{\milli\bar}}$ und Evakuierung 4 in einem Intervall von ${\SI{2.2 \pm 0.6 e-5}{\milli\bar}}$ bis ${\SI{1.3 \pm 0.4 e-5}{\milli\bar}}$.}
        \label{fig:saug_turbo}
\end{figure}
\noindent
Eine graphische Zusammenfassung aller berechneten Saugvermögen ist für die Drehschieberpumpe in 
Abbildung \ref{fig:saug_dreh} und für die Turbomolekularpumpe in Abbildung \ref{fig:saug_turbo} 
gegeben. Zudem sind die Herstellerangaben sind in den Abbildungen eingezeichnet.
\\
Die in Kapitel \ref{sec:Auswertung} betrachteten Daten für die Messungen der Drehschieberpumpe 
entsprechen weitestgehend den theoretischen Erwartungen. Somit beschreiben die gewählten 
Modelle der Regression die Daten im Rahmen der Fehler hinreichend. Bei den Leckratenmessungen
der Turbomolekularpumpe ist dies nicht der Fall, sodass die Datensätze in mehrere lineare
Bereiche aufgeteilt werden. Eine mögliche Erklärung ist die Druckabhängigkeit 
des Saugvermögens bei der Turbomolekularpumpe, welche durch die Änderung der mittleren freien
Weglänge der Gasteilchen entsteht. Ist die mittlere freie Weglänge der Gasteilchen
größer als der Abstand der Rotorschaufeln, nimmt das Saugvermögen der Pumpe ab. Somit ist zu 
erklären, dass das Saugvermögen wie in Abbildung \ref{fig:saug_turbo} für niedrige Drücke 
abfällt.
\\
Wie in Abbildung \ref{fig:saug_dreh} zu erkennen stimmen die ermittelten Saugbvermögen im Rahmen 
der systematischen Unsicherheiten im Grobvakuum um die \SI{100}{\milli\bar} mit den Herstellerangaben 
überein. Für niedrigere Drücke fällt das Saugvermögen wie zu erwarten ab. 
Die berechneten Saugvermögen für die Turbomolekularpumpe weichen deutlich von den Herstellerangaben
ab. Wie in Abbildung \ref{fig:saug_turbo} dargestellt, liegt die Herstellerangabe über den gemessenen Werten.
Eine Erklärung für diese Diskrepanz ist die Querschnittsverengung B4 direkt hinter der Turbomolekularpumpe,
da das Saugvermögen von dem Rohrdurchmesser abhängt.
Der Leitwert ist für molekulare Strömungen proportional zur dritten Potenz des Rohrdurchmessers und für 
laminare Strömungen proportional zu vierten Potenz. 
Somit senkt der verkleinerte Rohrdurchmesser den Leitwert und ach Gleichung \ref{eqn:saugvermögenleitwert} 
auch das effektive Saugvermögen. Der Einfluss dieser Abhängigkeit könnte 
durch Messungen mit anderen Rohrdurchmessern abgeschätzt werden. 
Des Weiteren führt in niedrigen Druckbereichen Desorption zu einer Verschlechterung des Vakuums, sodass
das effektive Saugvermögen sinkt. Unter anderem kann durch das Belüften der Anlage Wasser eindringen, 
welches während der Messung desorbiert. Idealerweise könnte der Versuchsaufbau mit reinem Stickstoff 
belüftet werden, um diesen Effekt zu vermeiden.
\\
Insgesamt können die Herstellerangaben der Drehschieberpumpe im Rahmen der systematischen Fehler 
bestätigt werden, bei der Turbomolekularpumpe sollten für eine ausführliche Diskussion des 
Saugvermögens auch weitere Rohrdurchmesser und eine weitere Verringerung von Desorptionseffekten 
eventuell durch gasbindende Vakuumpumpen in Betracht gezogen werden.
\newpage
\section{Zusammenfassung und Diskussion der Erebnisse}
\label{sec:Diskussion}
\begin{figure}[ht]
    \begin{subfigure}{0.8\textwidth}
            \centering
            \includegraphics[width=\textwidth]{build/saug_dreh.pdf}
            \caption{Graphische Darstellung aller in Kapitel \ref{sec:Auswertung} berechneten Saugvermögen mit logarithmischer Druckskala für die Drehschieberpumpe.}
            \label{fig:saug_dreh}
    \end{subfigure}
    \hfill
    \begin{subfigure}{0.8\textwidth}
            \centering
            \includegraphics[width=\textwidth]{build/saug_turbo.pdf}
            \caption{Graphische Darstellung aller in Kapitel \ref{sec:Auswertung} berechneten Saugvermögen mit logarithmischer Druckskala für die Turbomolekularpumpe.}
            \label{fig:saug_turbo}
    \end{subfigure}
\end{figure}
Eine graphische Zusammenfassung aller berechneten Saugvermögen ist für die Drehschieberpumpe in 
Abbildung \ref{fig:saug_dreh} und für die Turbomolekularpumpe in Abbildung \ref{fig:saug_turbo} 
gegeben. Auch die Herstellerangaben sind in Abbildung eingezeichnet.
\\
Die in Kaptiel \ref{sec:Auswertung} betrachteten Daten für die Messungen der Drehschieberpumpe 
entsprechen weitestgehend den theoretischen Erwartungen. Somit beschreiben die gewählten 
Modelle der Regression die Daten im Rahmen der Fehler hinreichend. Bei den Leckratenmessungen
der Turbomolekularpumpe ist dies nicht der Fall, sodass die Datensätze in mehrere lineare
Bereiche aufgeteilt werden. Eine mögliche Erklärung ist die vergleichsweise hohe Druckabhängigkeit 
des Saugvermögens bei der Turbomolekularpumpe. 
\\
Wie in Abbildung \ref{fig:saug_dreh} zu erkennen stimmen die ermittelten Saugbvermögen im Rahmen 
der systematischen Unsicherheiten in geeigneten Druckbereichen mit den Herstellerangaben überein.
Für niedrigere Drücke fällt das Saugvermögen wie zu erwarten ab. 
Die berechneten Saugvermögen für die Turbomolekularpumpe weichen deutlich von den Herstellerangaben
ab. Wie in Abbildung \ref{fig:saug_turbo} dargestellt, liegt die Herstellerangabe über den berechneten Werten.
Eine Erklärung für diese Diskrepanz ist die Querschnittsverengung B4 direkt hinter der Turbomolekularpumpe,
da das Saugvermögen von dem Rohrdurchmesser abhängt. Der Einfluss dieser Abhängigkeit könnte 
durch Messungen mit anderen Rohrdurchmessern abgeschätzt werden. In niedrigen Druckbereichen sind auch 
Effekte von virtuellen Pumpen nicht auszuschließen, welche durch Desorption das Vakuum verschlechtern 
und somit das effektive Saugvermögen senken. 
\\
Insgesamt können die Herstellerangaben der Drehschieberpumpe im Rahmen der systematischen Fehler 
bestätigt werden, bei der Turbomolekularpumpe sollten für eine ausführliche Diskussion des 
Saugvermögens auch weitere Rohrdurchmesser und eine weitere Verringerung von Desorptionseffekten 
eventuell durch Gasbindende Vakuumpumpen in Betracht gezogen werden.
\newpage
\section{Vorbereitung}
\subsection{Vakuum}
\begin{itemize}
    \item Druck innerhalb eines Gefäßes ist niedriger als Umgebungsdruck\\
    \to $\SI{300}{\milli\bar}$ ist minimaler Druck auf der Erde\\
    \to Vakuum heißt: $p<\SI{300}{\milli\bar}$
    \item absolutes Vakuum: vollständig leerer Raum
    \item Grobvakuum: $\num{1000}-\SI{1}{\milli\bar}$
    \item Feinvakuum: $\num{1}-\SI{e-3}{\milli\bar}$
    \item Hochvakuum: $\num{e-3}-\SI{e-8}{\milli\bar}$
    \item Ultrahoch-Vakuum: ab$\SI{e-8}{\milli\bar}$
\end{itemize}
\subsection{Ideale Gase}
\begin{itemize}
    \item sehr viele punktförmige Teilchen, die sich ungeordnet bewegen
    \item keine WW außer Stöße mit einander und mit Wand
    \item Zustandsgleichung: $pV=Nk_BT$
    \item für $Nk_BT=const.$ folgt Boyle-Mariottesches Gesetz: $p_1V_1=p_2V_2$
\end{itemize}

\subsection{Druck}
\begin{itemize}
    \item Druck: $p=-\left(\frac{\partial U}{\partial V}\right)_{S,N=const.}$
    \item Partialdruck: der Druck, den die einzelnen Gaskomponenten bei alleinigem Vorhandensein im Volumen hätten\\
    \to $p_{ges}=\sum_i p_i$ 
    \item Einheiten: $\SI{1}{\pascal}=\SI{1}{\newton\per\square\metre}=\SI{e-5}{\bar}$
    \item Teilchenzahl: $N=\frac{pV}{k_BT}$
    \item mittlere Teilchengeschwindigkeit: $2E_{kin}=mv^2=3k_BT\Rightarrow v=\sqrt{\frac{3k_BT}{m}}$\\
    \to Bolzmannverteilt
    \to entspricht ungefähr der Schallgeschwindigkeit
    \item mittlere freie Weglänge: mittlere Strecke zwischen zwei Stößen\\
    \to ungefähr $\SI{0.1}{\micro\metre}$
\end{itemize}
\subsection{Strömungen}
\begin{itemize}
    \item Laminare Strömung: Gas strömt in Schichten ohne Wirbel
    \item Turbolente Strömung: Gas strömt mit Wirbeln\\
    \to Teilchen stoßen vor allem mit einander
    \item Molekulare WW: so wenig Teilchen, dass diese kaum stoßen\\
    \to mittlere freie Weglänge > Weite des Strömungskanals
    \item Leitwert: $L=\frac{S_0S_{eff}}{S_0-S_{eff}}$\\
    \to Unterschied zwischen theoretischem und effektiven Saugvermögen
\end{itemize}
\subsection{Saugvermögen}
\begin{align*}
    &pV=c\\
    \iff& \frac{\partial p}{\partial t} V + p\frac{\partial V}{\partial t}=0\\
    \iff& \dot{V}=-\frac{V}{p}\dot{p}\\
    &S:=\dot{V}\\
    \Rightarrow& p(t)=p_0\exp{\left(-\frac{S}{V_0}t\right)} 
\end{align*}
Unter Beachtung des Enddrucks $p_E$ der Pumpe:
\begin{equation*}
    p(t)=(p_0-p_E)\exp{\left(-\frac{S}{V_0}t\right)}+p_E
\end{equation*}
\subsection{Leckrate}
\begin{align*}
    Q=V_0\frac{\partial p}{\partial t}
\end{align*}
Im Gleichgewichtszustand $p_G=const.$ gilt 
\begin{align*}
    &\dot{p}_+-\dot{p}_-=0\\
    \iff& pS=Q
\end{align*}
\subsection{Lecks}
\begin{itemize}
    \item Absorption: Stoff wird in Phase aufgenommen
    \item Adsorption: Stoff lägert sich auf Oberfläche an 
    \item Desorption: Teilchen verlassen Oberfläche und gehen in Gasphase
    \item Diffusion: Gleichverteilung durch Eigenbewegung der Teilchen\\
    \to Erhöhung der Entropie
    \item Leck: Prozess, der Vakuum reduziert
    \item Reale Lecks: Prozesse, die von außen messbar sind 
    \item virtuelle Lecks: von außen nicht zu messen\\
    \to Desorption innerhalb der Anlage
\end{itemize}
\subsection{Vakuumpumpen}
\begin{itemize}
    \item \textbf{Gastransfer-Pumpen}
    \item Verdränger-Pumpen\\
    \to nutzen Boyle-Mariotte aus\\
    \to Drehschiebervakuumpumpe
    \to hohe Druckbereiche
    \item kinetische Pumpen\\
    \to Teilchenbeschleunigung in Pumprichtung\\
    \to Turbomolekularpumpe
    \to niedrigere Druckbereiche
    \\
    \item \textbf{gasbindende Vakuumpumpen}\\
    \to bindet herumfliegende Gasteilchen\\
    \to sehr niedrige Druckbereiche
\end{itemize}
\subsection{Vakuummessung}
\begin{itemize}
    \item Piezo-Vakuummeter\\
    \to misst Kraft pro Oberfläche durch Piezo-Kristallen\\
    \to Piezo-Kristalle erzeugen bei Kompression elektrische Spannungen\\
    \to im Grobvakuum
    \item Pirani-Vakuummeter\\
    \to Wärmeleitfähigkeit $\propto$ Druck \\
    \to misst Wiederstand von Draht und somit Wärmeleitfähigkeit\\
    \to Draht kühlt sich je nach Umgebungsdruck unterschiedlich schnell ab \\
    \to im Feinvakuum
    \item Penning-Vakuummeter\\
    \to Kalt-Ionisations-Vakuummeter\\
    \to durch Feldemission werden Elektronen freigesetzt und fliegen zu Anode\\ 
    \to Gasatome werden ionisiert und fliegen zu Kathode\\
    \to Katthodenstrom ist Maß für Vakuum\\ 
    \to im Hoch- und Ultrahoch-Vakuum
    \item Bayard-Alpert-Vakuummeter\\
    \to Heiß-Ionisations-Vakuummeter\\
    \to Unterschied zu Penning: Elektronen werden thermisch emittiert
\end{itemize}
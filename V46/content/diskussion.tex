\section{Diskussion der Ergebnisse}
\label{sec:Diskussion}
Die Messung der magnetischen Flussdichte erscheint plausibel und zeigt ein
deutliches Maximum. Die Kurve der Messwerte weist keinerlei Sprünge oder
Ausreißer auf, was darauf schließen lässt, dass die Messfehler hier vermutlich
recht klein sind.
Der Literaturwert für die effektive Masse eines Elektrons in GaAs beträgt
 $ m_{\text{lit}}^{*} = 0.067 \cdot m_{e}$ \cite{effektiveMasse}.
Damit weicht das errechnete Ergebnis um $\SI{63}{\percent}$ ($m_{1}^{*}$)
beziehungsweise um $\SI{25}{\percent}$ ($m_{2}^{*}$) von dem  Literaturwert
ab. Die Abweichungen sind also gerade für die erste Probe sehr hoch. Auch
die gemessenen Winkeldifferenzen zeigen für die erste Probe keine lineare
Abhängigkeit von $\lambda^{2}$ sowie einen starken Ausreißer. Daher liegt hier
vermutlich ein Messfehler vor. Bei der zweiten Probe scheinen die
gemessenen Differenzen besser einem $\lambda^{2}$-Verlauf zu folgen, was
die geringere Abweichung der effektiven Masse berechnet auf Basis der
Messung der Probe 2 plausibler macht. Bei der Messung der Probe 1
kam es mehrmals zu einer Übersteuerung des Selektivverstärkers, was
die Messung verfälscht haben könnte. Außerdem waren einige Filter beschädigt,
etwa durch kleine Risse oder kaputte Befestigungen, sowie mit einem Schmutzfilm
belegt. Dies könnte weitere Fehler in der Messung erzeugt haben.
Auch haben vermutlich systematische Fehler die Messung beeinflusst. Die
Justage des Versuchsaufbaus konnte nur relativ grob durchgeführt werden,
da einige Skalen zur richtigen Ausrichtung der Instrumente zueinander
schwer sehr genau ablesbar waren.

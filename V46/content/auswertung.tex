\section{Auswertung der Messdaten}
\label{sec:Auswertung}
\subsection{Messung der Magnetflussdichte}
\label{subsec:Magnetflussdichte}
Um die maximale Magnetflussdichte im Bereich des Probenspaltes zu bestimmen,
ist in Abbildung \ref{fig:Magnetflussdichte} die magnetische Flussdichte gegen die Positionierung
der Hallsonde aufgetragen.
\begin{figure}[h]
  \centering
  \includegraphics[scale=0.7]{./build/B_Feld.pdf}
  \caption{Die magnetische Flussdichte aufgetragen gegen die Position der Hallsonde.}
  \label{fig:Magnetflussdichte}
\end{figure}
\noindent
Hier lässt sich ein Maximum sowohl bei $d=\SI{116}{\milli\meter}$ als auch bei
$d=\SI{117}{\milli\meter}$ erkennen. Das Maximum beträgt in beiden Fällen
$B_{\text{max}}=\SI{434}{\milli\tesla}$, was für die weitere Auswertung verwendet wird.

\subsection{Bestimmung der Rotationswinkel}
\label{subsec:rotationswinkel}
Es werden drei verschiedene GaAs Proben verwendet. Diese sind in Tabelle \ref{tab:proben}
aufgeführt.

\begin{table}[H]
  \centering
  \caption{Die verwendeten GaAs-Proben.}
  \label{tab:proben}
  \begin{tabular}{l c c c}
    \toprule
    {} & {Probe 1} & {Probe 2} & {Probe 3} \\
    \midrule
    Dotierung N [$\si{\per\centi\meter\cubed}$] & $ 1.2 \cdot 10^{18}$ & $2.8 \cdot 10^{-18}$ & - \\
    Dicke d [$\si{\milli\meter}$] & $1.36$ & $1.296$ & $5.1$ \\
    \bottomrule
  \end{tabular}
\end{table}
\noindent
Hierbei ist die Probe 3 hochreines GaAs, weshalb sie keine Dotierung aufweist.
Für jede der 3 Proben ist der Rotationswinkel bei verschiedenen Wellenlängen in
Tabelle \ref{tab:rotationswinkel} angegeben.

\begin{table}[H]
  \centering
  \caption{Die aufgenommenen Rotationswinkel für verschiedene Wellenlängen und Proben.}
  \label{tab:rotationswinkel}
  \sisetup{table-format=3.4}
  \begin{tabular}{S[table-format=1.3] S S S S S S}
    \toprule
    & \multicolumn{2}{c}{Probe 1} & \multicolumn{2}{c}{Probe 2} & \multicolumn{2}{c}{Probe 3} \\
    {$\lambda [\si{\nano\meter}]$} & {$\theta_{1} [\si{\degree}]$} & {$\theta_{2} [\si{\degree}]$} & {$\theta_{1} [\si{\degree}]$} & {$\theta_{2} [\si{\degree}]$} & {$\theta_{1} [\si{\degree}]$} & {$\theta_{2} [\si{\degree}]$}\\
    \midrule
    1.06 &    330.1667 & 340.2 &      329.9667 & 341.65 &     322.9334 & 345.8334 \\
    1.29 &    330.6667 & 340.0833 &   330.6 &    339.2667 &   326.5834 & 342.0 \\
    1.45 &    330.35 &   336.5 &      330.05 &   339.3 &      327.2834 & 339.1667 \\
    1.72 &    332.45 &   339.05 &     331.5167 & 341.3334 &   330.8334 & 339.7667 \\
    1.96 &    337.9334 & 343.5333 &   336.3167 & 347.0 &      337.9 &    343.0334 \\
    2.156 &   338.8667 & 345.7167 &   338.0167 & 349.85 &     339.6167 & 344.6 \\
    2.34 &    1.7 &      9.9167 &       0.05 &    13.5334 &     4.0334 &   8.4167 \\
    2.510 &   18.9667 &  15.0167 &     11.4167 &  34.25 &      28.6667 &  14.1 \\
    2.65 &    59.6334 &   68.3834 &    57.7667 &  71.4167 &   348.15 &   358.2334 \\
    \bottomrule
  \end{tabular}
\end{table}
\noindent
Für die weitere Auswertung werden die Messwerte aus Tabelle \ref{tab:rotationswinkel}
in Bogenmaß umgerechnet. Außerdem werden die Werte auf die jeweilige Probendicke
normiert. Dann ergibt sich der Rotationswinkel der Polarisationsebene nach
\ref{eqn:polarisationsebene}:

\begin{equation}
  \theta = \frac{\theta_{2} - \theta_{1}}{2}.
  \label{eqn:polarisationsebene}
\end{equation}
\noindent
Die sich ergebende Verteilung ist in Abbildung \ref{fig:drehwinkel} dargestellt.

\begin{figure}[H]
  \centering
  \includegraphics[scale=0.7]{./build/Drehwinkel.pdf}
  \caption{Der normierte Drehwinkel aufgetragen gegen das Quadrat der Wellenlänge.}
  \label{fig:drehwinkel}
\end{figure}

\subsection{Bestimmung der effektiven Masse}
\label{subsec:effektiveMasse}
Um die effektive Masse der freien Elektronen zu bestimmen, muss zuerst der Drehwinkel,
welcher aufgrund der Faraday-Rotation der freien Elektronen entsteht, bestimmt werden.
Dies geschieht, indem die Differenz der Faraday-Rotation der dotierten Probe und der
Faraday-Rotation der undotierten Probe genommen wird. Der verbleibende Drehwinkel
wird dann von den freien Elektronen verursacht:
\begin{equation}
  \theta_{\text{frei}} = \theta_{\text{dotiert}} - \theta_{\text{hochrein}}.
  \label{eqn:berechnungfrei}
\end{equation}
Die freien Drehwinkel sind in Abbildung \ref{fig:frei1} für die Probe 1 und in
\ref{fig:frei2} für die Probe 2 abgebildet. An die Werte wird eine lineare
Funktion der Form
\begin{equation}
  y = a \cdot x + b
  \label{eqn:linearerfit}
\end{equation}
angepasst. Dies wird durchgeführt mittels der \texttt{python}-Bibliothek \texttt{scipy.optimize} \cite{scipy}.
\begin{figure}[H]
  \centering
  \includegraphics[scale=0.7]{./build/Drehwinkel_frei_Probe1.pdf}
  \caption{Der freie Drehwinkel aufgetragen gegen das Quadrat der Wellenlänge für die Probe 1 mit linearem Fit.}
  \label{fig:frei1}
\end{figure}
\noindent
\begin{figure}[H]
  \centering
  \includegraphics[scale=0.7]{./build/Drehwinkel_frei_Probe2.pdf}
  \caption{Der freie Drehwinkel aufgetragen gegen das Quadrat der Wellenlänge für die Probe 2 mit linearem Fit.}
  \label{fig:frei2}
\end{figure}
\noindent
Da der Fit an die Messdaten der Probe 1 \ref{fig:frei1} durch einen Ausreißer
stark beeinflusst wird, wird der Ausreißer entfernt und der Fit erneut durchgeführt.
Dies ist in Abbildung \ref{fig:frei1v2} dargestellt.
\begin{figure}[H]
  \centering
  \includegraphics[scale=0.7]{./build/Drehwinkel_frei_Probe1v2.pdf}
  \caption{Der freie Drehwinkel aufgetragen gegen das Quadrat der Wellenlänge für die Probe 1 mit linearem Fit.}
  \label{fig:frei1v2}
\end{figure}
\noindent
Aus den Parametern des Fits kann die effektive Masse der freien Elektronen bestimmt werden,
da sie dem Proportionalitätsfaktor entsprechen:
\begin{equation*}
  \theta_{\text{frei}} (\lambda^{2}) = a \cdot \lambda^{2}.
\end{equation*}
Nach den Fits ergeben sich die Faktoren
\begin{align}
  a_{1} = & \SI{3.0(1.2)e12}{\per\meter\cubed} \\
  a_{2} = & \SI{1.2(0.3)e13}{\per\meter\cubed}.
\end{align}
Mit Gleichung \ref{eqn:winkel_masse} lässt sich zeigen, dass
\begin{equation}
  a = \frac{e_{0}^{3} N B}{8 \pi^{2} \epsilon_{0} c^{3} n (m^{*})^{2}}
  \label{eqn:faktor_a}
\end{equation}
gilt. Gleichung \ref{eqn:faktor_a} kann wiederum nach der effektiven Masse
umgestellt werden, um diese zu bestimmen:
\begin{equation}
  m^{*} = \sqrt{\frac{e_{0}^{3} N B}{8 \pi^{2} \epsilon_{0} c^{3} n a}}.
  \label{eqn:effektive_masse_aus_a}
\end{equation}
Dann ergibt sich für die effektive Masse, berechnet aus jeweils den
Messwerten der Probe 1 sowie der Probe 2:
\begin{align}
  m_{1}^{*} = & \SI{9.9(2.0)e-32}{\kilo\gram} = \num{0.109(0.022)} \cdot m_{e} \\
  m_{2}^{*} = & \SI{7.6(0.9)e-32}{\kilo\gram} = \num{0.084(0.010)} \cdot m_{e}.
\end{align}
Bei der Berechnung werden die Naturkonstanten der \texttt{python}-Bibliothek
\texttt{scipy.constants} verwendet. Für die magnetische Flussdichte wird die
in Abschnitt \ref{subsec:Magnetflussdichte} bestimmte maximale Flussdichte
$B_{\text{max}} = \SI{434}{\milli\tesla}$ genutzt. Der Wert für den
Brechungsindex $n$ beträgt $n = 3.857$ \cite{Brechungsindex}.

\section{Durchführung des Versuches}
\label{sec:Durchführung}
\subsection{Versuchsaufbau}
Im Folgenden wird die zur Messung verwendete Versuchsapparatur, die in Abbildung \ref{fig:apparatur} dargestellt ist, erklärt.

Die Lichtquelle ist durch eine Halogen-Lampe mit einem Emissionsspektrum, das überwiegend im Infrarotbereich liegt, realisiert. 
Zunächst wird das Licht durch eine Sammellinse gebündelt und dann anschließend in durch eine rotierende Sektorscheibe in 
einzelne Lichtimpulse zerteilt. 
Diese sogenannte Wechsellichtmethode hat den Vorteil, dass das Rauschen der Photowiderstände später in einem Selektivverstärker
herrausgefiltert werden kann.

Die einzelnen Lichtimpulse werden daraufhin in einem Glan-Thompson Prisma in zwei senkrecht 
zueinander polarisierte Teilstrahlen aufgespaltet. Es folgt ein Elektromagent mit einer Öffnung, in welcher die Probe eingeführt
wird. Nachdem das Licht die Probe durchdrungen hat wird mit einem Interferenzfilter eine ausgewählte Wellenlänge des Lichts 
selektiert, sodass es möglichst monochromatisch ist. Erneut wird der Lichtstrahl in einem Glan-Thompson Prisma in zwei 
senkrecht zueinander polarisierte Teile aufgespalten, die dann auf jeweils einen Photowiderstand gerichtet sind. Die Photowiderstände
übersetzen das optische Signal in eine elektrische Spannung. Die Spannungssignale Beider Photowiderstände werden dann an einen
Differenzverstärker weitergeleitet. Wenn beide Strahlen den gleichen Betrag und die gleiche Phase besitzen, verschwindet die Spannung
im Differenzverstärker. Über den bereits erwähnten Selektivverstärker wird das Signal an ein Oszilloskop weiter gegeben, sodass
es abgelesen werden kann.   

\begin{figure}[H]
    \centering
    \includegraphics[scale=0.5]{pictures/Versuchsaufbau.png}
    \caption{Schematische Darstellung der Versuchsapparatur. \cite{Versuchsbeschreibung}}
    \label{fig:apparatur}
\end{figure}

\subsection{Justage}
Um bei der Messung sinnvolle Ergebnisse zu erhalten, muss die Anlage vor Inbetriebnahme justiert werden. 

Eingangs werden die Probe und der Interferenzfilter aus der Apparatur entfernt, sodass die Funktionstüchtigkeit der
Polarisationsvorrichtung überprüft werden kann. Bei idealer Stellung des Prismas sollte die Lichtintensität in dem 
Austrittsfenster verschwinden. Zudem sollten die Strahle mittig auf die Photowiderstände treffen. 
Die rotierende Sektroscheibe sowie der Selektivverstärker werden auf eine Frequenz von $\SI{450}{\hertz}$ eingestellt. 
Der Gütefaktor wird auf den Wert $Q=\num{100}$ eingestellt. 
Nachdem die Probe und der Interferenzfilter wieder montiert wurden, wird der Polarisator so gedreht, dass die Spannung auf Null
abfällt. 

\subsection{Messprogramm}
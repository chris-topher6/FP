\section{Vorbereitung}

\subsection{Bandstruktur und der Begriff der \textit{effektiven Masse}}

\subsubsection*{Untschied der elektronischen Struktur zwischen Metallen, Halbleitern und Isolatoren}
\begin{enumerate}
\item Metall:     
    Fermienergie innerhalb eines Bandes \to e an Fermikante können Energie ändern
\item Isolator:   
    Fermienergie zwischen zwei Bändern  \to e können ihre Energie nicht ändern
\item Halbleiter: 
    wie Isolator mit sehr kleiner Bandlücke \to durch Anregung können e vom Valenz ins Leitungsband springen.
    Die Leitungselektronen können ihre Energie ändern.
    Leitfähigkeit kann so beeinflusst werden (z.B. durch Temperatur oder Dotierung)
\end{enumerate}


\subsubsection*{Konzept der effektiven Masse von freien Ladungsträgern in Festkörpern}
\begin{itemize}
\item effektive Masse ist ein Maß für die WW des e an das Kristallfeld
\item große effektive Masse \iff starke Kopplung an Kristallfeld \to das Potential ist schwach gekrümmt
\item mathematisch: 
    Taylorreihe von Energie
    \begin{equation*}
        \epsilon(\vec{k})=\epsilon(0)+\frac{1}{2}\sum_{i=1}^3\left(\frac{\partial \epsilon^2}{\partial k_i^2}\right)+O(k^4)
    \end{equation*}
    quadratischer Ausdruck unterscheidet sich in Ordnung O(k²) durch die effektive Masse
    vom Potential eines freien Teilchens.
    \begin{equation*}
        \epsilon_{V=0}=\frac{\hbar^2 k^2}{2m}
    \end{equation*}
    Die effektive Masse ist also 
    \begin{equation*}
        m_i^*=\frac{\hbar²}{\left(\frac{\partial \epsilon^2}{\partial k_i^2}\right)_{k=0}}.
    \end{equation*}
\end{itemize}

\subsection{Dotierung von Halbleitern}
\subsubsection*{Warum ist Dotierung nötig?}
\begin{itemize}
    \item Dotierung erhöht die Anzahl an elektrischen Ladungsträgern
        \to n-Dotierung erzeugt Erzeugt ein Elektron im Valenzband 
        \to p-Dotierung erzeugt ein Loch im Leitungsband 
    \item Leitfähigkeit des Halbleiters kann angepasst werden
\end{itemize}

\subsubsection*{Rolle der Donatoren und Akzeptoren, ihre Energieniveaus im Banddiagramm}
\begin{itemize}
    \item Donatoren besitzen ein Valenzelektron mehr als Halbleiter-Material
        \to n-Dotierung
        \to senkt Leitungsband
    \item Akzeptoren besitzen ein Valenzelektron weniger als Halbleiter-Material
        \to p-Dotierung 
        \to erhöht Valenzband
\end{itemize}
\begin{figure*}[H]
    \centering
    \includegraphics*[scale=0.8]{pictures/Dotierung.png}
\end{figure*}

\subsection{Faraday-Effek}
\subsubsection*{Definition des Effekts der Faraday-Rotation}
\begin{itemize}
    \item linear polarisierte Welle = Superposition zweier zikular polarisierter Wellen (gleiche Frequenz, entgegengesetzer Umlaufsinn)
        \begin{equation*}
            E(z)=\frac{1}{2}(E_R(z)+E_L(z))    
        \end{equation*}
    \item wenn Ausbreitungsrichtung parallel zum B-Feld unterscheiden sich (meist) die Brechungsindizes $n$
        \to unterschiedliche Wellenlängen
        \iff Polarisationsebene verdreht sich pro Schwingungsdauer $T$ um 
        \begin{equation*}
            \Delta\beta=\pi\cdot\left(\frac{n_{rechts}}{n_{links}}-1\right)
        \end{equation*} 
    \item Grund: induzierte elektrische Dipolmomente innerhalb des Materials erzeugen eine Polarisation des Materials
        \begin{equation*}
            \vec{P}=\epsilon_0 \chi\vec{E}
        \end{equation*}
        wobei $\chi$ eine 3x3-Matrix ist.
    \item Doppebrechung für 
        \begin{equation*}
            \chi = \left[ 
            \begin{array}{ccc}
                \chi_{xx}          & +i\chi_{xy} & 0         \\ 
                -i\chi_{xy}   & \chi_{yy}        & 0         \\
                0                  & 0                & \chi_{zz} \\ 
            \end{array}
            \right]
        \end{equation*}
\end{itemize}

\subsubsection*{Phänomenologische Beschreibung}

\subsection{Bestimmung der effektiven Masse von Ladungsträgern mittels des Faraday-Effekts}
\subsubsection*{Modelle und Ansatz.}
\subsubsection*{Wie beeinflussen Magnetfelder freie Elektronen?}
\begin{itemize}
    \item je nach Spin wird Energie erhöht oder erniedrigt
        \to Fermienergie bleibt gleich
        \iff 
\end{itemize}

\subsection{Detektionstechnik}
\subsubsection*{Warum ermöglicht die Modulation des Lichts in Kombination mit selektiver Verstärkung eine Unterdrückung des Signalrauschens?}

\subsection{Alternativer Ansatz zur Detektion} 
\subsection*{Was ist der Hauptvorteil der balancierten Detektion mit zwei Photodetektoren?}
\subsection*{Kann man die Faraday-Rotation auch mit nur einem Detektor messen?}

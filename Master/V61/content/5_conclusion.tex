\section{Conclusion}
The stability condition could not be thoroughly assessed due to certain limitations
encountered during the experiment. Firstly, the physical length of the optical bench
constrained the verification of the theoretical maximum during the measurement of the
first mirror configuration.
Secondly, during the measurement of the second configuration, the adjustment quality of
the laser varied and led to suboptimal results and prevented a measurement for higher
distances altogether. In both instances, it was not possible
to reach the theoretical limits.
Enhancements in the experimental setup could be made to overcome these challenges.
Utilizing a longer optical bench and implementing a method that allows for the mirrors’
repositioning without compromising their alignment could lead to a more consistent alignment
across varying resonator lengths.\\
\\
The measurement of the TEM-Modes could be done with sufficient accuracy, as both fits
show little deviation from the measured data (see Figure~\ref{fig:TEM-Messung1},~\ref{fig:TEM-Messung2}).
The uncertainties in the fit parameters as well as the estimated measurement uncertainty of
$\sigma_{I} = \SI{0.09}{\micro\ampere}$ collectively do not exceed $\SI{2.4}{\percent}$.\\
\\
The laser polarization exhibits the expected periodicity of $2\pi$,
with only minor deviations from the fit compared to the actual data, which can be seen
in Figure~\ref{fig:polarisation}.
The discrepancies are likely attributed to fluctuations that made it difficult reading
of the instrument. Moreover, the observed phase shift is presumably due to the imperfect
parallel alignment of the laser.\\
\\
The measurement of the longitudinal modes shows satisfactory accuracy in the initial
three measurements. A linear scaling of accuracy with the number of data points appears evident
from the observations for the first three resonator lengths. Limited statistics seem to be the
biggest challenge in this context, contributing to discrepancies in the results. Additionally,
the last measurement seems to be corrupted by the increased length of the resonator and the resulting
difficulties in the adjustment process. This leads to a more substantial deviation from the theoretical
values, which can be seen in Table~\ref{tab:deltaf}. \\
\\
The examination of the laser wavelength seems to be reasonably accurate for the first two measurements,
with the later two varying gravely from the theoretical values. Therefore, in calculating the mean of
all four measurements, the influence of each measurement was weighted by their uncertainty, as the later
two measurements also exhibit high uncertainty caused by lower statistics and larger, harder to measure
maximums in the screen. The weighted average $\bar{\lambda} = \SI{560.6150 \pm 3.5028}{\nano\meter}$
deviates from the theoretical value of $\lambda_{\text{Theory}}=\SI{632.8}{\nano\meter}$ by $\SI{11.41 \pm 0.55}{\percent}$.
This is most likely caused by the screen not being perfectly orthogonal to the optical axis as well as
imprecise measurements because of the need of darkening the room. Using a more sophisticated device to
measure distances than a normal tape measure, better results could probably be achieved. Also increasing
the width of the screen would allow more orders of diffraction to be measured, which could increase the accuracy of the
measurement, especially for the later two.

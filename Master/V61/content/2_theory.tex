\section{Theoretical Description of the functionality of a Laser}
\label{sec:theory}
\noindent
A laser (Light Amplification by Emission of Radiation) comprises
three essential components. These components include a laser medium, 
which in this instance is a gas mixture of helium and neon, a pump
source, specifically an electric discharger and an optical resonator, 
in this case one partially and one fully reflecting mirror.
\noindent
The laser medium is responsible for the laser's radiation spectrum, 
primarily due to its specific excitation energy. In the case 
of a HeNe laser, the active laser medium is the noble gas neon. 
This laser medium possesses various energy states, enabling three 
fundamental processes, as illustrated in Figure~\ref{fig:emission}.
\noindent
If the atom is in the ground state, it can absorb a photon with
the required energy for excitation, resulting in the atom having
one electron at a higher energy level. This process is known as 
absorption. An excited atom can spontaneously transition back 
to the lower energy ground state, emitting a photon with the remaining
energy, in line with energy conservation. The direction of the emitted
photon is random. This process is called spontaneous emission.
Another de-excitation process is stimulated emission. In this case
an excited atom is struck by a photon with precisely the same 
energy as the difference between the excited and ground state. 
This causes the atom to emit a photon and return to the ground state.
The incoming and emitted photons have the same energy, direction and 
phase, making them coherent.
\begin{figure}
    \centering
    \includegraphics[width=0.8\linewidth]{pictures/emission.png} %https://www.researchgate.net/figure/a-Spontaneous-Emission-b-Stimulated-Absorption-c-Stimulated-Emission_fig2_335822340
    \caption{Diagrammatic representation of absorption (a), spontaneous emission (b), and stimulated emission (c) for a two-state system.~\cite{emission}}
    \label{fig:emission}
\end{figure}
\noindent
In contrast to the illustration in Figure~\ref{fig:emission}, a laser
requires a laser medium with more than two energy levels. In a two
level system, the Einstein coefficients $B_{12}$ and $B_{21}$, which 
describe the transition probability between states, are equal. Due to 
this, the population inversion necessary for the laser is not achievable
in a tow-level system.
\noindent
To achieve population inversion, a pump source must input energy
into the system. In the case of a HeNe laser, an electric discharger 
pumps energy into the pump gas helium. The excited helium atoms 
collide with neon atoms, transferring their energy. The level diagram is
illustrated in Figure~\ref{fig:level}.
\begin{figure}
    \centering
    \includegraphics[width=0.7\linewidth]{pictures/level.png} %https://en.wikipedia.org/wiki/Helium%E2%80%93neon_laser
    \caption{Diagrammatic representation of the level schema for the HeNe lasers. \cite{Wikipedia}}
    \label{fig:level}
\end{figure}
The transition from the 3s to the p2 state predominates the process, 
resulting in the production of red laser light with a wavelength of
$\lambda = \SI{632.8}{\nano\meter}$.
\noindent
As amplification is dependent on the distance travelled, a resonator is
employed to increase the path length. This is implemented using two
mirrors, one fully reflecting and one partially reflecting, enabling
the beam to exit the laser tube in one direction.
\noindent
To control the laser, the resonator must satisfy the stability relation
\begin{equation}
    0 < g_1g_2 \leq 1
    \label{eqn:stability}
\end{equation}
\noindent
where $g_i=1-\frac{L}{r_i}$ is defined with the length $L$ and the 
curvature radius $r_i$ of the mirror $i$. If condition~\eqref{eqn:stability}
is not met, the laser beam may grow until it becomes larger than the mirror,
resulting in its loss. Figure~\ref{fig:resonator} displays two resonators,
one stable and one unstable.
\begin{figure}[H]
    \centering
    \includegraphics[scale=0.5]{pictures/resonator.png} %https://www.edmundoptics.de/knowledge-center/application-notes/lasers/laser-resonator-modes/
    \caption{Graphical illustration of the beam path in one resonator that fulfils the stability condition~\eqref{eqn:stability} (left) and an unstable resonator (right).~\cite{resonator}}
    \label{fig:resonator}
\end{figure}
\noindent
The light forms standing waves between the mirrors, denoted as $\text{TEM}_{pl}$,
where $p$ and $l$ describe the radial and angular mode orders. Figure 
\ref{fig:TEM} shows the patterns of the first orders.
%\begin{figure}[H]
%    \centering
%    \includegraphics[scale=0.3]{pictures/modes.png} %https://www.researchgate.net/figure/Different-Transverse-electromagnetic-mode-for-lasers-19_fig8_321225714
%    \caption{The intensity distribution of the laser modes $\text{TEM}_{00}$, $\text{TEM}_{01}$ and $\text{TEM}_{10}$. \cite{TEM}}
%    \label{fig:TEM}
%\end{figure}
\begin{figure}[H]
    \centering
    \includegraphics[scale=0.8]{pictures/modes2.jpg} %https://www.researchgate.net/figure/Different-Transverse-electromagnetic-mode-for-lasers-19_fig8_321225714
    \caption{The intensity distribution of the laser modes $\text{TEM}_{00}$, $\text{TEM}_{01}$, $\text{TEM}_{10}$, $\text{TEM}_{11}$ and $\text{TEM}_{02}$.~\cite{TEM2}}
    \label{fig:TEM}
\end{figure}
\noindent
The amplitude of the mode $\text{TEM}_{pl}$ is described by 
\begin{equation}
    A_{pl}(x,y)=\text{H}_l(x)\text{H}_m(y)\exp{-(x^2+y^2)}
    \label{eqn:TEM}
\end{equation}
where $\text{H}_i$ represents the Hermite polynomial of order $i$. 
The absolute square $|A_{pl}(x,y)|^2$ corresponds to the intensity
$I_{pl}(x,y)$.

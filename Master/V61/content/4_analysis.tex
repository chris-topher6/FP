\section{Analysis}
\subsection{Measurement of the stability condition}
As described in chapter~\ref{sec:theory}, in order to achieve a stable resonator
for laser emission, the stability condition~\ref{eqn:stability} must hold true.
The two mirror configurations available are Flat/Flat - Flat/$\SI{1400}{\milli\meter}$ and
Flat/$\SI{1400}{\milli\meter}$ - Flat/$\SI{1400}{\milli\meter}$. The stability condition~\ref{eqn:stability}
yields a maximum resonator length of $L_{1} = r_{2} = \SI{1400}{\milli\meter}$ for the former and
$L_{2} = r_{1} + r_{2} = \SI{2800}{\milli\meter}$ for the later configuration.
To test these theoretical calculations, the achieved laser intensity in different
mirror and length configurations is measured. The resulting data is displayed in figure~\ref{fig:distance-intesity}.
\begin{figure}[H]
 \centering
 \includegraphics[scale=0.8]{./build/Distanz-Intensität.pdf}
 \caption{Plot of the intensity measurement for different resonator configurations.}
 \label{fig:distance-intesity}
\end{figure}
\noindent
The data is given in tabular form in Table~\ref{tab:distance-intensity}.
The measurement was carried out until a consistent intensity measurement was not possible or the length of the
optical bench was reached.
\begin{table}[H]
    \centering
    \caption{Measurements of the Laser intensity for different resonator lengths and mirror configurations.}
    \label{tab:distance-intensity}
    \sisetup{table-format=3.2}
    \begin{tabular}{S S | S S}
        \toprule
        \multicolumn{2}{c|}{Flat/1400mm-Flat/1400mm} & \multicolumn{2}{c}{Flat/Flat-Flat/1400mm} \\
        {$d/\si{\centi\meter}$} & {$I/\si{\milli\watt}$} & {$d/\si{\centi\meter}$} & {$I/\si{\milli\watt}$} \\
        \midrule
        63.5 &  1.41    & 55.0 &   2.36 \\
        69.0 &  3.50    & 60.0 &   2.14 \\
        73.4 &  2.30    & 60.7 &   1.96 \\
        76.9 &  4.75    & 65.0 &   2.32 \\
        81.7 &  4.5     &  66.0 &   2.02 \\
        87.1 &  5.67    & 71.0 &   1.45 \\
        92.0 &  5.3     & 72.0 &   1.45 \\
        96.7 &  4.4     & 74.5 &   1.16 \\
        101.4 & 4.5     & 78.0 &   1.32 \\
        105.5 & 2.68    & 79.2 &   1.13 \\
        109.9 & 3.21    & 83.1 &   1.65 \\
        113.5 & 3.08    & 85.8 &   0.78 \\
        117.1 & 2.05    & 90.0 &   1.57 \\
        121.2 & 2.3     & 91.3 &   0.1 \\
        125.1 & 1.98    & 92.4 &   0 \\
        127.9 & 0.68    & 96.5 &   0 \\
        131.2 & 0.9     & 126.3 &  1.3  \\
        136.6 & 0.7     &       &       \\
        141.8 & 1.43        &       &       \\
        147.1 & 1.9     &       &       \\
        152.4 & 1.87        &       &       \\
        161.0 & 1.4     &       &       \\
        165.8 & 0.55        &       &       \\
        177.7 & 2.1     &       &       \\
        185.3 & 2.75        &       &       \\
        191.2 & 2.1     &       &       \\
        198.0 & 2.15        &       &       \\
        202.2 & 2.35        &       &       \\
        \bottomrule
    \end{tabular}
\end{table}
\noindent

\subsection{TEM Modes}
In this section, the $\text{TEM}_{00}$ and $\text{TEM}_{10}$ modes described in~\ref{sec:theory} are
observed by measuring the laser intensity perpendicular to the optical axis.
\subsubsection{\texorpdfstring{$\text{TEM}_{00}$}{TEM} Mode}
The amplitudes of the Modes are given by Equation~\ref{eqn:TEM}. Therefore the intensity of the $\text{TEM}_{00}$ mode
is described by
\begin{equation}
 I = I_{\text{max}} \cdot \exp{(\frac{-(r-r_{0})^{2}}{2 \omega^{2}})} + I_{0}.
 \label{eqn:TEM00-Fit}
\end{equation}
\noindent
Here, $I_{\text{max}}$ denotes the maximum intensity, $I_{0}$ the intensity contribution by background light, $r_{0}$ the
centre of the laser beam and $\omega$ describes the width of the gaussian curve. Measurement of the background light with the laser
being deactivated yielded an intensity of $I_{0} = \SI{0.0001}{\micro\ampere}$.
The parameters of \ref{eqn:TEM00-Fit} are fitted to the measurements using \texttt{scipy.optimize.curve\_fit};
the resulting values are shown below.
\begin{align*}
  I_{\text{max}} = & \SI{9.04 \pm 0.22}{\micro\ampere} \\
  r_{0} = & \SI{8.67 \pm 0.13}{\centi\meter} \\
  \omega = & \SI{4.72 \pm 0.13}{\centi\meter}.
\end{align*}
\noindent
The fitted curve and the data is displayed in Figure~\ref{fig:TEM-Messung1}.
\begin{figure}
  \centering
  \includegraphics[scale=0.8]{./build/TEM00-Mode+Pull.pdf}
  \caption{Intensity measurement of the $\text{TEM}_{00}$ mode.}
  \label{fig:TEM-Messung1}
\end{figure}
\noindent
\subsubsection{\texorpdfstring{$\text{TEM}_{01}$}{TEM} Mode}
The intensity of the $\text{TEM}_{10}$ is given by equation~\ref{eqn:TEM10-Fit}
\begin{equation}
 I = I_{\text{max}} \cdot \frac{4(r-r_{0})^{2}}{\omega^{2}} \cdot \exp{\bigl( \frac{-(r-r_{0})^{2}}{2\omega^{2}} \bigr)} + I_{0}.
 \label{eqn:TEM10-Fit}
\end{equation}
Equation \ref{eqn:TEM10-Fit} is again fitted to the measured data using \texttt{scipy.optimize.curve\_fit}, which yields the
parameters
\begin{align*}
  I_{\text{max}} = & \SI{0.91 \pm 0.02}{\micro\ampere} \\
  r_{0} = & \SI{6.54 \pm 0.10}{\centi\meter} \\
  \omega = & \SI{4.71 \pm 0.07}{\centi\meter}.
\end{align*}
The resulting curve is depicted in Figure~\ref{fig:TEM-Messung2}.
\begin{figure}
  \centering
  \includegraphics[scale=0.8]{./build/TEM10-Mode+Pull.pdf}
  \caption{Intensity measurement of the $\text{TEM}_{10}$ mode.}
  \label{fig:TEM-Messung2}
\end{figure}
\noindent
\subsection{Polarisation of the laser}
The Brewster windows mentioned in~\ref{sec:procedure} ensure that most light passing through is p-polarised. To verify this,
a measurement of the intensity dependence on the polarisation angle is caried out. The resulting data is given in
Table~\ref{tab:intensity-angles}.
The theoretical intensity curve is given by Equation~\ref{eqn:intensity-angle}
\begin{equation}
 I = I_{\text{max}} \cdot \sin{(\alpha + \alpha_{0})^{2}} + I_{0},
 \label{eqn:intensity-angle}
\end{equation}
where $I_{\text{max}}$ describes the maximum intensity, $\alpha_{0}$ the zero angle and $I_{0}$ again describes the intensity
contribution by background light. Using \texttt{scipy.optimize.curve\_fit}, this function is fitted to the measured data.
\begin{table}[H]
    \centering
    \caption{Measurements of the Laser intensity for different angles.}
    \label{tab:intensity-angles}
    \sisetup{table-format=3.0}
    \begin{tabular}{S S[table-format=1.2] | S S[table-format=1.2]}
        \toprule
      {$\alpha/\si{\degree}$} & {$I/\si{\milli\watt}$} & {$\alpha/\si{\degree}$} & {$I/\si{\milli\watt}$} \\
        \midrule
        0     &     0.15  & 180   &     0.13  \\
        10    &     0.39  & 190   &     0.34  \\
        20    &     0.72  & 200   &     0.66  \\
        30    &     1.13  & 210   &     0.13  \\
        40    &     1.42  & 220   &     1.41  \\
        50    &     1.75  & 230   &     1.75  \\
        60    &     1.97  & 240   &     1.96  \\
        70    &     2.05  & 250   &     2.08  \\
        80    &     2.13  & 260   &     2.12  \\
        90    &     2.00  & 270   &     1.86  \\
        100   &     1.72  & 280   &     1.68  \\
        110   &     1.46  & 290   &     1.15  \\
        120   &     1.08  & 300   &     1.04  \\
        130   &     0.67  & 310   &     0.68  \\
        140   &     0.39  & 320   &     0.37  \\
        150   &     0.14  & 330   &     0.14  \\
        160   &     0.02  & 340   &     0.02  \\
        170   &     0.02  & 350   &     0.02  \\
        \bottomrule
    \end{tabular}
\end{table}
\noindent

\subsection{Examination of the frequency spectrum of the laser}
\subsection{Laser wavelength}

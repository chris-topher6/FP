\section{Analysis}
\subsection{Measurement of the stability condition}
As described in chapter~\ref{sec:theory}, in order to achieve a stable resonator
for laser emission, the stability condition~\ref{eqn:stability} must hold true.
To test the theoretical calculations, the achieved laser intensity in different
mirror and length configurations is measured. The resulting data is displayed in figure~\ref{fig:distance-intesity}.
\begin{figure}[H]
 \centering
 \includegraphics[scale=0.8]{./build/Distanz-Intensität.pdf}
 \caption{Plot of the intensity measurement for different resonator configurations.}
 \label{fig:distance-intesity}
\end{figure}
\noindent
The data is given in tabular form in Table~\ref{tab:distance-intensity}.
The measurement was carried out until a consistent intensity measurement was not possible or the length of the
optical bench was reached.
\begin{table}[H]
    \centering
    \caption{Measurements of the Laser intensity for different resonator lengths and mirror configurations.}
    \label{tab:distance-intensity}
    \sisetup{table-format=3.2}
    \begin{tabular}{S S | S S}
        \toprule
        \multicolumn{2}{c|}{Flat/1400mm-Flat/1400mm} & \multicolumn{2}{c}{Flat/Flat-Flat/1400mm} \\
        {$d/\si{\centi\meter}$} & {$I/\si{\milli\watt}$} & {$d/\si{\centi\meter}$} & {$I/\si{\milli\watt}$} \\
        \midrule
        63.5 &  1.41    & 55.0 &   2.36 \\
        69.0 &  3.50    & 60.0 &   2.14 \\
        73.4 &  2.30    & 60.7 &   1.96 \\
        76.9 &  4.75    & 65.0 &   2.32 \\
        81.7 &  4.5     &  66.0 &   2.02 \\
        87.1 &  5.67    & 71.0 &   1.45 \\
        92.0 &  5.3     & 72.0 &   1.45 \\
        96.7 &  4.4     & 74.5 &   1.16 \\
        101.4 & 4.5     & 78.0 &   1.32 \\
        105.5 & 2.68    & 79.2 &   1.13 \\
        109.9 & 3.21    & 83.1 &   1.65 \\
        113.5 & 3.08    & 85.8 &   0.78 \\
        117.1 & 2.05    & 90.0 &   1.57 \\
        121.2 & 2.3     & 91.3 &   0.1 \\
        125.1 & 1.98    & 92.4 &   0 \\
        127.9 & 0.68    & 96.5 &   0 \\
        131.2 & 0.9     & 126.3 &  1.3  \\
        136.6 & 0.7     &       &       \\
        141.8 & 1.43        &       &       \\
        147.1 & 1.9     &       &       \\
        152.4 & 1.87        &       &       \\
        161.0 & 1.4     &       &       \\
        165.8 & 0.55        &       &       \\
        177.7 & 2.1     &       &       \\
        185.3 & 2.75        &       &       \\
        191.2 & 2.1     &       &       \\
        198.0 & 2.15        &       &       \\
        202.2 & 2.35        &       &       \\
        \bottomrule
    \end{tabular}
\end{table}
\noindent

\subsection{TEM Modes}
In this section, the $\text{TEM}_{00}$ and $\text{TEM}_{10}$ modes described in~\ref{sec:theory} are
observed by measuring the laser intensity perpendicular to the optical axis.
The amplitudes of the Modes are given by Equation~\ref{eqn:TEM}. Therefore the intensity of the $\text{TEM}_{00}$ mode
is described by
\begin{equation}
 I = I_{\text{max}} \cdot \exp{(\frac{-(r-r_{0})^{2}}{2 \omega^{2}})} + I_{0}.
 \label{eqn:TEM00-Fit}
\end{equation}
\noindent
% Using \texttt{scipy.optimize.curve_fit}, the exponential~\ref{eqn:TEM} is fitted to the measured data.


This is displayed in Figure~\ref{fig:TEM-Messung}.
\begin{figure}
  \centering
  \includegraphics[scale=0.8]{./build/TEM-Moden.pdf}
  \caption{Intensity measurement for two TEM Modes.}
  \label{fig:TEM-Messung}
\end{figure}
\noindent

\subsubsection{\texorpdfstring{$\text{TEM}_{00}$}{TEM} Mode}
\subsubsection{\texorpdfstring{$\text{TEM}_{01}$}{TEM} Mode}
\subsection{Polarisation of the laser}
\subsection{Examination of the frequency spectrum of the laser}
\subsection{Laser wavelength}

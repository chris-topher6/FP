\section{Preparation}
\subsection*{By what process do photons interact with matter?}
\begin{itemize}
    Photoelectric Effect
    \item $\gamma$ interacts with orbital $e^-$
    \item $E_\gamma > W$, $W$: workfunction
    \item $e^-$ leaves atom
    \item $\sigma_\text{Photo}\approx 3\cdot 10^{12} Z^\alpha E_\gamma^\delta$ 
    with $3<\alpha<4$ and $\delta\approx -3$
    \\
    Compton Effect
    \item inelastic scattering of $\gamma$ and quasifree $e^-$
    \item $\lambda'=\lambda+\lambda_\text{Compton}(1-\cos{\theta})$
    with $\lambda_\text{Compton}=h/(m_ec)$
    \item $\sigma_\text{Compton}=\frac{\pi\alpha^2}{m^2}\frac{1}{x^3}\left(\frac{2x(2+x(1+x)(8+x))}{(1+2x)^2}+((x-2)x-2)\log(1+2x)\right)$
    (Klein Nashina)
    \\
    Pair Production
    \item $\gamma\to e^-e^+$ in nucleus potential
    \item $E_\gamma > 2m_e$
    \item $\sigma_\text{Pair}\propto \alpha Z^2$
\end{itemize}
\subsection*{Which processes dominate at small or very large energies?}
\begin{itemize}
    \item Photoeffect dominates for small energies
    \item Pairproduction dominates for large energies
\end{itemize}
\subsection*{Which material properties have an influence on the 
probability of interaction?}
\begin{itemize}
    \item Atomic number $Z$
    \item Density $\rho$
    \item Thickness $d$
\end{itemize}
\subsection*{How is this related to the extinction coefficient? How does this depend on energy?}
\begin{itemize}
    \item extinction coefficient: measure of how strongly the material
    absorbs the photon
    \item high for low energies, because of Photoeffect
    \item high for very large energies, because of Pairproduction
\end{itemize}


\subsection*{What is the prediction for the differential cross section
as a function of energy for Compton scattering?}
\begin{itemize}
    \item $\frac{\text{d}\sigma}{\text{d}E}=\frac{\pi r_e^2}{m_ec^2\epsilon^2}\left(2+\left(\frac{E}{h\nu-E}\right)^2\left[\frac{1}{\epsilon^2}+\frac{h\nu-E}{h\nu}-\frac{2}{\epsilon}\left(\frac{h\nu-E}{h\nu}\right)\right]\right)$
\end{itemize}

\subsection*{What physical size of semiconductor limits the line width
of gamma lines?}
\begin{itemize}
    \item ???
\end{itemize}
\subsection*{Compare direct and indirect semiconductors.}
\begin{itemize}
    Direct:
    \item conduction band minimum and valence band maximum have the same k-vector 
    \item Electrons and holes recombine directly 
    \item higher absorption efficiency and faster response 
    \\
    Indirect: 
    \item momentum of electrons and holes is different in the conduction and valence band
    \item Electrons must pass through an intermediate state and transfer momentum to 
    the crystal lattice, making photon emission less probable
    
\end{itemize}
\subsection*{Which experimental effects lead to further line 
propagation?}
\begin{itemize}
    Doppler Broadening
    \item motion of particles (electron in Compton scattering) leads to Doppler shift in the energy 
    \\
    Multiscattering
    \item photon (in Compton scattering) can deposit more energy in the detector than with one collision hat $180°$  
    \\
    Intercation outside of the detector
    \item leads to missing energy
\end{itemize}

\subsection*{How are the signals from a germanium detector amplified?}
\begin{itemize}
    Preamplification
    \item amplification of signals
    \item convert signal (voltage <-> current)
    \\
    Filter
    \item minimize noie 
    \\
    Discriminator
    \item sharping the signal (gauss -> pulse)
\end{itemize}
\subsection*{What is the difference between charge and current 
amplifiers?}
\begin{itemize}
    Charge amplifier
    \item focus: measure the total charge produced by the gamma 
    \\
    Current amplifier
    \item focus: measure the current generated by the flow of charge carriers 
\end{itemize}

\subsection*{How are energy measuring detectors calibrated?}
\begin{itemize}
    \item calibration source \\
    \to gamma source with well-defined energies \\
    \to for example  $^{152}\symup{Eu}$
\end{itemize}

\subsection*{How can the full energy probability of germanium 
detectors be determined using a point source at a large distance?}
\begin{itemize}
    \item $Q=\frac{I_\text…{experiment}}{I_\text{theory}}$
    \item $I_\text{experiment} = \int_{-\infty}^{\infty} a \cdot \text{exp}\left(-\frac{\left(x - \mu\right)^2}{2\sigma^2}\right)\text{d}x = \sqrt{2 \pi} \cdot a \cdot \sigma$\\
    \to integral of the measured distribution
    \item $I_\text{theory} = P \cdot \frac{\Omega}{4 \pi} \cdot A \cdot t $ \\
    -> emissionprobalbility $P$, activity $A$, Solid angle component $\Omega$
\end{itemize}

\subsection*{What characteristic structures can be observed in a 
monochromatic gamme spectrum?}
\begin{itemize}
    Full energy preak (FEP)
    \item gammas deposit whole energy via photoeffect in the detector
    \\
    Compton Continuum and Edge
    \item Continuum with maximum energy at the compton edge below the FEP 
    \\ 
    Backscatter Peaks 
    \item gamms that undergo Compton scattering outsinde the detector but backscatter into the detector before being fully absorbed 
    \item at higher energies than the FEP 
    \\
    Escape Peaks 
    \item 1 or 2 gammas origining from positron annihilation escape from the detector 
\end{itemize}

\subsection*{What is the photon energy with the greatest probability
of emission for $^{152}\text{Eu}$, $^{137}\text{Cs}$ and $^{133}\text{Ba}$}
\begin{itemize}
    \item $^{152}\text{Eu}$: $\SI{121.78}{\kilo\electronvolt}$
    \item $^{137}\text{Cs}$: $\SI{661.66}{\kilo\electronvolt}$
    \item $^{133}\text{Ba}$: $\SI{356.01}{\kilo\electronvolt}$
\end{itemize}

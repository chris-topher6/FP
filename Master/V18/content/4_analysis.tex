\section{Analysis}
\subsection{Detector Calibration}

To calibrate the germanium detector, the spectrum of an $^{152}\text{Eu}$ source is measured;
this can be seen in Figure~\ref{fig:spectrumeu}.
\begin{figure}
  \centering
  \begin{adjustbox}{width=1.4\textwidth, center}
  \includegraphics[scale=0.3]{./build/Europium-Peaks.pdf}
  \end{adjustbox}
  \caption{Spectrum of an $^{152}\text{Eu}$ source.}
  \label{fig:spectrumeu}
\end{figure}
\noindent
To calibrate the germanium detector, an energy is assigned to each detector channel.
The 13 most prominent full energy peaks are selected and compared to the 13 most probable emission
lines of the known $^{152}\text{Eu}$ spectrum. The most prominent full enery peaks are marked in
orange in Figure~\ref{fig:spectrumeu} and given in Table~\ref{tab:calibrationvalues}.
A linear fit
\begin{equation}
 E(K) = \alpha \cdot K + \beta
\end{equation}
is performed. The resulting fit parameters are
\begin{align*}
  \alpha = &  \num{0.218 \pm 0.000}\\
  \beta = &  \num{104.803 \pm 0.000}.
\end{align*}
The fit is displayed in Figure~\ref{fig:calibrationfit}.
\begin{figure}
  \centering
  \begin{adjustbox}{width=1.4\textwidth, center}
  \includegraphics[scale=0.3]{./build/Europium-Fit.pdf}
  \end{adjustbox}
  \caption{Fit of the detector channels and corresponding energy.}
  \label{fig:calibrationfit}
\end{figure}
\noindent
\begin{table}
  \centering
  \caption{Energy and emission probability of the full energy peaks as well as their assigned channel number.}
  \sisetup{table-format=3.2}
  \begin{tabular}{S S S}
    \toprule{$E$ & }

  \end{tabular}
\end{table}

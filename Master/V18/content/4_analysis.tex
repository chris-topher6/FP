\section{Analysis}
\subsection{Detector Calibration}
To calibrate the germanium detector, the spectrum of an $^{152}\text{Eu}$
source is measured; this can be seen in Figure~\ref{fig:spectrumeu}. A
background measurement is conducted and removed from the $^{152}\text{Eu}$
measurement. For this, the background measurement is scaled according to
the difference in measurement time compared to the $^{152}\text{Eu}$
measurement. The background can be seen in figure~\ref{fig:background} and
the scaled background as well as the $^{152}\text{Eu}$ measurement in
figure~\ref{fig:eubackground}.
\begin{figure}
	\centering
	\begin{adjustbox}{width=1.4\textwidth, center}
		\includegraphics[scale=0.3]{./build/Untergrund.pdf}
	\end{adjustbox}
	\caption{Spectrum of background radiation.}
	\label{fig:background}
\end{figure}
\begin{figure}
	\centering
	\begin{adjustbox}{width=1.4\textwidth, center}
		\includegraphics[scale=0.3]{./build/Europium-Untergrund.pdf}
	\end{adjustbox}
	\caption{$^{152}\text{Eu}$ measurement and scaled spectrum of background radiation.}
	\label{fig:eubackground}
\end{figure}
\begin{figure}
	\centering
	\begin{adjustbox}{width=1.4\textwidth, center}
		\includegraphics[scale=0.3]{./build/Europium-Peaks.pdf}
	\end{adjustbox}
	\caption{Spectrum of an $^{152}\text{Eu}$ source and its most prominent peaks.}
	\label{fig:spectrumeu}
\end{figure}
\noindent
To calibrate the germanium detector, an energy is assigned to each detector channel.
The 13 most prominent full energy peaks are selected and compared to the 13 most probable emission
lines of the known $^{152}\text{Eu}$ spectrum~\cite{laraweb}. The most prominent full enery peaks are marked in
orange in Figure~\ref{fig:spectrumeu} and given in Table~\ref{tab:calibrationvalues}. The prominence
calculation is performed using the algorithm \texttt{find\_peaks} from the python library \texttt{scipy} \cite{scipy}.
A linear fit
\begin{equation}
	E(K) = \alpha \cdot K + \beta
	\label{eqn:linear}
\end{equation}
is performed to get a relation between energy $E$ and channel number $K$. The resulting fit parameters are
\begin{align*}
	\alpha & =  \num{0.218 \pm 0.000}       \\
	\beta  & =  \num{104.803 \pm 0.000292}.
\end{align*}
The fit is displayed in Figure~\ref{fig:calibrationfit}.
\begin{figure}
	\centering
	\begin{adjustbox}{width=1.4\textwidth, center}
		\includegraphics[scale=0.3]{./build/Europium-Fit.pdf}
	\end{adjustbox}
	\caption{Fit of the detector channels and corresponding energy.}
	\label{fig:calibrationfit}
\end{figure}
\noindent
The relation~\ref{eqn:linear} with the determined parameters is used thoughout this analysis for energy measurements. The fit is performed
using the \texttt{iminuit}~\cite{iminuit} python package.
\begin{table}[H]
	\centering
	\caption{Energy and emission probability~\cite{laraweb} of the full energy peaks, their assigned channel number as well as the yield of each peak, determined by a fit.}
	\label{tab:calibrationvalues}
	\sisetup{table-format=4.4}
	\begin{tabular}{S S[table-format=2.2] S[table-format=4.0] S}
		\toprule {$E_{\text{lit}}/\si{\kilo\electronvolt}$} & {$P_{\text{lit}}/\si{\percent}$} & {$C$} & {$N$}                 \\
		\midrule
		121.7817                                            & {28.410 \pm  0.13}               & 198   & {842.958 \pm 37.370}  \\
		244.6974                                            & {7.550 \pm   0.04}               & 225   & {866.660 \pm 47.906}  \\
		344.2785                                            & {26.590 \pm  0.12}               & 595   & {8198.356 \pm 91.156} \\
		411.1165                                            & {2.238 \pm   0.010}              & 1187  & {1560.251 \pm 39.797} \\
		443.9650                                            & {2.800 \pm   0.02}               & 1666  & {3338.623 \pm 57.831} \\
		778.9045                                            & {12.970 \pm  0.06}               & 1989  & {304.132 \pm 18.265}  \\
		867.3800                                            & {4.243 \pm   0.023}              & 2145  & {438.597 \pm 21.296}  \\
		964.0790                                            & {14.500 \pm  0.06}               & 3760  & {751.993 \pm 29.308}  \\
		1085.8370                                           & {10.130 \pm  0.06}               & 4192  & {275.474 \pm 20.183}  \\
		1089.7370                                           & {1.730 \pm   0.01}               & 4654  & {631.712 \pm 25.596}  \\
		1112.0760                                           & {13.410 \pm  0.06}               & 5238  & {438.699 \pm 23.124}  \\
		1299.1420                                           & {1.633 \pm   0.009}              & 5370  & {562.977 \pm 25.103}  \\
		1408.0130                                           & {20.850 \pm  0.08}               & 6797  & {640.583 \pm 27.287}  \\
		\bottomrule
	\end{tabular}
\end{table}
\begin{figure}[H]
	% Erste Zeile
	\centering
	\includegraphics[width=\linewidth]{./build/Europium-Fit-Peak1.pdf}
	\caption{Fit of the first peak of the $^{152}\text{Eu}$ spectrum.}
\end{figure}

\begin{figure}[H]
	\centering
	% Erste Zeile
	\begin{adjustbox}{width=1.3\textwidth, center}
		\begin{subfigure}{.5\textwidth}
			\centering
			\includegraphics[width=\linewidth]{./build/Europium-Fit-Peak2.pdf}
			% \caption{Plot 1}
		\end{subfigure}%
		\begin{subfigure}{.5\textwidth}
			\centering
			\includegraphics[width=\linewidth]{./build/Europium-Fit-Peak3.pdf}
			% \caption{Plot 2}
		\end{subfigure}
	\end{adjustbox}
	% Zweite Zeile
	\begin{adjustbox}{width=1.3\textwidth, center}
		\begin{subfigure}{.5\textwidth}
			\centering
			\includegraphics[width=\linewidth]{./build/Europium-Fit-Peak4.pdf}
			% \caption{Plot 1}
		\end{subfigure}%
		\begin{subfigure}{.5\textwidth}
			\centering
			\includegraphics[width=\linewidth]{./build/Europium-Fit-Peak5.pdf}
			% \caption{Plot 2}
		\end{subfigure}
	\end{adjustbox}
	% Dritte Zeile
	\begin{adjustbox}{width=1.3\textwidth, center}
		\begin{subfigure}{.5\textwidth}
			\centering
			\includegraphics[width=\linewidth]{./build/Europium-Fit-Peak6.pdf}
			% \caption{Plot 1}
		\end{subfigure}%
		\begin{subfigure}{.5\textwidth}
			\centering
			\includegraphics[width=\linewidth]{./build/Europium-Fit-Peak7.pdf}
			% \caption{Plot 2}
		\end{subfigure}
	\end{adjustbox}
	\caption{Fits of peaks of the $^{152}\text{Eu}$ spectrum.}
\end{figure}
\begin{figure}[H]
	% Vierte Zeile
	\begin{adjustbox}{width=1.3\textwidth, center}
		\begin{subfigure}{.5\textwidth}
			\centering
			\includegraphics[width=\linewidth]{./build/Europium-Fit-Peak8.pdf}
			% \caption{Plot 1}
		\end{subfigure}%
		\begin{subfigure}{.5\textwidth}
			\centering
			\includegraphics[width=\linewidth]{./build/Europium-Fit-Peak9.pdf}
			% \caption{Plot 2}
		\end{subfigure}
	\end{adjustbox}
	% Fünfte Zeile
	\begin{adjustbox}{width=1.3\textwidth, center}
		\begin{subfigure}{.5\textwidth}
			\centering
			\includegraphics[width=\linewidth]{./build/Europium-Fit-Peak10.pdf}
			% \caption{Plot 1}
		\end{subfigure}%
		\begin{subfigure}{.5\textwidth}
			\centering
			\includegraphics[width=\linewidth]{./build/Europium-Fit-Peak11.pdf}
			% \caption{Plot 2}
		\end{subfigure}
	\end{adjustbox}
	% Sechste Zeile
	\begin{adjustbox}{width=1.3\textwidth, center}
		\begin{subfigure}{.5\textwidth}
			\centering
			\includegraphics[width=\linewidth]{./build/Europium-Fit-Peak12.pdf}
			% \caption{Plot 1}
		\end{subfigure}%
		\begin{subfigure}{.5\textwidth}
			\centering
			\includegraphics[width=\linewidth]{./build/Europium-Fit-Peak13.pdf}
			% \caption{Plot 2}
		\end{subfigure}
	\end{adjustbox}
	\caption{Fits of peaks of the $^{152}\text{Eu}$ spectrum.}
\end{figure}


\subsection{The full energy detection probability}
The amount of photons which fully penetrate and are registered by the detector is called the full energy
detection probability $Q$. It is given by
\begin{equation}
	Q = \frac{4 \pi}{\Omega}\frac{N}{APt}.
	\label{eqn:fullenergy}
\end{equation}
$Q$ depends on the activity of the source $A$, the emission probability $P$, the number of measured photons $N$ and
the measurement time $t$. $N$ and $P$ are given in
table~\ref{tab:calibrationvalues}. $\Omega$ is the solid angle covered by the
detector. For the germanium detector, $\Omega$ is given by
\begin{equation}
	\frac{\Omega}{4 \pi} = \frac{1}{2} \left( 1 - \frac{a}{\sqrt{a^{2} + r^{2}}} \right)
	\label{eqn:solidangle}
\end{equation}
with the detectors radius $r$ and the distance to the probe $a$.

\subsubsection{Determination of the probe activity}
\label{subsubsec:activity}
To determine the full energy detection probability, first the activation of the $^{152}\text{Eu}$ source has to be estimated.
The source had an activation of $A_{0} = \SI{4130 \pm 60}{\becquerel}$ on the 1.10.2000. According to the law of
radioactivate decay,
\begin{equation}
	A(t) = A_{0} \cdot \exp{\left( - \frac{\ln{(2)}}{\tau} \cdot t \right)}
	\label{eqn:activation}
\end{equation}
with elapsed time $t$ and half life $\tau$,
the activation on the day of the measurement of $^{152}\text{Eu}$ amounts to
$A(t_{\text{M}}) = \SI{1262 \pm 18}{\becquerel}$.

\subsubsection{Determination of the solid angle}
\label{subsubsec:solidangle}
The solid angle $\frac{\Omega}{4 \pi}$ covered by the germanium detector is given
by equation~\ref{eqn:solidangle}. The detector cylinder has a width of
$\SI{4.5}{\centi\meter}$. The source is located $\SI{6.91}{\centi\meter}$
away from the aluminium cover; the cover itself is $\SI{1.5}{\centi\meter}$
away from the detector. Therefore, the parameters in
equation~\ref{eqn:solidangle} are given by
\begin{align}
	r & =  \SI{2.25}{\centi\meter}  \\
	a & =  \SI{8.91}{\centi\meter}.
	\label{eqn:params}
\end{align}
Using equation~\ref{eqn:solidangle} and~\ref{eqn:params} the solid angle
amounts to
\begin{equation}
	\frac{\Omega}{4 \pi} = 0.01522.
\end{equation}

\subsubsection{Determination of the full energy detection probability}
\label{subsubsec:determinationfep}
The estimation of the full energy detection probability is performed for
all measured spectral lines with more than $\SI{150}{\kilo\electronvolt}$,
which is the limit of the detector sensitivity.

\begin{table}[H]
	\centering
	\caption{Energy and the calculated full energy detection probability of the 12 peaks of the $^{152}\text{Eu}$ spectrum with more than $\SI{150}{\kilo\electronvolt}$.}
	\label{tab:calibrationvalues}
	\sisetup{table-format=4.4}
	\begin{tabular}{S  S[table-format=1.5] }
		\toprule {$E_{\text{lit}}/\si{\kilo\electronvolt}$} & {$Q/\si{\percent}$}     \\
		\midrule
		% 121.7817                                            & { 0.000552 \pm  0.000026} \\
		244.6974                                            & { 0.02134 \pm  0.0012}  \\
		344.2785                                            & { 0.05731 \pm  0.0010}  \\
		411.1165                                            & { 0.12959 \pm  0.0038}  \\
		443.9650                                            & { 0.22165 \pm  0.0052}  \\
		778.9045                                            & { 0.00436 \pm  0.00027} \\
		867.3800                                            & { 0.01922 \pm  0.00098} \\
		964.0790                                            & { 0.00964 \pm  0.0004}  \\
		1085.8370                                           & { 0.00506 \pm  0.00038} \\
		1089.7370                                           & { 0.06788 \pm  0.0029}  \\
		1112.0760                                           & { 0.00608 \pm  0.00033} \\
		1299.1420                                           & { 0.06409 \pm  0.0030}  \\
		1408.0130                                           & { 0.00571 \pm  0.00026} \\
		\bottomrule
	\end{tabular}
\end{table}
\noindent
To obtain the energy dependency of full energy detection probability, a fit using the function
\begin{equation}
	Q(E) = a \cdot E^{b}
	\label{eqn:fedpfit_function}
\end{equation}
is performed. The result is displayed in Figure~\ref{fig:fedpfit}.
\begin{figure}
  \centering
  \includegraphics[width=\linewidth]{./build/FEDP-Fit.pdf}
  \caption{Fit of the full energy detection probabilities for different peaks of the $^{152}\text{Eu}$ spectrum.}
  \label{fig:fedpfit}
\end{figure}
Values which seem to be far out of the expected value range are excluded from the fit (for details, see Section~\ref{sec:conclusion}). The resulting fit parameters are
\begin{align*}
  a & = \num{22.673 \pm 5.031} \\
  b & = \num{-0.821 \pm 0.033}.
\end{align*}
The relation~\ref{eqn:fedpfit_function} with the determined parameters is used to estimate the full energy detection
probability throughout this analysis.

\subsection{Monochromatic gamma spectrum}

\section{Analysis}
\subsection{Detector Calibration}
To calibrate the germanium detector, the spectrum of an $^{152}\text{Eu}$
source is measured; this can be seen in Figure~\ref{fig:spectrumeu}. A
background measurement is conducted and removed from the $^{152}\text{Eu}$
measurement. For this, the background measurement is scaled according to
the difference in measurement time compared to the $^{152}\text{Eu}$
measurement. The background can be seen in figure~\ref{fig:background} and
the scaled background as well as the $^{152}\text{Eu}$ measurement in
figure~\ref{fig:eubackground}.
\begin{figure}
	\centering
	\begin{adjustbox}{width=1.4\textwidth, center}
		\includegraphics[scale=0.3]{./build/Europium-Peaks.pdf}
	\end{adjustbox}
	\caption{Spectrum of an $^{152}\text{Eu}$ source and its most prominent peaks.}
	\label{fig:spectrumeu}
\end{figure}
\begin{figure}
	\centering
	\begin{adjustbox}{width=1.4\textwidth, center}
		\includegraphics[scale=0.3]{./build/Untergrund.pdf}
	\end{adjustbox}
	\caption{Spectrum of background radiation.}
	\label{fig:background}
\end{figure}
\begin{figure}
	\centering
	\begin{adjustbox}{width=1.4\textwidth, center}
		\includegraphics[scale=0.3]{./build/Europium-Untergrund.pdf}
	\end{adjustbox}
	\caption{$^{152}\text{Eu}$ measurement and scaled spectrum of background radiation.}
	\label{fig:eubackground}
\end{figure}
\noindent
To calibrate the germanium detector, an energy is assigned to each detector channel.
The 10 most prominent full energy peaks are selected and compared to the 10 most probable emission
lines of the known $^{152}\text{Eu}$ spectrum~\cite{laraweb}. The most prominent full enery peaks are marked in
orange in Figure~\ref{fig:spectrumeu} and given in Table~\ref{tab:calibrationvalues}. The prominence
calculation is performed using the algorithm \texttt{find\_peaks} from the python library \texttt{scipy} \cite{scipy}.
A linear fit
\begin{equation}
	E(K) = \alpha \cdot K + \beta
	\label{eqn:linear}
\end{equation}
is performed to get a relation between energy $E$ and channel number $K$. The resulting fit parameters are
\begin{align*}
	\alpha & =  \num{0.207452 \pm 0.000}      \\
	\beta  & =  \num{-1.622490 \pm 0.000381}.
\end{align*}
The fit is displayed in Figure~\ref{fig:efit}.
The relation~\ref{eqn:linear} with the determined parameters is used throughout this analysis for energy measurements. The fit is performed
using the \texttt{iminuit}~\cite{iminuit} python package.
\begin{table}[H]
	\centering
	\caption{Energy and emission probability~\cite{laraweb} of the full energy peaks, their assigned channel number as well as the yield of each peak, determined by a fit.}
	\label{tab:calibrationvalues}
	\sisetup{table-format=4.4}
	\begin{tabular}{S S[table-format=2.2] S[table-format=4.0] S}
		\toprule {$E_{\text{lit}}/\si{\kilo\electronvolt}$} & {$P_{\text{lit}}/\si{\percent}$} & {$C$} & {$N$}                  \\
		\midrule
		121.7817                                            & {28.410 \pm  0.13}               & 595   & {8479.012 \pm 103.393} \\
		244.6974                                            & {7.550 \pm   0.04}               & 1187  & {1597.296 \pm 40.497}  \\
		344.2785                                            & {26.590 \pm  0.12}               & 1666  & {3301.229 ± 60.211}    \\
		443.9650                                            & {2.800 \pm   0.02}               & 2145  & {423.909 ± 24.967}     \\
		778.9045                                            & {12.970 \pm  0.06}               & 3760  & {751.993 ± 29.308}     \\
		867.3800                                            & {4.243 \pm   0.023}              & 4192  & {247.359 ± 24.580}     \\
		964.0790                                            & {14.500 \pm  0.06}               & 4654  & {621.455 ± 26.067}     \\
		1085.8370                                           & {10.130 \pm  0.06}               & 5238  & {469.800 ± 38.635}     \\
		1112.0760                                           & {13.410 \pm  0.06}               & 5370  & {586.025 ± 35.943}     \\
		1408.0130                                           & {20.850 \pm  0.08}               & 6797  & {627.462 ± 26.437}     \\
		\bottomrule
	\end{tabular}
\end{table}
\begin{figure}[H]
	\centering
	% Erste Zeile
	\begin{adjustbox}{width=1.3\textwidth, center}
		\begin{subfigure}{.5\textwidth}
			\centering
			\includegraphics[width=\linewidth]{./build/Europium-Fit-Peak1.pdf}
			% \caption{Plot 1}
		\end{subfigure}%
		\begin{subfigure}{.5\textwidth}
			\centering
			\includegraphics[width=\linewidth]{./build/Europium-Fit-Peak2.pdf}
			% \caption{Plot 2}
		\end{subfigure}
	\end{adjustbox}
	% Zweite Zeile
	\begin{adjustbox}{width=1.3\textwidth, center}
		\begin{subfigure}{.5\textwidth}
			\centering
			\includegraphics[width=\linewidth]{./build/Europium-Fit-Peak3.pdf}
			% \caption{Plot 1}
		\end{subfigure}%
		\begin{subfigure}{.5\textwidth}
			\centering
			\includegraphics[width=\linewidth]{./build/Europium-Fit-Peak4.pdf}
			% \caption{Plot 2}
		\end{subfigure}
	\end{adjustbox}
	% Dritte Zeile
	\begin{adjustbox}{width=1.3\textwidth, center}
		\begin{subfigure}{.5\textwidth}
			\centering
			\includegraphics[width=\linewidth]{./build/Europium-Fit-Peak5.pdf}
			% \caption{Plot 1}
		\end{subfigure}%
		\begin{subfigure}{.5\textwidth}
			\centering
			\includegraphics[width=\linewidth]{./build/Europium-Fit-Peak6.pdf}
			% \caption{Plot 2}
		\end{subfigure}
	\end{adjustbox}
	\caption{Fits of peaks of the $^{152}\text{Eu}$ spectrum.}
\end{figure}
\begin{figure}[H]
	% Vierte Zeile
	\begin{adjustbox}{width=1.3\textwidth, center}
		\begin{subfigure}{.5\textwidth}
			\centering
			\includegraphics[width=\linewidth]{./build/Europium-Fit-Peak7.pdf}
			% \caption{Plot 1}
		\end{subfigure}%
		\begin{subfigure}{.5\textwidth}
			\centering
			\includegraphics[width=\linewidth]{./build/Europium-Fit-Peak8.pdf}
			% \caption{Plot 2}
		\end{subfigure}
	\end{adjustbox}
	% Fünfte Zeile
	\begin{adjustbox}{width=1.3\textwidth, center}
		\begin{subfigure}{.5\textwidth}
			\centering
			\includegraphics[width=\linewidth]{./build/Europium-Fit-Peak9.pdf}
			% \caption{Plot 1}
		\end{subfigure}%
		\begin{subfigure}{.5\textwidth}
			\centering
			\includegraphics[width=\linewidth]{./build/Europium-Fit-Peak10.pdf}
			% \caption{Plot 2}
		\end{subfigure}
	\end{adjustbox}
	\caption{Fits of the 10 most prominent peaks of the $^{152}\text{Eu}$ spectrum.}
\end{figure}
\FloatBarrier


\subsection{The full energy detection probability}
\label{sec:fullenergy}
The amount of photons which fully penetrate and are registered by the detector is called the full energy
detection probability $Q$. It is given by
\begin{equation}
	Q = \frac{4 \pi}{\Omega}\frac{N}{APt}.
	\label{eqn:fullenergy}
\end{equation}
$Q$ depends on the activity of the source $A$, the emission probability $P$, the number of measured photons $N$ and
the measurement time $t$. $N$ and $P$ are given in
table~\ref{tab:calibrationvalues}. $\Omega$ is the solid angle covered by the
detector. For the germanium detector, $\Omega$ is given by
\begin{equation}
	\frac{\Omega}{4 \pi} = \frac{1}{2} \left( 1 - \frac{a}{\sqrt{a^{2} + r^{2}}} \right)
	\label{eqn:solidangle}
\end{equation}
with the detectors radius $r$ and the distance to the probe $a$.

\subsubsection{Determination of the probe activity}
\label{subsubsec:activity}
To determine the full energy detection probability, first the activation of the $^{152}\text{Eu}$ source has to be estimated.
The source had an activation of $A_{0} = \SI{4130 \pm 60}{\becquerel}$ on the 1.10.2000. According to the law of
radioactive decay,
\begin{equation}
	A(t) = A_{0} \cdot \exp{\left( - \frac{\ln{(2)}}{\tau} \cdot t \right)}
	\label{eqn:activation}
\end{equation}
with elapsed time $t$ and half life $\tau = \num{13.522} \, \text{years}$~\cite{laraweb},
the activation on the day of the measurement of $^{152}\text{Eu}$ amounts to
$A(t_{\text{M}}) = \SI{1262 \pm 18}{\becquerel}$.

\subsubsection{Determination of the solid angle}
\label{subsubsec:solidangle}
The solid angle $\frac{\Omega}{4 \pi}$ covered by the germanium detector is given
by equation~\ref{eqn:solidangle}. The detector cylinder has a width of
$\SI{4.5}{\centi\meter}$. The source is located $\SI{6.91}{\centi\meter}$
away from the aluminium cover; the cover itself is $\SI{1.5}{\centi\meter}$
away from the detector. Therefore, the parameters in
equation~\ref{eqn:solidangle} are given by
\begin{align}
	r & =  \SI{2.25}{\centi\meter}  \\
	a & =  \SI{8.91}{\centi\meter}.
	\label{eqn:params}
\end{align}
Using equation~\ref{eqn:solidangle} and the parameters~\ref{eqn:params} the solid angle
amounts to
\begin{equation}
	\frac{\Omega}{4 \pi} = 0.01522.
\end{equation}

\subsubsection{Determination of the full energy detection probability}
\label{subsubsec:determinationfep}
The estimation of the full energy detection probability is performed for
all measured spectral lines.
\begin{table}[H]
	\centering
	\caption{Energy and the calculated full energy detection probability of the 12 peaks of the $^{152}\text{Eu}$ spectrum with more than $\SI{150}{\kilo\electronvolt}$.}
	\label{tab:calibrationvalues}
	\sisetup{table-format=4.4}
	\begin{tabular}{S  S[table-format=1.5] }
		\toprule {$E/\si{\kilo\electronvolt}$} & {$Q/\si{\percent}$}   \\
		\midrule
		{121.7817 \pm 0.0003}                  & {0.05548 \pm 0.00108} \\
		{244.6974 \pm 0.0008}                  & {0.03933 \pm 0.00117} \\
		{344.2785 \pm 0.0012}                  & {0.02308 \pm 0.00055} \\
		{443.965 \pm 0.003}                    & {0.02814 \pm 0.00172} \\
		{778.9045 \pm 0.0024}                  & {0.01078 \pm 0.00045} \\
		{867.38 \pm 0.003}                     & {0.01083 \pm 0.00109} \\
		{964.079 \pm 0.018}                    & {0.00797 \pm 0.00036} \\
		{1085.837 \pm 0.01}                    & {0.00862 \pm 0.00072} \\
		{1112.076 \pm 0.003}                   & {0.00812 \pm 0.00051} \\
		{1408.013 \pm 0.003}                   & {0.00559 \pm 0.00025} \\
		\bottomrule
	\end{tabular}
\end{table}
\noindent
To obtain the energy dependency of full energy detection probability, a fit using the function
\begin{equation}
	Q(E) = a \cdot E^{b}
	\label{eqn:fedpfit_function}
\end{equation}
\noindent
is performed for all measured spectral lines with more than $\SI{150}{\kilo\electronvolt}$,
which is the limit of the detector sensitivity. The result is displayed in Figure~\ref{fig:efit}.
\begin{figure}[H]
	\centering
	% Erste Zeile
	\begin{adjustbox}{width=1.3\textwidth, center}
		\begin{subfigure}{.5\textwidth}
			\centering
			\includegraphics[width=\linewidth]{./build/Europium-Fit.pdf}
		\end{subfigure}%
		\begin{subfigure}{.5\textwidth}
			\centering
			\includegraphics[width=\linewidth]{./build/FEDP-Fit.pdf}
		\end{subfigure}
	\end{adjustbox}
	\caption{The fit of the measured energy and the detector channels as well as the Fit of the full energy detection probabilities for different peaks of the $^{152}\text{Eu}$ spectrum.}
	\label{fig:efit}
\end{figure}
\noindent
The resulting fit parameters are
\begin{align*}
	a & = \num{4.757 \pm 0.347}   \\
	b & = \num{-0.915 \pm 0.013}.
\end{align*}
The relation~\ref{eqn:fedpfit_function} with the determined parameters is used to estimate the full energy detection
probability throughout this analysis.
\noindent
\FloatBarrier

\subsection{Gamma spectrum of \texorpdfstring{$^{137}\mathrm{Cs}$}{caesium}}
A $^{137}\text{Cs}$ probe was measured as well. The measurement was carried out for $t=\SI{2865}{\second}$. The
resulting spectrum is given in Figure~\ref{fig:csspectrum}. The background measurement was again scaled to the
appropriate measurement time and then removed.
\begin{figure}
	\centering
	\begin{adjustbox}{width=1.4\textwidth, center}
		\includegraphics[scale=0.3]{./build/Caesium-Peaks.pdf}
	\end{adjustbox}
	\caption{Spectrum of the $^{137}\text{Cs}$ measurement.}
	\label{fig:csspectrum}
\end{figure}
\noindent
The two peaks marked in Figure~\ref{fig:csspectrum} are determined using \texttt{find\_peaks} from the python library \texttt{scipy} \cite{scipy}. A scaled normal distribution is fitted to the peaks using \texttt{iminuit}~\cite{iminuit}
to determine the yield. The results are displayed in Figure~\ref{fig:csfit}.
\begin{figure}[H]
	\centering
	% Erste Zeile
	\begin{adjustbox}{width=1.3\textwidth, center}
		\begin{subfigure}{.5\textwidth}
			\centering
			\includegraphics[width=\linewidth]{./build/Caesium-Fit-Peak1.pdf}
			% \caption{Plot 1}
		\end{subfigure}%
		\begin{subfigure}{.5\textwidth}
			\centering
			\includegraphics[width=\linewidth]{./build/Caesium-Fit-Peak2.pdf}
			% \caption{Plot 2}
		\end{subfigure}
	\end{adjustbox}
	\caption{Fits of peaks of the $^{137}\text{Cs}$ spectrum.}
	\label{fig:csfit}
\end{figure}
\noindent
These fits include a further background estimation parameter $b$\,; the resulting distribution is given in
equation~\ref{eqn:fitplus}.
\begin{equation}
	g(x) = s \cdot \frac{1}{\sqrt{2 \pi} \sigma} \exp{\left( -\frac{1}{2} \frac{(x - \mu)^{2}}{\sigma} \right)} + b
	\label{eqn:fitplus}
\end{equation}
\noindent
The fit parameters including the number of events of each peak is given in Table~\ref{tab:csfitparams}; the yield of each peak is given by the scale factor $s$.
\begin{table}[H]
	\centering
	\caption{Fit parameters of the two peaks of the $^{137}\text{Cs}$ spectrum with more than $\SI{150}{\kilo\electronvolt}$.}
	\label{tab:csfitparams}
	\sisetup{table-format=3.5}
	\begin{tabular}{S S[table-format=4.3] S[table-format=2.0] S[table-format=4.3] S[table-format=2.3]}
		\toprule {$E/\si{\kilo\electronvolt}$} & {$s$}                 & {$b$}              & {$\mu$}              & {$\sigma$}         \\
		\midrule
		{188.1961 \pm 0.0004}                  & {1275.237 \pm 70.842} & {20.000 \pm 0.457} & {929.282 \pm 1.989}  & {31.257 \pm 2.129} \\
		{661.3941 \pm 0.0004}                  & {9282.567 \pm 98.438} & {15.000 \pm 0.165} & {3195.882 \pm 0.050} & {4.494 \pm 0.038}  \\
		\bottomrule
	\end{tabular}
\end{table}
\noindent
The full width at half maximum and full width at tenth maximum of the second peak (which can be identified as the peak caused by the photo effect)
of the $^{137}\text{Cs}$ spectrum are estimated. To do this, the fitted normal distribution is utilized. This is displayed in Figure~\ref{fig:csfwhm}.
\begin{figure}
	\centering
	\begin{adjustbox}{width=1.4\textwidth, center}
		\includegraphics[scale=0.3]{./build/Caesium-FWHM.pdf}
	\end{adjustbox}
	\caption{FWHM and FWTM of the photoelectric peak of the $^{137}\text{Cs}$ measurement.}
	\label{fig:csfwhm}
\end{figure}
\noindent
The full width at half maximum and the full width at tenth maximum amount to
\begin{align}
	\text{FWHM} & = \num{10.5826 \pm 0.0895} = \SI{0.5729 \pm 0.0186}{\kilo\electronvolt}  \\
	\text{FWTM} & = \num{19.2879 \pm 0.1631} = \SI{2.3788 \pm 0.0338}{\kilo\electronvolt}.
\end{align}
The ratio of the two is given by
\begin{equation}
	\frac{\text{FWHM}}{\text{FWTM}} = \num{0.5487 \pm 0.0000} = \SI{1.5087 \pm 0.0004}{\kilo\electronvolt}.
\end{equation}
The energy of the photoelectric peak is determined from the fit to
\begin{equation}
	E_{\gamma} = \SI{661.3941 \pm 0.0004}{\kilo\electronvolt}.
	\label{eqn:egamma}
\end{equation}
The energy of the compton edge can be calculated using equation~\ref{eqn:emax}
\begin{equation}
	E_{\gamma}^{\prime} \vert_{\theta=\SI{180}{\degree}} = \frac{E_{\gamma}}{1 + \frac{m_{e} c^{2}}{2 E_{\gamma}}}.
	\label{eqn:emax}
\end{equation}
This amounts to
\begin{equation}
	E_{\gamma}^{\prime}\vert_{\theta=\SI{180}{\degree}} = \SI{477.0915 \pm 0.0004}{\kilo\electronvolt}.
\end{equation}
The spectrum, including the theoretical value computed above as well as an estimate based on this measurement, is displayed in Figure~\ref{fig:csshort}.
The compton continuum is estimated and marked in orange.
\begin{figure}
	\centering
	\begin{adjustbox}{width=1.4\textwidth, center}
		\includegraphics[scale=0.3]{./build/Caesium-Peaks-Short.pdf}
	\end{adjustbox}
	\caption{Estimates of the compton edge and continuum of the $^{137}\text{Cs}$ measurement.}
	\label{fig:csshort}
\end{figure}
\noindent
The compton edge is visually estimated to be at $E_{\gamma}^{\prime}\vert_{\theta=\SI{180}{\degree}} = \SI{474.7}{\kilo\electronvolt}$. The energy
region below the compton edge should be dominated by the compton effect and include minimal amounts of background. The
backscatter peak is estimated to be at $E_{\gamma} = \SI{188.1961 \pm 0.0004}{\kilo\electronvolt}$. To estimate the event count
in the compton continuum, a linear fit is carried out in the region between the backscatter peak and the compton edge.
To reduce the influence of the backscatter peak only a certain region from $\SI{278.438 \pm 0.000}{\kilo\electronvolt}$ (marked in
Figure~\ref{fig:csshort} as a dashed line) up to the compton edge (at $\SI{474.7}{\kilo\electronvolt}$) is used.
The fit is displayed in Figure~\ref{fig:comptonfit}.
\begin{figure}[H]
	\centering
	\includegraphics[scale=0.7]{./build/Caesium-Compton-Fit.pdf}
	\caption{Linear Fit of the compton continuum.}
	\label{fig:comptonfit}
\end{figure}
\noindent
The function utilized here is given in Equation~\ref{eqn:comptonfit}
\begin{equation}
	f(x) = \alpha \cdot x + \beta .
	\label{eqn:comptonfit}
\end{equation}
The parameters are estimated to the values
\begin{align}
	\alpha & = \num{0.008 \pm 0.0}    \\
	\beta  & = \num{1.010 \pm 0.784}.
\end{align}
This yield of the compton continuum as estimated by this fit amounts to
\begin{equation}
	N = \num{15498.84830 \pm 0.00000}.
\end{equation}
\FloatBarrier

\subsection{Gamma spectrum of \texorpdfstring{$^{60}\mathrm{Co}$}{cobalt}}
A measurement of the spectrum of a cobalt probe was conducted for a total measurement time of
$t=\SI{4021}{\second}$. Again, the background measurement is subtracted. The algorithm \texttt{find\_peaks}
from the python library \texttt{scipy} \cite{scipy}
is utilized to determine the exact positions of the peaks. The result is displayed in
Figure~\ref{fig:cospectrum}. A comparison to literature values allows the probe to be identified as
$^{60}\text{Co}$~\cite{laraweb}.
\begin{figure}
	\centering
	\begin{adjustbox}{width=1.4\textwidth, center}
		\includegraphics[scale=0.3]{./build/Cobalt-Peaks.pdf}
	\end{adjustbox}
	\caption{Spectrum of the $^{60}\text{Co}$ measurement.}
	\label{fig:cospectrum}
\end{figure}
\noindent
To determine the amount of events belonging to each peak, a fit of the form~\ref{eqn:fitplus2} is carried out.
\begin{equation}
	g(x) = s \cdot \frac{1}{\sqrt{2 \pi} \sigma} \exp{\left( -\frac{1}{2} \frac{(x - \mu)^{2}}{\sigma} \right)} + b
	\label{eqn:fitplus2}
\end{equation}
\noindent
The results are displayed in Figure~\ref{fig:cofit} and the parameters as well as the peak yields are given in
Table~\ref{tab:cofitparams}.
\begin{figure}[H]
	\centering
	% Erste Zeile
	\begin{adjustbox}{width=1.3\textwidth, center}
		\begin{subfigure}{.5\textwidth}
			\centering
			\includegraphics[width=\linewidth]{./build/Cobalt-Fit-Peak1.pdf}
			% \caption{Plot 1}
		\end{subfigure}%
		\begin{subfigure}{.5\textwidth}
			\centering
			\includegraphics[width=\linewidth]{./build/Cobalt-Fit-Peak2.pdf}
			% \caption{Plot 2}
		\end{subfigure}
	\end{adjustbox}
	\caption{Fits of peaks of the $^{60}\text{Co}$ spectrum.}
	\label{fig:cofit}
\end{figure}
\noindent
\begin{table}[H]
	\centering
	\caption{Fit parameters of the two peaks of the $^{60}\text{Co}$ spectrum.}
	\label{tab:cofitparams}
	\sisetup{table-format=3.5}
	\begin{tabular}{S S[table-format=4.3] S[table-format=2.0] S[table-format=4.3] S[table-format=2.3]}
		\toprule {$E/\si{\kilo\electronvolt}$} & {$s$}                & {$b$}              & {$\mu$}              & {$\sigma$}        \\
		\midrule
		{1173.3856 \pm 0.0004}                 & {357.811 \pm 26.149} & {15.000 \pm 0.358} & {5662.087 \pm 0.372} & {4.361 \pm 0.295} \\
		{1332.2939 \pm 0.0004}                 & {246.984 \pm 24.086} & {15.000 \pm 0.477} & {6429.239 \pm 0.502} & {4.376 \pm 0.415} \\
		\bottomrule
	\end{tabular}
\end{table}
\noindent
To estimate the activity of the $^{60}\text{Co}$ source, Equation~\ref{eqn:fullenergy} is used. The full energy
detection probabilities estimated in Section~\ref{sec:fullenergy}. Equation~\ref{eqn:fullenergy} is rearranged
to be able to compute the probe activity:
\begin{equation}
	Q = \frac{4 \pi}{\Omega}\frac{N}{APt} \leftrightarrow A = \frac{4 \pi}{\Omega} \frac{N}{QPt}.
\end{equation}
The resulting activity and the values used are given in Table~\ref{tab:coactivity}.
\begin{table}[H]
	\centering
	\caption{Parameters and result of the activity calculation based on the two peaks of the $^{60}\text{Co}$ spectrum.}
	\label{tab:coactivity}
	\sisetup{table-format=3.5}
	\begin{adjustbox}{width=1.3\textwidth, center}
		\begin{tabular}{S S[table-format=4.4] S[table-format=3.4] S[table-format=3.3] S[table-format=2.4] S[table-format=4.0] S[table.format=1.4]}
			\toprule {$E/\si{\kilo\electronvolt}$} & {$A/\si{\becquerel}$}   & {$N$}                & {$P/\si{\percent}$}  & {$t/\si{\second}$} & {$Q$}               \\
			\midrule
			{1173.3856 \pm 0.0004}                 & {995.4611 \pm 137.5887} & {357.811 \pm 26.149} & {99.85 \pm 0.03}     & {4021}             & {0.0074 \pm 0.0009} \\
			{1332.2939 \pm 0.0004}                 & {770.7859 \pm 118.3582} & {246.984 \pm 24.086} & {99.9826 \pm 0.0006} & {4021}             & {0.0066 \pm 0.0008} \\
			\bottomrule
		\end{tabular}
	\end{adjustbox}
\end{table}
\noindent
The averaged activity then results to
\begin{equation}
	\bar{A} = \SI{883.1235 \pm 116.5034}{\becquerel}.
\end{equation}

\subsection{Spectrum of an unknown source}
A measurement of an unknown uranium source is conducted. The measured spectrum is displayed in
Figure~\ref{fig:urspectrum}; the measurement time is $t=\SI{2968}{\second}$.
\begin{figure}[H]
	\centering
	\begin{adjustbox}{width=1.4\textwidth, center}
		\includegraphics[scale=0.3]{./build/Uran-Peaks.pdf}
	\end{adjustbox}
	\caption{Spectrum of the uranium measurement.}
	\label{fig:urspectrum}
\end{figure}
\noindent
The background measurement is displayed in Figure~\ref{fig:urbackground}.
\begin{figure}[H]
	\centering
	\begin{adjustbox}{width=1.4\textwidth, center}
		\includegraphics[scale=0.3]{./build/Untergrund-Peaks-unskaliert.pdf}
	\end{adjustbox}
	\caption{Spectrum of the background measurement.}
	\label{fig:urbackground}
\end{figure}
\noindent
Through comparing the peaks of both spectra, the ones which likely originate from the uranium probe
can be identified. These peaks are shown in Figure~\ref{fig:urcomp}.
\begin{figure}[H]
	\centering
	\begin{adjustbox}{width=1.4\textwidth, center}
		\includegraphics[scale=0.3]{./build/Uran-Uran-Peaks.pdf}
	\end{adjustbox}
	\caption{Spectrum of the uranium measurement, with background peaks removed.}
	\label{fig:urcomp}
\end{figure}
\noindent
The exact channels of the peaks of the uranium measurement and the background measurement are given in
Table~\ref{tab:urcomp}. A normal distribution is fitted to the remaining peaks to estimate the yield;
the results are given in Figure~\ref{fig:urgauss}.
\begin{table}[H]
	\centering
	\caption{Channels of peaks in the background and uranium spectrum.}
	\label{tab:urcomp}
	\sisetup{table-format=4.0}
		\begin{tabular}{S[table-format=2.0] S S}
			\toprule {Index} & {$C_{\text{Ur}}$} & {$C_{\text{B}}$}              \\
			\midrule
			0 &       60 &    59 \\
			1 &      312 &    {-}   \\
			2 &      379 &   370 \\
			3 &      428 &   {-} \\
			4 &      453 &   453 \\
			5 &      543 &   {-} \\
			6 &      701 &   {-} \\
			7 &      750 &   {-} \\
			8 &      904 &   904 \\
			9 &     1144 &   {-} \\
			10 &    1173 &  1158 \\
			11 &    1256 &   {-} \\
			12 &    1303 &   {-} \\
			13 &    1430 & 	1430 \\
			14 &    1703 & 	1703 \\
			15 &    2943 & 	2946 \\
			16 &    3213 & 	3196 \\
			17 &    3707 &  {-} \\
			18 &    3889 &  {-} \\
			19 &    4508 &  {-} \\
			20 &    5406 &	5410 \\
			21 &    5974 &  {-} \\
			22 &    6647 &  {-} \\
			\bottomrule
		\end{tabular}
\end{table}
\noindent

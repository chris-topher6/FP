\section{Analysis}
\subsection{Detector Calibration}
To calibrate the germanium detector, the spectrum of an $^{152}\text{Eu}$ source is measured;
this can be seen in Figure~\ref{fig:spectrumeu}.
\begin{figure}
	\centering
	\begin{adjustbox}{width=1.4\textwidth, center}
		\includegraphics[scale=0.3]{./build/Europium-Peaks.pdf}
	\end{adjustbox}
	\caption{Spectrum of an $^{152}\text{Eu}$ source.}
	\label{fig:spectrumeu}
\end{figure}
\noindent
To calibrate the germanium detector, an energy is assigned to each detector channel.
The 13 most prominent full energy peaks are selected and compared to the 13 most probable emission
lines of the known $^{152}\text{Eu}$ spectrum~\cite{laraweb}. The most prominent full enery peaks are marked in
orange in Figure~\ref{fig:spectrumeu} and given in Table~\ref{tab:calibrationvalues}. The prominence
calculation is performed using the algorithm \texttt{find\_peaks} from the python library \texttt{scipy} \cite{scipy}.
A linear fit
\begin{equation}
	E(K) = \alpha \cdot K + \beta
	\label{eqn:linear}
\end{equation}
is performed to get a relation between energy $E$ and channel number $K$. The resulting fit parameters are
\begin{align*}
	\alpha = & \num{0.218 \pm 0.000}    \\
	\beta =  & \num{104.803 \pm 0.000}.
\end{align*}
The fit is displayed in Figure~\ref{fig:calibrationfit}.
\begin{figure}
	\centering
	\begin{adjustbox}{width=1.4\textwidth, center}
		\includegraphics[scale=0.3]{./build/Europium-Fit.pdf}
	\end{adjustbox}
	\caption{Fit of the detector channels and corresponding energy.}
	\label{fig:calibrationfit}
\end{figure}
\noindent
The relation~\ref{eqn:linear} with the determined parameters is used thoughout this analysis for energy measurements.
\begin{table}[H]
	\centering
	\caption{Energy and emission probability of the full energy peaks as well as their assigned channel number~\cite{laraweb}.}
	\label{tab:calibrationvalues}
	\sisetup{table-format=4.4}
	\begin{tabular}{S S[table-format=2.2] S[table-format=4.0]}
		\toprule {$E_{\text{lit}}/\si{\kilo\electronvolt}$} & {$P_{\text{lit}}/\si{\percent}$} & {$K$} \\
		\midrule
		121.7817                                            & 28.410                           & 198   \\
		244.6974                                            & 7.550                            & 225   \\
		344.2785                                            & 26.590                           & 595   \\
		411.1165                                            & 2.238                            & 1187  \\
		443.9650                                            & 2.800                            & 1666  \\
		778.9045                                            & 12.970                           & 1989  \\
		867.3800                                            & 4.243                            & 2145  \\
		964.0790                                            & 14.500                           & 3760  \\
		1085.8370                                           & 10.130                           & 4192  \\
		1089.7370                                           & 1.730                            & 4654  \\
		1112.0760                                           & 13.410                           & 5238  \\
		1299.1420                                           & 1.633                            & 5370  \\
		1408.0130                                           & 20.850                           & 6797  \\
		\bottomrule
	\end{tabular}
\end{table}

\subsection{Determination of the full energy detection probability}
The amount of photons which fully penetrate and are registered by the detector is called the full energy
detection probability $Q$. It is given by
\begin{equation}
Q = \frac{4 \pi}{\Omega}\frac{N}{AWt}.
\label{eqn:fullenergy}
\end{equation}
$Q$ depends on the activity of the source $A$, the emission probability $W$, the number of measured photons $N$ and
the measurement time $t$. $\Omega$ is the solid angle covered by the detector. For the germanium detector, $\Omega$ is given
by
\begin{equation}
 \frac{\Omega}{4 \pi} = \frac{1}{2} \left( 1 - \frac{a}{\sqrt{a^{2} + r^{2}}} \right)
 \label{eqn:solidangle}
\end{equation}
with the detectors radius $r$ and the distance to the probe $a$.
To determine the full energy detection probability, first the activation of the $^{152}\text{Eu}$ source has to be estimated.
The source had an activation of $A_{0} = \SI{4130 \pm 60}{\becquerel}$ on the 1.10.2000. According to the law of
radioactivate decay,
\begin{equation}
 A(t) = A_{0} \cdot \exp{\left( - \frac{\ln{(2)}}{\tau} t \right)}
 \label{eqn:activation}
\end{equation}
with the measurement time $t$ of $\SI{46}{\minute}$ $\SI{42}{\second}$ and half life $\tau$
the activation on the day of the measurement of $^{152}\text{Eu}$ amounts to

\section{Conclusion}
The detector calibration shows satisfactory results, which is important as all
other results depend on the energy calculation. Therefore the low uncertainties
of the fit parameters help to reduce uncertainties in the following computations.
The result for the activity of the europium source is of the expected order.
The full energy detection probability calculation follows the expected distribution,
although the results deviate more for lower energies; however, this is expected as the
detector sensitivity is better for higher energy values. The limiting factor in the
analysis of the europium spectrum (as well as all subsequent spectrum analysis) is the
rudimentary background estimation. The results could probably be improved by a significant
amount if a background fit would be performed. A more sophisticated fit distribution for the peaks could
be useful as well, as there are often slight asymmetries in the peaks which the normal distribution
can not account for. Aside from these possible improvements, the caesium peak fits and calculation
of the full and tenth width at half maximum seem plausible. The compton edge can only
be roughly estimated; a longer measurement time should improve the contrast and result
in more accurate values. The estimation of the compton continuum and the influence of
the backscatter peak on it would both benefit greatly from performing additional fits
of the backscatter peak and the background. The cobalt spectrum analysis shows (aside from
the already mentioned shortcomings of the background estimation) satisfactory results.
The identification of the unknown source probably suffers the most by the not ideal background
estimation; the matching of peaks of the measurement by comparing it to the background
measurement could probably be improved. For some fits (e. g. Peak 9) the statistics are
to low to perform a meaningful fit, resulting in compromised results. A longer measurement
time could help mitigate this problem. However, the given nuclides seem to explain the
measured spectrum reasonably well, but it remains unclear if there is another combination
with higher emission probabilities, which are relatively low for some emission lines.

\section{Analysis}
\subsection{Determination of \texorpdfstring{$C_{p}$}{C_p} and \texorpdfstring{$C_{V}$}{C_V}}
The molar heat capacity at constant pressure can be calculated using equation~\ref{eqn:cp}.
\begin{equation}
  C_{p} = \frac{U \cdot I \cdot \Delta t \cdot M}{\Delta T \cdot m}.
  \label{eqn:cp}
\end{equation}
Here, $U$ denotes the heating voltage, $I$ the heating current, $\Delta t$ the heating time interval,
$M$ the molar mass of copper, $\Delta T$ the achieved temperature increase and $m$ the mass of the probe.
The molar heat capacity at constant volume is computed using~\ref{eqn:cv}.
The needed physical constants for these calculations are given in Table~\ref{tab:constants}.
\begin{table}
  \centering
  \caption{Physical constants used.}
  \label{tab:constants}
  \sisetup{table-format=3.3}
  \begin{tabular}{S S S}
    \toprule
    {Constant} & {Value} & {Source} \\
    \midrule
    {$M$} & {$\SI{0.06355}{\kilogram\per\mole}$} & \cite{ciaaw} \\
    {$m$} & {$\SI{0.342}{\kilogram}$} & \cite{V47} \\
    {$\kappa$} & {$\SI{137.8}{\giga\pascal}$} & \cite{pse} \\

    \bottomrule
  \end{tabular}
	
\end{table}

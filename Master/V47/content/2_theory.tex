\section{Theory}
\label{sec:theory}

The heat capacity of a solid is defined as the amount of 
heat $\Delta Q$ that must be added to the solid to raise 
its temperature by $\Delta T$. 
The heat capacity can be expressed as the limit for infinitesimal 
temperature changes
\begin{equation}
    C=\lim_{\Delta T\to 0}\frac{\Delta Q}{\delta T}.
    \label{eqn:C}
\end{equation}
The specific heat capacity refers to the heat capacity per unit 
mass, volume, or amount of substance of the sample. It is 
distinguished by the following definitions
\begin{align*}
    c^n=\frac{C}{n} \qquad
    c^V=\frac{C}{V} \qquad
    c^m=\frac{C}{m}
\end{align*}
where $n$ is the molar amount, $V$ is the volume, and $m$ is 
the mass of the substance, respectively. 

According to the first law of thermodynamics
\begin{equation*}
    \text{d}Q=\text{d}U+p\text{d}V
    \label{eqn:TD1}
\end{equation*}
two different heat capacities can be defined. 
Under the assumption of constant volume ($\text{d}V=0$), 
equation \eqref{eqn:C} yields 
\begin{equation}
    C_V=\left(\frac{\partial Q}{\partial T}\right)_{V=const.}=\left(\frac{\partial U}{\partial T}\right)_V.
    \label{eqn:CV}
\end{equation}
Expressing the inner energy as a function of pressure $p$ and temperature $T$
equation \eqref{eqn:TD1} results in
\begin{equation*}
    \text{d}Q 
    =\left(\frac{\partial U}{\partial T}\right)_p \text{d} T 
    +\left(\frac{\partial U}{\partial p}\right)_T \text{d} p
    +p\left(\frac{\partial V}{\partial T}\right)_p \text{d} T
    +p\left(\frac{\partial V}{\partial p}\right)_T \text{d} p.
\end{equation*}
For an isobaric process ($\text{d}p=0$), the heat capacity at 
constant pressure is given by
\begin{equation}
    C_p=\left(\frac{\partial Q}{\partial T}\right)_{p=const.}
    =\left(\frac{\partial U}{\partial T}\right)_p \text{d} T 
    +p\left(\frac{\partial V}{\partial T}\right)_p \text{d} T.
    \label{eqn:Cp}
\end{equation}
The difference between the heat capacities at constant pressure and volume is
determined to be
\begin{equation}
    C_p-C_V=9\alpha^2KV_0T
\end{equation}
where $\alpha$ is the coefficient of thermal expansion, and $K$ is the bulk modulus. 

While $C_V$ measures the increase in inner energy per degree of 
temperature increase, $C_p$ also includes the energy used for the work 
done due to expansion. As a result, $C_p-C_V>0$. Since gases expand 
significantly more than solids with temperature changes, the difference between 
$C_p$ and $C_V$ is much larger for gases compared to solids.

\subsection{Classical and Quantum Mechanical Description of Temperature}
The oscillations of atoms inside a solid can be approximated by 
harmonic oscillations. 

In a classical system, a harmonic oscillator has a continuous energy spectrum.
Each mode contributes an average energy of $\frac{1}{2}k_BT$. 
In a three-dimensional system with $N$ atoms, 
there are six degrees of freedom — three spatial and three for momentum.
This leads to a total energy of $3Nk_BT$ and, accordingly, 
a heat capacity of $3Nk_B$. 
This is the high-temperature approximation of heat capacity known as the Dulong-Petit law.

In contrast, a quantum mechanical harmonic oscillator has discrete energy levels 
given by $\hbar\omega(n+1/2)$. To excite this oscillator, an energy 
large enough to reach the next level is required, i.e., $k_BT>\hbar\omega$.
For $k_BT<\hbar\omega$, no new modes are excited, leading to a reduction in heat capacity 
as the temperature decreases. This quantum mechanical behavior is key to explaining 
the deviation from the Dulong-Petit law at low temperatures.

\subsection{Einstein Model}
The Einstein model assumes $3N$ eigenmodes, each with the same frequency $\omega_E$.
This leads to an approximated average inner energy of 
\begin{equation}
    \bigl<U\bigr> = 3N \hbar\omega_\text{E} \left(\frac{1}{2} + \text{exp}\left(\frac{\hbar \omega_\text{E}}{k_\text{B}T}\right)
     - 1\right) .
\end{equation}
This yields a heat capacity of 
\begin{equation}
    C_V^\text{E} = 3Nk_\text{B} \left(\frac{\theta_\text{E}}{T}\right)^2 \frac{\text{exp}\left(\frac{\theta_\text{E}}{T}\right)}
    {\left[\text{exp}\left(\frac{\theta_\text{E}}{T}\right) - 1\right]^2} = 
    \begin{cases}
        3Nk_\text{B} \left(\frac{\theta_\text{E}}{T}\right)^2 \text{exp}\left(\frac{-\theta_\text{E}}{T}\right), 
        & T \ll \theta_\text{E} \\
        3Nk_\text{B} , & T \gg \theta_\text{E}
    \end{cases}
\end{equation}
where $\theta_\text{E} = \frac{\hbar\omega_\text{E}}{k_\text{B}}$ is the specific 
Einstein temperature.

This model predicts the Dulong-Petit law at high temperatures. However, 
at low temperatures, it fails to account for the experimentally observed $T^3$ 
decrease in heat capacity. This limitation can be attributed to the fact that 
the model only considers optical modes and neglects acoustic modes.

\subsection{Debye Model}

The Debye model, an improvement over the Einstein model, 
is particularly effective at low temperatures as it incorporates acoustic modes. 
Contrary to the Einstein model, which assumes a uniform oscillation frequency for all atoms, 
the Debye model accounts for a continuous spectrum of vibrations.

In the Debye model, vibrational modes are approximated with a linear dispersion relation 
for low frequencies, described as $\omega = v_i k$. 

The state density function in the Debye model is given by
\begin{equation}
    D(\omega) = \frac{Vk^2}{2\pi^2v} = \frac{V\omega^2}{2\pi^2 v^3}.
    \label{eqn:sdf}
\end{equation}

The heat capacity at constant volume is derived by integrating over all vibrational modes
\begin{equation}
    C_V^\text{E} = 9Nk_\text{B} \left(\frac{T}{\theta_\text{D}}\right)^3 \int_0^{\frac{\theta_\text{D}}{T}} 
    \frac{x^4\text{e}^x \; \text{d}x}{(\text{e}^x - 1)^2} = 
    \begin{cases}
        \frac{12\pi^4}{5}Nk_\text{B}\left(\frac{T}{\theta_\text{D}}\right)^3, & T \ll \theta_\text{E} \\
        3Nk_\text{B} , & T \gg \theta_\text{E}
    \end{cases}
\end{equation}
where  \( x = \hbar\omega / k_\text{B}T \), and
$\theta_\text{D}$ is the material-specific Debye temperature, defined as 
\begin{equation}
    \theta_\text{D} = \frac{\hbar\omega_\text{D}}{k_\text{B}} = \
    \frac{\hbar v}{k_\text{B}} \left(6\pi^2\frac{N}{V}\right)^{\frac{1}{3}}.
    \label{eqn:Debye}
\end{equation}
The Debye model accurately predicts the $T^3$ dependence of heat capacity at low temperatures, 
consistent with experimental observations. 
At high temperatures, it aligns with the Dulong-Petit law, similar to the Einstein model.

Integrating the state density function \eqref{eqn:sdf} from 
$\omega=0$ to the Debye frequency $\omega_D$ gives
\begin{equation}
    \int_0^{\omega_D}D(\omega)\text{d}\omega=3N
\end{equation}
resulting in the Debye frequency
\begin{equation}
    \omega_D=v_i\left(6\pi^2\frac{N}{V}\right).
\end{equation}
The speed of sound $v_s$ can be determined from the speeds $v_i$ for each mode $i$ using
\begin{equation}
    \frac{1}{v_s^2}=\frac{1}{3}\sum_{i=1}{^3}\frac{1}{v_i^3}.
\end{equation}
\section{Discussion}
The measurement for the most part displays satisfactory results. The specific heat
capacities show the expected relation $C_{p} - C_{V} > 0$ (see Section~\ref{sec:theory}).
The values for different temperatures are close to the spectrum shown in the literature, e. g.~\cite{Arblaster2015}.
The Debye temperature can be calculated theoretically using Equation~\ref{eqn:Debye}.
For the copper probe used, this results in
\begin{equation}
 \theta_{D, \mathrm{T}} = \SI{332.6}{\kelvin}.
\end{equation}
The experimentally determined value differs by $\SI{37.1}{\percent}$.
However, some measurements seem to be faulty, which can be seen in Figure~\ref{fig:results} where
prominent outliers are produced in the calculation of the specific heat capacities.
Additionally, the method used to calculate the Debye temperature $\theta_{D}$ seems to result in nonsensical negative
values. The cause of this appears to be the interpolation; the full influence of which should be investigated further.

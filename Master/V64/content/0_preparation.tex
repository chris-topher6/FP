\section{preparation}

\subsection*{What does coherence mean?}
\begin{itemize}
    \item phaserelation of waves
\end{itemize}
\subsection*{Familiarise yourself with the terms on degree of coherence, temporal coherence an spatial coherence.}
\begin{itemize}
    \item degree of coherence: measure of the coherence of light from 0 (completely incoherent) to 1 (perfectly coherent)
    \item temporal: consistency of phase over time. This indicates how long the wave behaves predictably in terms of its phase and amplitude.
    \item spatial: consistency of phase over space. High spatial coherence means that the phase of the light wave is consistent across different areas of the wavefront.
\end{itemize}

\subsection*{What are the different polarisation properties of light?}
\begin{itemize}
    \item Linear Polarization: The electric field of light oscillates in a single plane.
    \item Circular Polarization: The electric field of light rotates in a circle, creating a spiral pattern as the light travels.
    \item Elliptical Polarization: A general form of polarization where the electric field describes an ellipse in its oscillation pattern.
\end{itemize}
\subsection*{How must two light beams be polarised in relation to each other so that interference effects can be observed?}
\begin{itemize}
    \item Parallel Polarization: The polarization direction of both beams should be parallel or nearly parallel for effective interference
\end{itemize}

\subsection*{How can the contrast (also visibility) of an interferometer be defined?}
\begin{itemize}
    \item Contrast Formula: Defined as the ratio of the difference between maximum and minimum intensities to the sum of these intensities in the interference pattern.
    \item $K=\frac{I_\text{max}-I_\text{min}}{I_\text{max}+I_\text{min}}$
\end{itemize}

\subsection*{How does a polarising beam-spliter cube (PBSC) work?}
\begin{itemize}
    \item Splits light by polarisation
    \item reflects ine polarisation while transmittung the orthogonal polarisation
\end{itemize}
\subsection*{Which orientation has the polarisation of the outcoming light beam of the PBSC, if the polarisation of the incoming light beam lies in the vertical and the PBSC is titled at $\SI{45}{\degree}$ away from the vertical?}
\begin{itemize}
    \item The outgoing light beam's polarization will be oriented at an angle of $\SI{45}{\degree}$ relative to the vertical. This is due to the PBSC's property of splitting and reflecting light based on its polarization direction. Since the PBSC is tilted at $\SI{45}{\degree}$, it changes the orientation of the reflected polarization by the same angle.
\end{itemize}

% Am besten auf Papier
\subsection*{Show that the maximum intensity $I_\text{max}$ in case of constructive interference respectively the minimal intensity $I_\text{min}$ in case of deconstructive interferene, detected by a diode is given by:}
\begin{equation*}
    I_\text{max/min}\propto I_\text{Laser}(1\pm2\cos\phi\sin\phi)
\end{equation*}
\begin{itemize}
    \item 
\end{itemize}
\subsection*{Explain why this assumption about the detected diode signal in justified.}
\subsection*{What is the significance of the variable $\delta$?}
\subsection*{What is the relationship between $I_\text{Laser}$, $\phi$ and the electric field component $E_i$}

\subsection*{How does the contrast depend on the polarisation angle?}
\begin{itemize}
    \item perfect interference requires the same polarisation of both beams 
    \item a polarisation angle difference worsens the contrast
\end{itemize}

\subsection*{Which kind of measurement uses one and which two diodes?}
\begin{itemize}
    \item 
\end{itemize}
\subsection*{What is the advantage of measuring the differential voltage of both diodes (differential voltage method)?}
\begin{itemize}
    \item more precise measurement
    \item cancellation of noise or fluctuations that affect both diodes
\end{itemize}

\subsection*{How can you determine the refraction index $n$ depend on temperature $T$ and the pressure $p$ of a gas (Lorentz-Lorenz-law)?}
\begin{itemize}
    \item The refractive index nn of a gas as a function of temperature $T$ and pressure $p$ can be determined using the Lorentz-Lorenz law. 
    \item $\frac{n^2-1}{n^2+2}=\frac{K\rho}{3}$
    \item $\rho=\frac{pm}{k_BT}$: gas density 
\end{itemize}
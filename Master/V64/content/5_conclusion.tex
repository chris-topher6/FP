\section{Discussion}
%\begin{itemize}
%    \item contrast $K<1$, why?\\
%    \to maximum sometimes hard to find \\
%    \to no perfect interference, alignment
%    \item contrast fit, high deviations \\
%    \to sys. uncertainties
%    \item refraction index glas, okay, difficult to find literatur value \\ 
%    \to different glass sorts 
%    \item refraction index gas, okay \\
%    \to uncertainties meight be to low, sys. uncertainties \\
%    \to humidity of air unknown
%    \to Zusammensetzung der Raumluft unklar
%\end{itemize}

The measurement of the contrast leads to the expected values. 
The offset angle of $\delta=\SI{2.27\pm0.33}{\degree}$ could be 
explained with a constant offset of the polariser or an offset 
of the lasers polarisation angle. 
The fact that the amplitude $K_0=\num{0.895\pm0.015}$ of the 
contrast function is below the theoretical value of one 
indicates a not perfect alignment of the interferometer. 

One problem of the measurement is the fit of the theory function 
\eqref{eqn:K_ana} to the data. The high residuals of up to $\SI{50}{\sigma}$
indicates that either the uncertainties are underastimated or that 
the theory does not fit to the data with due to effects which are not 
considered in the theory. Since the values of $I_\text{min}$ and 
$I_\text{max}$ were sometimes hard to measure, because the intensity 
distribution is sometimes sharp and the device which was used to 
measure the values has in intrensic systematic uncertaintie and 
the values jumpt during the measurement, it can be assumed, that 
the uncertainties in the fit must be larger. This would 
decrease the values of the residuals. 

The refraction index of glass is measured to be $n=\num{1.485\pm0.0013}$.
Since there exist various different sorts of glass and the 
sort of the examined glass is unknown, it is not possible to 
compare the measurement to a literatur value. Most glass sorts 
have a refraction index which lies between $\numrange{1.4}{1.9}$
\cite{glass}. This is compatible with the measurement. 

The refraction index of the gas in the room at standard 
atmospheric conditions is $n_\text{gas}=\num{1.0002734\pm 0.0000015}$.
The literatur value for air at standard conditions is given as 
$n_\text{air}=\num{1.00029}$ \cite{air}. The deviation is small, 
but because of the small uncertaintie, but significant. 
But since the exact composition of the gas in the room and the 
humidity is unknown, the deviation explainable. 
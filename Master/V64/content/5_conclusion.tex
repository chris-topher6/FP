\section{Discussion}

The contrast measurement yielded values within 
the expected range. The offset angle of 
$\delta = \SI{2.27 \pm 0.33}{\degree}$ could be 
attributed to a constant offset in the polarizer 
or an offset in the laser's polarization angle. 
The amplitude of the contrast function, 
$K_0 = \num{0.895 \pm 0.015}$, being below the 
theoretical value of one, suggests an imperfect 
alignment of the interferometer.

A notable issue in the measurement was fitting the 
theoretical function \eqref{eqn:K_ana} to the data. 
The high residuals, reaching up to $\num{50}\,\sigma$, 
indicate that either the uncertainties were 
underestimated or that the theory does not fully 
account for certain effects present in the data. 
Given the challenges in measuring $I_{\text{min}}$ 
and $I_{\text{max}}$, such as the sharp intensity 
distribution and intrinsic systematic uncertainties 
of the measuring device, which caused fluctuations 
during the measurement, it is plausible that the 
uncertainties in the fit are larger than estimated. 
This would, in turn, reduce the magnitude of the 
residuals.

The refractive index of glass was measured to be 
$n = \num{1.485 \pm 0.0013}$. Due to the variety 
of glass types and the unknown specific type of 
glass examined, a direct comparison with literature 
values is not feasible. However, most types of glass 
have a refractive index ranging between 
$\numrange{1.4}{1.9}$ \cite{glass}, which is consistent 
with our measurement.

The refractive index of the gas in the room, under 
standard atmospheric conditions, was determined to 
be $n_{\text{gas}} = \num{1.0002734 \pm 0.0000015}$. 
The literature value for air under standard conditions 
is $n_{\text{air}} = \num{1.00029}$ \cite{air}. While 
the deviation is small, it is significant due to the 
low uncertainty. However, this deviation can be 
explained by the unknown exact composition of the gas 
in the room and the humidity levels.
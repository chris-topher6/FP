\section{Analysis}
\subsection{Determination of the Contrast in dependency of the polarisation angel}

As described in section \ref{sec:contrast} the minimum and maximum for 
different polarisation angles are measured. The measured values are 
listed in Table \ref{tab:contrast}.

\begin{table}
    \centering
    \begin{tabular}{S S S S S}
        \toprule
        {$\phi/\si{\degree}$} & {$I_\text{min1}/\si{\volt}$} & {$I_\text{max1}/\si{\volt}$} & {$I_\text{min2}/\si{\volt}$} & {$I_\text{max2}/\si{\volt}$}\\
        \midrule
        0   &  2.02 & 1.75 & 2.12 & 1.80 \\
        5   &  1.82 & 1.55 & 1.92 & 1.53 \\
        10  &  1.82 & 1.26 & 1.87 & 1.23 \\
        15  &  1.91 & 0.99 & 1.88 & 0.95 \\
        20  &  1.73 & 0.74 & 1.78 & 0.69 \\
        25  &  1.64 & 0.52 & 1.63 & 0.51 \\
        30  &  1.59 & 0.33 & 1.56 & 0.43 \\
        35  &  1.51 & 0.27 & 1.51 & 0.31 \\
        40  &  1.55 & 0.22 & 1.61 & 0.25 \\
        45  &  1.61 & 0.18 & 1.51 & 0.20 \\
        50  &  1.73 & 0.16 & 1.76 & 0.18 \\
        55  &  1.88 & 0.15 & 1.94 & 0.17 \\
        60  &  2.05 & 0.19 & 2.10 & 0.20 \\
        65  &  2.26 & 0.28 & 2.31 & 0.26 \\
        70  &  2.37 & 0.37 & 2.19 & 0.41 \\
        75  &  2.34 & 0.60 & 2.31 & 0.56 \\
        80  &  2.50 & 0.86 & 2.38 & 0.88 \\
        85  &  2.47 & 1.37 & 2.45 & 1.22 \\
        90  &  2.31 & 1.77 & 2.25 & 1.62 \\
        95  &  2.65 & 1.79 & 2.58 & 1.77 \\
        100 &  3.23 & 1.54 & 3.27 & 1.53 \\
        105 &  3.77 & 1.28 & 3.68 & 1.32 \\
        110 &  4.12 & 1.06 & 4.13 & 1.03 \\
        115 &  4.50 & 0.72 & 4.63 & 0.70 \\
        120 &  5.18 & 0.49 & 5.28 & 0.47 \\
        125 &  5.89 & 0.29 & 5.73 & 0.35 \\
        130 &  6.09 & 0.24 & 6.07 & 0.25 \\
        135 &  5.87 & 0.27 & 6.03 & 0.26 \\
        140 &  5.95 & 0.37 & 5.96 & 0.35 \\
        145 &  5.75 & 0.55 & 5.65 & 0.61 \\
        150 &  5.46 & 0.68 & 5.71 & 0.83 \\
        155 &  5.00 & 0.97 & 5.12 & 1.01 \\
        160 &  4.64 & 1.31 & 4.69 & 1.34 \\
        165 &  3.72 & 1.48 & 3.78 & 1.55 \\
        170 &  3.21 & 1.60 & 3.29 & 1.72 \\
        175 &  2.47 & 1.79 & 2.70 & 1.85 \\
        180 &  2.12 & 1.79 & 2.15 & 1.81 \\
        \bottomrule
    \end{tabular}
    \caption{Measured minimal and maximal intensity for different polarisation angles $\phi$.}
    \label{tab:contrast}
\end{table}
These values are used to calculate via Equation \eqref{eqn:contrast} the 
Kontrast $K$ for both measurements. After that, the mean and the standarddiviation 
of $K$ is calculated using \textit{numpy} \cite{numpy}. Figure \ref{fig:contrast}
visualises these values of $K$. 
\\
For the fit the function \eqref{eqn:contrast2} is modified with an constant off set angle $\delta$
and a factor $K_0$ which describes the maximal contrast. Both parameters are free to variie in 
the fit and compansate deviations from the experimental setup to the theory.
The resulting function is 
\begin{equation}
    K = 2 K_0|\sin{(\phi-\delta)}\cos{(\phi-\delta)}|
    \label{eqn:K_ana}
\end{equation}
The fit, which was done with \textit{scipy} \cite{scipy} is displayed in Figure \ref{fig:contrast}. 
The values of the parameters are 
\begin{align*}
    K_0=\num{0.895\pm 0.015} \quad\text{and}\quad \delta=\SI{2.27\pm0.33}{\degree}.
\end{align*}

\begin{figure}
    \centering 
    \includegraphics[width=0.8\textwidth]{build/contrast.pdf}
    \caption{Averaged contrast $K$ in dependency of the polarisation angle $\phi$ with a fit of the function \eqref{eqn:K_ana}.}
    \label{fig:contrast}
\end{figure}

The maximum contrast was measured to be at $\phi=\SI{130}{\degree}$. From now on the polarisation
angle is set to this value.

\subsection{Determination of the refraction index of glass}
For the measurement of the refraction index two glas planes are 
placed into the beams of the interferometer as described in 
section \ref{sec:measurement_glass}. Since the glass planes are 
tilted by $\theta_0=\pm\SI{10}{\degree}$, Equation \eqref{eqn:delta}
can be simplified with $\theta_1=-\theta_2=10$ to 
\begin{equation}
    \Delta\Phi(\theta)=2\pi\frac{T}{\lambda_\text{vac}}\frac{n-1}{n}\cdot 2\theta_0\theta.
\end{equation}
This results with the relation \eqref{eqn:M} in 
\begin{equation}
    M=\frac{T}{\lambda_\text{vac}}\frac{n-1}{n}\cdot 2\theta_0\theta
    \label{eqn:M_ana}
\end{equation}
with the wavelength $\lambda=\SI{632.99}{\nano\metre}$ of the laser and 
the thickness $T=\SI{1}{\milli\metre}$ of the glass planes. Table \ref{tab:glass}
shows the measured values for $M$ for different $\theta$. Figure \ref{fig:glass}
shows the fit of the function \eqref{eqn:M_ana} and the means of $M$ with the 
standarddiviation as errorbars. 

\begin{table}
    \centering
    \begin{tabular}{S S S S S S S S S S}
        \toprule
        {$\theta/\si{\degree}$} & {$M_1$} & {$M_2$} & {$M_3$} & {$M_4$} & {$M_5$} & {$M_6$} & {$M_7$} & {$M_8$} & {$M_9$}\\
        \midrule
        2 & 6  & 5  & 7  & 5  & 7  & 6  & 7  & 8  & 9  \\
        4 & 12 & 9  & 13 & 13 & 13 & 11 & 11 & 15 & 14 \\
        6 & 19 & 15 & 21 & 16 & 20 & 17 & 20 & 22 & 20 \\
        8 & 27 & 16 & 25 & 23 & 24 & 25 & 25 & 26 & 25 \\
        10 & 33 & 22 & 36 & 36 & 33 & 35 & 33 & 34 & 33 \\
        \bottomrule
    \end{tabular}
    \caption{Measured values of the numbers of maxima $M$ which passed the center for different angles $\theta$.}
    \label{tab:glass}
\end{table}

\begin{figure}
    \centering 
    \includegraphics[width=0.8\textwidth]{build/n_glass.pdf}
    \caption{Averaged numbers of maxima $M$ in dependency of the angle $\theta$ with a fit of the function \eqref{eqn:M_ana}.}
    \label{fig:glass}
\end{figure}

The measured value of the refraction index results in 
\begin{align*}
    n=\SI{1.485\pm 0.013}.
\end{align*}

\subsection{Determination of the refraction index of gas}
The measured values for the number of maxima $M$ which passed the center 
for different pressures are listed in Table \ref{tab:gas}. 
Figure \ref{fig:gas} visualises the averaged values of $M$ in dependency
of the pressure. 

In the case of $n\approx \num{1}$ the Lorentz-Lorenz law \eqref{eqn:LLL}
can be simplified as 
\begin{equation}
    n=\frac{3}{2}\frac{Ap}{RT}+1.
    \label{eqn:n_gas}
\end{equation}
With the measured room temperautre $T_\text{room}=\SI{21.8}{\celsius}$, $L=\SI{100.0\pm0.1}{\milli\metre}$
this results in a linear function 
\begin{equation}
    n(p)=\frac{3}{2}\frac{p}{RT_\text{room}\cdot a+b}
\end{equation}
where the parameters $a$ and $b$ are fitted to the data. The fit is showed 
in Figure \ref{fig:gas}. The values of the parameters are 
\begin{align*}
    a=\num{4.44 \pm 0.02} \quad\text{and}\quad b=\num{0.9999987 \pm 0.0000008}.
\end{align*}
Plugging this in Equation \eqref{eqn:n_gas} with a temperautre of 
$T=\SI{15}{\degree}$ and $p=\SI{1013}{\hecto\pascal}$ leads to 
the measured refraction index of air at standard atmosphere 
\begin{equation*}
    n_\text{gas}=\SI{1.0002734\pm 0.0000015}.
\end{equation*}

\begin{table}
    \centering
    \begin{tabular}{S S S S S S}
        \toprule
        {$p/\si{\milli\bar}$} & {$M_1$} & {$M_2$} & {$M_3$} & {$M_4$} & {$M_5$}\\
        \midrule
        50   & 2  &  1 &  2 &  2 &  2 \\ 
        100  & 4  &  3 &  4 &  4 &  4 \\ 
        150  & 6  &  6 &  6 &  7 &  7 \\ 
        200  & 9  &  8 &  9 &  9 &  9 \\
        250  & 11 & 10 & 11 & 11 & 11 \\
        300  & 13 & 12 & 13 & 13 & 13 \\
        350  & 15 & 14 & 15 & 15 & 15 \\
        400  & 17 & 16 & 18 & 17 & 17 \\
        450  & 19 & 18 & 20 & 19 & 19 \\
        500  & 21 & 20 & 22 & 21 & 21 \\
        550  & 23 & 22 & 24 & 23 & 23 \\
        600  & 25 & 25 & 26 & 25 & 26 \\
        650  & 28 & 27 & 29 & 28 & 28 \\ 
        700  & 30 & 29 & 31 & 30 & 30 \\ 
        750  & 32 & 31 & 33 & 32 & 32 \\
        800  & 34 & 33 & 35 & 34 & 34 \\
        850  & 36 & 35 & 37 & 36 & 36 \\
        900  & 38 & 37 & 40 & 38 & 38 \\
        950  & 40 & 40 & 43 & 40 & 41 \\
        1000 & 43 & 42 & 45 & 43 & 43 \\
        \bottomrule
    \end{tabular}
    \caption{Measured values of the numbers of maxima $M$ which passed the center for different gas pressures $p$.}
    \label{tab:gas}
\end{table}

\begin{figure}
    \centering 
    \includegraphics[width=0.8\textwidth]{build/n_air.pdf}
    \caption{Averaged numbers of maxima $M$ in dependency of the gas pressure $p$ with a fit of the function \eqref{eqn:M_gas}.}
    \label{fig:gas}
\end{figure}
